\documentclass[a4paper,12pt,twoside]{book}
\usepackage{tfg}
\usepackage{appendix}
\begin{document}

\pagestyle{empty}
Portada
// TO DO //
\newpage
\chapter*{Agradecimientos}
Agradecimientos
// TO DO //
\newpage
\chapter*{Resumen}
Se realizará una aplicación que permite la organización de visitas a diferentes lugares de
una localidad mediante la creación de rutas entre los mismos. Se implementará un
servicio web en Java bajo el paradigma Rest que sirva de API tanto a la aplicación web
como móvil para el acceso a los datos. Los datos de los lugares y sitios que un usuario
pueda buscar serán obtenidos por el servicio web a través de una fuente externa. Los
usuarios podrán consultar los diferentes lugares que existan en una localidad y
mostrarlos en un mapa, para luego, seleccionar los que el usuario desee visitar. Entre
los lugares que el usuario desea visitar se generará una ruta. En dicha ruta, el usuario
podrá especificar el tiempo de comienzo (día y hora que va estar en esa localidad y que
vaya a comenzar la ruta), el tiempo que desea o estima pasar en cada sitio, el orden en
el que los desea visitar, así como el tipo de desplazamiento que va realizar entre cada
sitio (coche, andando, etc…). La aplicación mostrará el tiempo y la distancia que hay
entre cada lugar, así como el total. De esta forma el usuario podrá saber cuánto tiempo
le llevará el viaje y podrá realizar modificaciones en función de sus necesidades.
Además, en la aplicación se podrán registrar eventos temporales (Ejemplo: Feria
gastronómica – Madrid – 20 de Enero) que se integrarán con los lugares ofrecidos por la
fuente externa y que permitirá al usuario incluirlos en sus rutas (si coincidiese en espacio
y tiempo). Los usuarios podrán compartir sus rutas guardadas con el resto de usuarios y
a mayores, en la aplicación móvil se podrá consultar a tiempo real la realización de la
ruta, mediante el uso de la geolocalización, que permitirá al usuario saber si está
cumpliendo o no sus estimaciones indicadas a la hora de la creación de la ruta.
La aplicación hará uso de la API de Foursquare para el acceso a los datos de lugares y
empleará la API de Google para los cálculos de distancias y tiempos, así como para la
visualización en los
mapas. 
\newpage
\chapter*{Palabras Clave}
// TO DO //
\newpage

\setcounter{page}{1}
\cleardoublepage
\tableofcontents
\cleardoublepage
\listoffigures
\cleardoublepage
\listoftables


\setcounter{page}{1}
\chapter[Introducción]{
  \label{chp:introduccion}
  INTRODUCCIÓN
}
\thispagestyle{numberingStyle}
\pagestyle{numberingStyle}


\section{Contextualización}
Durante las últimas décadas, el turismo ha experimentado un continuo crecimiento que lo ha llegado a convertirse en uno de los sectores económicos más importantes. Por su parte, el éxito de la telefonía móvil y el continuo uso de smartphones en la sociedad, promulga el desarrollo de las aplicaciones móviles, otro sector que continúa en crecimiento.

El desarrollo de está aplicación está englobado entre los dos sectores anteriormente comentados. Se pretende ofrecer a los usuarios finales una aplicación móvil, fácilmente accesible, que permita ofrecer un servicio que ayude a los usuarios a planificar rutas turísticas en sus viajes. A parte de la planificación de rutas, se ofrecerá la posibilidad de descubrir diferentes lugares del mundo así como explorar los eventos que haya disponibles 

\section{Objetivos}

\section{Estructura de la memoria}



\chapter[Análisis de alternativas]{
  \label{chp:analisisdealternativas}
  ANÁLISIS DE ALTERNATIVAS
}
\thispagestyle{numberingStyle}
\pagestyle{numberingStyle}



Antes de comenzar el desarrollo de este proyecto, se realizó un breve estudio de algunas de las aplicaciones, de temática similar, disponibles en el mercado. 

La primera en ser analizada ha sido la propia aplicación de Foursquare que, como se comentó con anterioridad, será la fuente de datos de lugares de la aplicación. La aplicación de Foursquare, tanto en su versión web como móvil, ofrece la posibilidad de buscar lugares, sitios de interés y demás, según diferentes criterios, y que pueden ser guardados por el usuario en lo que la aplicación denomina listas. Sin embargo, Foursquare no permite establecer ningún tipo de planificación ni ruta entre dichos sitios.

Por otra parte, existen numerosas aplicaciones enfocadas principalmente a la planificación de rutas. Una de las más populares es la aplicación móvil Google Trips: Travel Planner, aplicación del grupo Google. Está aplicación ofrece gran nivel de detalle en lo que se refiere a la planificación de viajes y se puede considerar una aplicación todo en uno. Permite crear viajes en función de las reservas remitidas al correo electrónico de Google del usuario, añadir planes de viaje diarios, mostrar información sobre el transporte disponible, ofrecer los diferentes descuentos disponibles a determinados lugares, como pueden ser museos, y demás funcionalidades relacionadas.

Por su contra, a la hora de establecer los planes diarios, esta aplicación no permite una completa personalización. El modo de transporte utilizado será escogido por la propia aplicación, siendo este el más eficiente en tiempo. Por otra parte, te recomienda el tiempo a pasar en cada visita en función de la media de los demás usuarios, pero no ofrece la posibilidad de establecer un tiempo específico. Estas limitaciones impiden conocer el tiempo total que empleará el usuario en el plan de ese día, a que horas debería salir de un determinado sitio para llegar a otro antes de una hora específica, modificar el modo de viaje en función de sus necesidades, y demás.


A parte de las aplicaciones mencionadas, existe gran variedad de ellas que ofrecen una funcionalidad principal similar, la planificación de viajes turísticos. Cada una de ellas presenta características diferentes, como pueden ser, la fuente de datos utilizada, el nivel de detalle en la planificación, el número de funcionalidades ofrecidas, etc. Estas características y la manera de cómo ofrecerlas, serán las que hagan únicas a las aplicaciones en el mercado y que harán que los usuarios se decanten por el uso de unas u otras.














\chapter[Fundamentos tecnológicos]{
  \label{chp:fundamentos}
  FUNDAMENTOS TECNOLÓGICOS
}

\thispagestyle{numberingStyle}
\pagestyle{numberingStyle}


\section{Lenguajes utilizados}
\subsection{Java}
Java es un lenguaje de programación de propósito general, concurrente, orientado a objetos. Fue diseñado para tener tan pocas dependencias de implementación como fuera posible tal que permitiera a los desarrolladores escribir el programa una vez y ejecutarlo en cualquier dispositivo sin necesidad de recompilarlo.

Fue originalmente desarrollado por James Gosling, de Sun Microsystems, la cual fue adquirida por la compañía Oracle.

Puede ejecutarse en cualquier máquina virtual Java (JVM) sin importar la arquitectura de la computadora subyacente y su sintaxis deriva en gran medida de lenguajes como C y C++, pero con menos utilidades de bajo nivel.

\subsection{JavaScript}
TypeScript es un lenguaje de programación interpretado, dialecto del estándar ECMAScript, orientado a objetos, basado en prototipos, imperativo, débilmente tipado y dinámico.

Se utiliza principalmente del lado del cliente, implementado como parte de un navegador web, permitiendo mejoras en la interfaz de usuario y páginas web dinámicas.

JavaScript se diseñó con una sintaxis similar a C, aunque adopta nombres y convenciones del lenguaje de programación Java. Sin embargo, Java y JavaScript tienen semánticas y propósitos diferentes.

\subsection{TypeScript}
TypeScript es un lenguaje de programación libre y de código abierto desarrollado y mantenido por Microsoft.

Este lenguaje es un superset del ya conocido JavaScript y que está pensado para grandes proyectos, los cuáles a traves de un compilador de TypeScript se traducen a código JavaScript original.

Un aspecto característico de TypeScript es su sistema de tipos. Permite a los desarrolladores definir variables y funciones tipadas sin perder la esencia de JavaScript gracias a una representación estática de los tipos dinámicos. Definir tipos durante el diseño, nos ayudará a evitar errores en tiempo de ejecución.

\subsection{HyperText Markup Language}
HyperText Markup Language, conocido comúnmente por sus siglas, HTML, es un lenguaje de marcado cuya finalidad es la elaboración de páginas web. Define una estructura básica y un código (denominado código HTML) para la definición de contenido de una página web. 

Es un estándar a cargo del World Wide Web Consortium (W3C), organización dedicada a la estandarización de casi todas las tecnologías ligadas a la web.

\subsection{Cascading StyleSheets}
Las hojas de estilo en cascada (o CSS, por sus siglas en ingles.) es un lenguaje de diseño gráfico para definir y crear la presentación de un documento estructurado escrito en un lenguaje de marcado. Especifica como se mostrarán por pantalla, o otro tipo de media, los denominados elementos HTML.

Junto con HTML y JavaScript, CSS es una tecnología usada por muchos sitios web para crear páginas web visualmente atractivas, interfaces de usuario de aplicaciones web y GUIs para muchas aplicaciones móviles.

Su principal objeto es mantener la separación del contenido del documento de su forma de presentación. Con las hojas de estilo se puede prescindir del uso de formatos de estilo dentro de la propia página HTML, de manera que se pueda modificar el estilo de toda una web modificando un único archivo CSS.




\section{Frameworks, librerías y técnicas de desarrollo}
\subsection{Spring}
Spring es un framework cuya finalidad es facilitar el desarrollo de aplicaciones desarrolladas en Java. Es de código abierto y la primera versión fue elaborada por Rod Johnson. A pesar de que no impone ningún modelo de programación en particular, este framework se ha vuelto popular en la comunidad al ser considerado una alternativa, sustituto, e incluso un complemento al modelo Enterprise JavaBean (EJB)

Spring está compuesto de diversos módulos que se pueden agregar a nuestras aplicaciones, permitiendo a los desarrolladores agregar sólo los módulos que vayan usar. El único módulo necesario para trabajar con Spring es el Spring Core puesto que es el que contiene la DI (Inyección de Dependencias) y la configuración de uso de objetos Java.

\subsubsection*{Spring MVC}
Spring MVC es un framework de aplicaciones web basado en el patrón MVC(model-viewcontroller) y que alberga todas las ventajas del framework de Spring.
	\begin{itemize}
		 \item Separación clara de roles. Cada objeto controlador, validador, formulario, de modelo pueden ser realizados por objetos especializados.
		 \item Configuración potente y directa. Capacidad de configuración que permite una fácil referencia a través de contextos, como por ejemplo, desde controladores web a objetos de negocio.
		 \item Adaptabilidad, flexibilidad y no intrusividad. Definir un controlador usando une de las anotaciones de parámetros (@RequestParam, @PathVariable,…) para un escenario dado.
		 \item Código de negocio reutilizable. No existe necesidad de duplicación.
	\end{itemize}
	
Spring MVC es, como otros frameworks MVC, basado en solicitud (request-driven). Están diseñados en torno a un Servlet central que sirve las solicitudes a los controladores y ofrece unas funcionalidades que facilitan el desarrollo de las aplicaciones web. Sin embargo, el DispatcherServlet de Spring es más que eso, está completamente integrado con el contenedor Spring IoC y permite hacer uso de las características y funcionalidades de Spring.

\subsubsection*{Spring Security}
Spring Security ofrece exhaustivos servicios de seguridad para las aplicaciones empresariales basadas en Java EE. 
Las dos principales áreas en las que se enfoca Spring Security son la Autenticación y la Autorización, probablemente, los dos temas más relevantes en la seguridad de las aplicaciones.
	\begin{itemize}
		\item Autenticación es el proceso por el que se determina que uno es el que dice ser.
		\item Autorización hace referencia al proceso de determinar qué acción o acciones puede realizar en la aplicación.
	\end{itemize}
	
A nivel de autenticación, Spring Security soporta un amplio rango de modelos de autenticación. La mayoría de estos modelos de autenticación son proporcionados por terceros, o desarrollados por los organismos estándar pertinentes, como Internet Engineering Task Force. A mayores, Spring Security provee su propio conjunto de mecanismos de autenticación y soporta integración de autenticación con diferentes tecnologías


\subsection{Java Persistence API}
Java Persistence API, comúnmente conocida por sus siglas JPA, es la API que describe la gestión de datos relacionales en aplicaciones que utilicen Java. La primera especificación fue lanzada en mayo de 2006 como parte del trabajo del JSR 220.

JPA en sí mismo es solo una especificación, no un producto. Son un conjunto de interfaces que requieren una implementación. Existen implementaciones de JPA de código abierto y comerciales, y cualquier servidor de aplicaciones JAVA EE 5 debe proporcionar soporte para su uso.

El objetivo de esta API es no perder las ventajas de la orientación a objetos al interactuar con una base de datos y permitir usar objetos regulares, comúnmente conocidos como POJOs. 

\subsection{JAX-RS}
JAX-RS es la API de Java para la elaboración de servicios web RESTful que brinda soporte en la creación de servicios web de acuerdo con el patrón arquitectónico REST. Desde la versión 1.1, JAX-RS es una parte oficial de Java EE 6. Una característica notable de ser parte oficial de Java EE es que no es necesaria ninguna configuración para comenzar a utilizar JAX-RS.

Esta API utiliza anotaciones, introducidas en Java SE 5, para simplificar el desarrollo y la implementación de clientes y recursos web. 

De la misma forma que sucedía con JPA, JAX-RS no es más que una especificación, necesita un producto que la implemente. Jersey y RESTEasy son implementaciones de JAX-RS,

\subsection{Thymeleaf}
Thymeleaf es una librería Java que implementa un motor de plantillas válidas para entornos web como independientes. Es un software de código abierto creado originalmente por un ingeniero de software español llamado Daniel Fernández. No está hecho ni respaldado por ningún software de ninguna compañía y se ofrece al público de manera totalmente gratuita, tanto en formato binario como en código fuente, bajo licencia Apache.

Su objetivo principal es la creación de plantillas de una manera elegante y con un código bien formateado.

Thymeleaf ofrece una buena integración con Spring MVC a través de su dialecto SpringStandard, pero esta integración con Spring es completamente opcional y el dialecto estándar está destinado a usarse sin Spring.

\subsection{jQuery}
jQuery es una librería multiplataforma de JavaScript, creada incialmente por John Resig. Es un software libre y de código abierto, y posee un doble licenciamiento bajo la licencia MIT y la licencia Pública General de GNU, permitiendo su uso en proyectos libres y privados.

Su objetivo es la realización de funcionalidades basadas en JavaScript de forma rápida y sencilla. Permite realizar recorridos y manipulaciones de documentos HTML, manejar eventos, animaciones y usar AJAX  mucho más simple con una API fácil de usar que funciona en multitud de navegadores.

\subsection{Bootstrap}
Bootstrap es un kit de herramientas de código abierto para desarrollo web junto a HTML, CSS y JS.  

Originalmente creado por un diseñador y desarrollador de Twitter, Bootstrap es uno de los frameworks front-end y proyectos de código abierto más populares en el mundo. Antes de ser un framework de código abierto, Bootstrap era conocido como Twitter Blueprint.

Bootstrap incluye plantillas de diseño basadas en HTML y CSS para tipografías, formularios, botones, tablas,... así como complementos de JavaScript opcionales. También brinda la capacidad de crear fácilmente diseños receptivos.

\subsection{AJAX}
AJAX es una técnica de desarrollo web para crear aplicaciones interactivas. Es una tecnología asíncrona, en el sentido que los datos adicionales solicitados al servidor, se cargan en segundo plano sin inferir con la visualización ni el comportamiento de la página.

La funciones de llamada AJAX se efectúan, normalmente, bajo el lenguaje de programación JavaScript mientras que el acceso a los datos se realiza mediante XMLHttpRequest.

AJAX no constituye una tecnología en sí misma, sino que es un término que engloba a un grupo de éstas que trabajan conjuntamente.

	\begin{itemize}
		\item XHTML (o HTML) y CSS para el diseño que acompaña a la información.
		\item Document Object Model (DOM) accedido con un lenguaje de scripting por parte del usuario, generalmente, JavaScript.
		\item El objeto XMLHttpRequest para intercambiar datos de forma asíncrona con el servidor web.
		\item XML es el formato usado generalmente para la transferencia de datos solicitados al servidor.
	\end{itemize}
	

\section{Herramientas de desarrollo}
\subsection{Eclipse}
Eclipse es una plataforma software compuesto por un conjunto de herramientas de programación de código abierto multiplataforma.

Fue desarrollado originalmente por IBM como el sucesor de su familiar de herramientas VisualAge. Ahora, está siendo desarrollado por la Fundación Eclipse, una organización independiente, sin ánimo de lucro, que fomenta una comunidad de código abierto y un conjunto de productos complementarios.

\subsection{Maven}
Maven es una herramienta de software para la gestión y construcción de proyectos Java creada por Jason van Zyl en 2002 que tiene un modelo de configuración de construcción basado en formato XML.

Maven utiliza un Project Object Model, conocido como POM, para describir el proyecto software a construir, sus dependencias de otros módulos y componentes externos, y el orden de construcción de los elementos.

Una de sus características clave son su disponibilidad para usarse en la red puesto que el motor incluido en su núcleo puede dinámicamente descargar plugins de un repositorio.


\subsection{Git}
Git es un sistema de control de versiones distribuido de código abierto y gratuito diseñado para manejar todo, desde proyectos pequeños a muy grandes, con velocidad y eficiencia.

Originalmente fue diseñado como un motor de sistema de control de versiones de bajo nivel sobre el cual otros podían codificar interfaces frontales. Desde entonces hasta ahora, el núcleo del proyecto Git se ha vuelto un sistema de control de versiones completo, utilizable en forma directa.

Git es fácil de aprender y ofrece un rendimiento increíblemente rápido. Su principal objetivo es llevar el registro de cambios en archivos y coordinar el trabajo que varias personas realizan sobre archivos compartidos.


\subsection{Oracle SQL Developer}
Oracle SQL Developer es un entorno de desarrollo integrado y gratuito que simplifica el desarrollo y la administración de las bases de datos de Oracle, tanto de implementaciones tradicionales como en la nube.

Este software admite productos de Oracle y una grana variedad de complementos de terceros que los usuarios pueden implementar para conectarse a bases de datos que no sean de Oracle.

SQL Developer ofrece un desarrollo completo de extremo a extremo de sus aplicaciones PL / SQL, una hoja de trabajo para ejecutar consultas y scripts, una consola DBA para administración, una interfaz de informes y una solución completa de modelado de datos.



\section{Sistema de gestión de bases de datos}
\subsection{Oracle Database}
Oracle Database es un sistema de gestión de base de datos de tipo objeto-relacional, desarrollado por Oracle Corporation. Es la base de datos más popular para el procesamiento de transacciones el línea(OLTP) y almacenes de datos (Data warehousing).

La tecnología Oracle se encuentra prácticamente en todas las industrias alrededor y es el proveedor mundial líder de software para administración de información.

El producto Oracle para el sistema de base de datos cuenta con 7 ediciones diferentes de las cuales, una es completamente gratuita.

\section{Servidor de aplicaciones}
\subsection{Apache Tomcat}
Apache Tomcat funciona como un contenedor de servlets desarrollado por Apache Software Foundation. Es un una implementación de código abierto de las tecnologías Java Servlet, JavaServer Pages, Java Expression Language y Java WebSocket. Se publica bajo la versión 2 de Apache License.

Tomcat no es un servidor de aplicaciones, pero puede funcionar como servidor web por sí mismo. Actualmente es usado como servidor web autónomo en entornos con alto nivel de tráfico y alta disponibilidad.

Está escrito en Java, de manera que puede funcionar en cualquier sistema operativo que disponga de la máquina virtual Java.



\section{Sin especificar}
\subsection{Ionic}
Ionic es un completo SDK de código abierto diseñado para el desarrollo de aplicaciones móviles híbridas. La versión original fue lanzada en 2013 y construida sobre AngularJS y Apache Cordova. Las versiones más recientes, conocidas como Ionic 3, están construidas sobre Angular.

Proviene de herramientas y servicios para desarrollar aplicaciones móviles híbridas usando tecnologías web (CSS, HTML,...).

Ionic proviene de toda la funcionalidad que se puede encontrar en los SDK de desarrollo móvil. Las aplicaciones construidas se pueden personalizar en función del uso final, Android o iOS, por ejemplo.

Además del SDK, Ionic también brinda servicios que los desarrolladores pueden usar para habilitar funciones, como la tecnología push, test A/B, construcciones automáticas, etc...

Ionic también proporciona una poderosa interfaz de línea de comandos, lo que permite poder comenzar a desarrollar un proyecto con un simple comando. También permite a los desarrolladores agregar complementos de Cordova.




\chapter[Metodología]{
  \label{chp:metodologia}
  METODOLOGÍA
}
\thispagestyle{numberingStyle}
\pagestyle{numberingStyle}

El Proceso Unificado es un marco de desarrollo software caracterizado por estar dirigido por casos de uso, centrado en la arquitectura y por ser iterativo e incremental.


\section{Características}

Para el desarrollo de este proyecto se ha escogido una metodología basada en el Proceso Unificado. Esta metodología nos permite obtener un producto de alta calidad gracias a su carácter iterativo e incremental. De manera que, en cada ciclo o iteración, el producto es revisado y mejorado.

\subsection{Dirigido por casos de uso}
Un caso de uso es un fragmento de funcionalidad del sistema que proporciona un resultado de valor a un usuario. Los casos de uso modelan los requerimientos funcionales del sistema y todos juntos constituyen el  \textit{modelo de casos de uso}.

Los casos de uso también guían el proceso de desarrollo (diseño, implementación y prueba). Basándose en los casos de uso, los desarrolladores crean una serie de modelos de diseño e implementación que llevan a cabo. De este modo, los casos de uso no son solo una herramienta para especificar los requerimientos e iniciar el proceso de desarrollo, sino que también dirigen su diseño, implementación, pruebas, etc...

\subsection{Centrado en la arquitectura}
La arquitectura de un sistema software se describe mediante diferentes vistas del sistema en construcción. Se define arquitectura como el conjunto de decisiones significativas acerca de la organización de un sistema software, la selección de los elementos estructurales a partir de los cuales se compone el sistema y su composición.

Los casos de uso y la arquitectura están profundamente relacionados. Los casos de uso deben encajar en la arquitectura, y esta a su vez, debe permitir el desarrollo de todos los casos de uso requeridos, actualmente y a futuro.

De tal forma que, el arquitecto software desarrolla la forma o arquitectura a partir de la comprensión de un conjunto reducido de casos de uso fundamentales o críticos.

\subsection{Iterativo e incremental}
Se divide el desarrollo de un proyecto software en partes más pequeñas o mini proyectos. Cada mini proyecto es una iteración que supono un incremento. Las iteraciones hacen referencia a pasos en el flujo de trabajo, y los incrementos, a crecimientos en el producto.

Las iteraciones deben estar controladas de manera que se deben seleccionar y ejecutar de forma planificada. 

En cada iteración los desarrolladores identifican y especifican los casos de uso relevantes y crean un diseño utilizando la arquitectura seleccionada para implementar dichos casos de uso. Si la iteración cumple sus objetivos, se continúa con la próxima, si no, deben revisarse las decisiones previas y probar un nuevo enfoque.

\section{Ciclo de vida}
El Proceso Unificado se repite a lo largo de una serie de ciclos que constituyen la vida de un sistema.


\begin{figure}[!h]

\includegraphics[
   keepaspectratio=true
]{./03_Met_Plan/01_Metodologia/diagramaProcesoUnificado.JPG}
\caption{Ciclo de vida - Proceso Unificado de Desarrollo}
\end{figure}


Cada ciclo consta de cuatro fases: \textbf{Inicio}, \textbf{Elaboración}, \textbf{Construcción} y \textbf{Transición}. Cada fase se divide en iteraciones, en las cuales se desarrolla, secuencialemte, una serie de disciplinas o flujos de trabajo como por ejemplo: \textit{Análisis, Diseño, Implementación y Pruebas.}

En la figura podemos observar que el ciclo de vida del Proceso Unificado presenta dos dimensiones:
\begin{itemize}
	\item Un eje horizontal que representa el aspecto dinámico del proceso conforme se va desarrollando. Expresado en términos de \textit{Fases, Iteraciones e Hitos}.
	\item Un eje vertical que representa el aspecto estático del proceso mediante las disciplinas o flujos de trabajo.
\end{itemize}


\subsection{Fases}

\subsection{Fase de inicio}
Durante la fase de inicio se desarrolla una descripción del producto final y se presenta el análisis de negocio.

El objetivo de esta fase es ayudar al equipo de proyecto a decidir cuales son los verdaderos objetivos del proyecto.

\subsection{Fase de elaboración}
En esta fase se especifican la mayoría de los casos de uso del producto y se diseña la arquitectura del mismo. Se establece una firme comprensión del problema a solucionar.


\subsection{Fase de construcción}
Durante la fase de construcción se elabora el producto, de tal manera que la línea base de la arquitectura avanza hasta convertirse en el sistema completo. Al final de esta fase se obtienen todos los casos de uso implementados aunque pueden incluir defectos.


\subsection{Fase de transición}
La fase de transición hace referencia al período donde el software ya debe estar listo para ser instalado, probado y utilizado. Las iteraciones en esta fase continúan agregando características al software.


\section{Iteraciones}
\subsection{Iteración 1: Análisis de requisitos y diseño}

En esta primera iteración se determinará la planificación del proyecto, se realizará la recogida de requesitos y el modelado de casos de uso. También se especificará el diseño del sistema y se elaborará el diagrama de clases de la aplicación.


\subsection{Iteración 2: Base inicial del proyecto}
La segunda iteración implicará el comienzo del desarrollo del proyecto. Se preparán los entornos necesarios, se realizará una pequeña formación en las tecnologías que se emplearán y se preparará y configurará la base datos. Al hacer uso de APIs externas, en esta iteración se configurarán para su posterior uso.


\subsection{Iteración 3: Usuarios}
La iteración 3, tiene como objetivo la implementación de las funcionalidades relacionadas con los usuarios de la aplicación. Se realizará el análisis y diseño, así como la implementación de la persistencia y de los servicios para la gestión de los usuarios. Finalmente, se realizarán las pruebas necesarias.


\subsection{Iteración 4: Rutas}
En esta iteración se realizará el análisis y diseño, la implementación de la persistencia y los servicios, y las pruebas necesarias para la gestión de las rutas en la aplicación.


\subsection{Iteración 5: Eventos}
Análogamente, en la iteración número 5, se realizará el mismo ciclo de acciones. Análisis y diseño, implementación de la persistencia y serivicios, y pruebas, esta vez, para la gestión de eventos.



\subsection{Iteración 6: Servicios externos}
En esta iteración se llevará a cabo la gestión de las funcionalidades y datos obtenidos de las fuentes externas. Se realizará el análisis y diseño para la implementación de los servicios externos necesarios, se realizarán las modificaciones necesarias para el correcto funcionamiento de la librería Java de \textit{Foursquare}, así como la implementación y pruebas necesarias para dichos servicios.


\subsection{Iteración 7: Lugares y categorías}
Se realizará análisis, diseño, implementación y pruebas para la gestión de lugares y categorías, cuyos datos serán obtenidos por los servicios definidos e implementados en la iteración anterior.


\subsection{Iteración 8: Autenticación y Autorización}
En este iteración se llevará a cabo el proceso de autenticación y autorización de los usuarios. Se realizará el análisis y diseño necesarios, así como una formación en las tecnologías a utilizar: Json Web Tokens para autenticación y Spring Security en la autorización. Ambos serán implementados y probados.


\subsection{Iteración 9: Cliente móvil}
De igual forma, en esta iteración, se realizará anáisis, diseño, implementación y pruebas para la elaboración del cliente móvil. Se realizará también una formación en las tecnologías utilizadas en su creación, la implementación de las pantallas y controladores, y de los módulos de acceso de a los servicios previamente implementados.


\subsection{Iteración 10: Cliente web}
Iteración cuyo objetivo es el análisis y desarrollo del cliente web. De igual forma que para en cliente móvil, se realizará el análisis y diseño necesario, la implementación del módulo de acceso a los servicios y las pantallas y controladores. Posteriormente, se realizarán las pruebas necesarias.


\subsection{Iteración 11: Panel de administración}
En esta iteración se realizará una ampliación del cliente web con la creación de un panel de administración de la aplicación. Se realizará análisis y diseño, junto a la implementación y pruebas.


\subsection{Iteración 12: Cierre}
En la última iteración se realizará la evaluación final sistema y su revisión.







\chapter[Planificación y costes]{
  \label{chp:planificacion}
  PLANIFICACIÓN Y COSTES
}
\thispagestyle{numberingStyle}
\pagestyle{numberingStyle}



\section{Planificación}

\subsection{Planificación de las tareas}
Para las iteraciones definidas anteriormente se establecerá una división en tareas, sobre las que se realizará la planificación necesaria para obtener la estimación del esfuerzo requerido en cada iteración.

A continuación, se mostrará las diferentes tareas que componen cada una de las iteraciones junto con los valores de esfuerzo necesario para su realización (en horas).

\subsubsection*{Iteración 1: Análisis de requisitos y diseño}

\begin{figure}[H]
\vspace{-0.5cm}
\includegraphics[
   keepaspectratio=true
]{./03_Met_Plan/02_Planificacion/img/PlanIter1.png}
\caption{Diagrama tareas iteración 1}
\end{figure}


\subsubsection*{Iteración 2: Base inicial proyecto}

\begin{figure}[H]
\vspace{-1cm}
\includegraphics[
   keepaspectratio=true
]{./03_Met_Plan/02_Planificacion/img/PlanIter2.png}
\caption{Diagrama tareas iteración 2}
\end{figure}


\subsubsection*{Iteración 3: Usuarios}

\begin{figure}[H]
\vspace{-1cm}
\includegraphics[
   keepaspectratio=true
]{./03_Met_Plan/02_Planificacion/img/PlanIter3.png}
\caption{Diagrama tareas iteración 3}
\end{figure}


\subsubsection*{Iteración 4: Rutas}

\begin{figure}[H]
\vspace{-1cm}
\includegraphics[
   keepaspectratio=true
]{./03_Met_Plan/02_Planificacion/img/PlanIter4.png}
\caption{Diagrama tareas iteración 4}
\end{figure}



\subsubsection*{Iteración 5: Eventos}

\begin{figure}[H]
\vspace{-1cm}
\includegraphics[
   keepaspectratio=true
]{./03_Met_Plan/02_Planificacion/img/PlanIter5.png}
\caption{Diagrama tareas iteración 5}
\end{figure}


\subsubsection*{Iteración 6: Servicios externos}

\begin{figure}[H]
\vspace{-1cm}
\includegraphics[
   keepaspectratio=true
]{./03_Met_Plan/02_Planificacion/img/PlanIter6.png}
\caption{Diagrama tareas iteración 6}
\end{figure}


\subsubsection*{Iteración 7: Lugares y categorías}

\begin{figure}[H]
\vspace{-1cm}
\includegraphics[
   keepaspectratio=true
]{./03_Met_Plan/02_Planificacion/img/PlanIter7.png}
\caption{Diagrama tareas iteración 7}
\end{figure}


\subsubsection*{Iteración 8: Autenticación y autorización}

\begin{figure}[H]
\vspace{-1cm}
\includegraphics[
   keepaspectratio=true
]{./03_Met_Plan/02_Planificacion/img/PlanIter8.png}
\caption{Diagrama tareas iteración 8}
\end{figure}


\subsubsection*{Iteración 9: Cliente móvil}

\begin{figure}[H]
\vspace{-1cm}
\includegraphics[
   keepaspectratio=true
]{./03_Met_Plan/02_Planificacion/img/PlanIter9.png}
\caption{Diagrama tareas iteración 9}
\end{figure}


\subsubsection*{Iteración 10: Cliente web}

\begin{figure}[H]
\vspace{-1cm}
\includegraphics[
   keepaspectratio=true
]{./03_Met_Plan/02_Planificacion/img/PlanIter10.png}
\caption{Diagrama tareas iteración 10}
\end{figure}


\subsubsection*{Iteración 11: Panel de administración}

\begin{figure}[H]
\vspace{-1cm}
\includegraphics[
   keepaspectratio=true
]{./03_Met_Plan/02_Planificacion/img/PlanIter11.png}
\caption{Diagrama tareas iteración 11}
\end{figure}


\subsubsection*{Iteración 12: Cierre}

\begin{figure}[H]
\vspace{-1cm}
\includegraphics[
   keepaspectratio=true
]{./03_Met_Plan/02_Planificacion/img/PlanIter12.png}
\caption{Diagrama tareas iteración 12}
\end{figure}


\subsubsection*{Planificación global}

\begin{figure}[H]
\vspace{-0.5cm}
\includegraphics[
   keepaspectratio=true
]{./03_Met_Plan/02_Planificacion/img/PlanIterGlobal.png}
\caption{Diagrama tareas global}
\end{figure}


\section{Evaluación de costes}



\chapter[Análisis]{
  \label{chp:analisis}
  ANÁLISIS
}
\thispagestyle{numberingStyle}
\pagestyle{numberingStyle}

En este apartado se explicará, detalladamente, el análisis realizado.

\section{Análisis de requerimientos}

\subsection{Requerimientos funcionales}
A continuación se muestran los requerimientos funcionales del sistema, clasificados en distintas áreas.

\subsubsection*{Acceso a la aplicación}
\begin{itemize}
\setlength\itemsep{1pt}
\item El sistema ofrecerá la posibilidad de que un usuario se registre en la aplicación.
\item El sistema ofrecerá la posibilidad de que el usuario se identifique en el sistema. Los usuarios deben ingresar al sistema con  nombre de usuario y contraseña.
\end{itemize}

\subsubsection*{Cliente Móvil}
\begin{itemize}
\setlength\itemsep{1pt}
\item El sistema ofrecerá la posibilidad de crear nuevas rutas.
\item El usuario podrá consultar las rutas, propias y de otros usuarios, según ciertos criterios de búsqueda.
\item El sistema permitirá a los usuarios autorizados eliminar las rutas propias que deseen.
\item El sistema permitirá a los usuarios autorizados a consultar los detalles de las rutas.
\item Para las rutas, el sistema permitirá:
	\begin{itemize}
	\item Establecer las fechas de inicio y fin.
	\item Consultar, asignar y desasignar a la ruta, los eventos disponibles en esas fechas.
	\item Consultar el itinerario por días.
	\item Modificar la hora de comienzo establecida para cada día.
	\item Consultar, añadir y eliminar lugares de interés a cada día de la ruta.
	\item Editar el modo de viaje a realizar entre diferentes lugares.
	\item Mostrar el itinerario, por días y en total, en el mapa.
	\item Habilitar y deshabilitar el sistema de geolocalización para conocer la ruta hecha en tiempo real.
	\item Consultar y comparar, el itinerario definido con el obtenido a tiempo real.
	\item Editar los permisos de la ruta.
	\end{itemize}
\end{itemize}

\subsubsection*{Cliente Web}
\begin{itemize}
\setlength\itemsep{1pt}
\item El usuario podrá consultar las rutas existentes, propias y de otros usuarios, según ciertos criterios de búsqueda.
\item El sistema permitirá a los usuarios autorizados a consultar los detalles de las rutas.
\item El sistema permitirá a los usuarios autorizados marcar las rutas propias como privadas, con el fin de no compartirlas con los demás usuarios.
\item El sistema solo ofrecerá la posibilidad de consulta sobre los detalles de una ruta, permitiendo ver el itinerario, si tiene datos en tiempo real guardados, etc...
\item El sistema permitirá a los usuarios autorizados eliminar las rutas propias que deseen.
\end{itemize}

\subsubsection*{Cliente Administración Web}
\begin{itemize}
\setlength\itemsep{1pt}
\item El sistema solo permitirá acceso a usuarios con permisos de administración.
\item El sistema permitirá a los usuarios con dichos permisos, dar de alto nuevos usuarios.
\item El sistema permitirá las altas, bajas, modificaciones y consultas de las entidades del sistema.
	\begin{itemize}
	\item El sistema ofrecerá la posibilidad de crear, eliminar, modificar y consultar datos de usuarios.
	\item El sistema ofrecerá la posibilidad de crear, eliminar, modificar y consultar datos de rutas.
	\item El sistema ofrecerá la posibilidad de crear, eliminar, modificar y consultar datos de lugares.
	\item El sistema ofrecerá la posibilidad de crear, eliminar, modificar y consultar datos de categorías.
	\item El sistema ofrecerá la posibilidad de crear, eliminar, modificar y consultar datos de eventos.
	\end{itemize}
\item El sistema permitirá la existencia de usuarios con capacidades para la administración y gestión, exclusivamente, de los eventos. Permitiendo así, sus altas,  bajas, modificaciones y consultas de los mismos en el sistema.
\end{itemize}

\subsubsection*{Seguridad}
\begin{itemize}
\setlength\itemsep{1pt}
\item El sistema ofrecerá la posibilidad de que el usuario modifique sus datos de acceso al sistema.
\item El sistema solo permitirá acciones correctamente autenticadas, exceptuando las de acceso a la aplicación.
\item Los usuarios de la aplicación solo podrán modificar los datos a los que el usuario esté autorizado. Un usuario no podrá modificar la información de los recursos de los que no es propietario.
\item Los intercambios de datos que realice el sistema a través de internet, serán mediante el uso del protocolo encriptado https.
\end{itemize}


\subsection{Requerimientos no funcionales}





\section{Modelo de casos de uso}

\subsection{Actores del sistema}
Analizando los requerimientos funcionales funcionales del sistema, se detectan tres tipos de actores, que demandan una determinada funcionalidad en el sistema. Estos tres actores son, el cliente, el administrador y el gestor de eventos.


\begin{figure}[htbp]
\centering
\vspace*{0.2in}
\includesvg{actores.svg}
\caption{Diagrama casos de uso aplicación móvil.}
\end{figure}


\subsection{Diagrama de casos de uso}
Tras conocer los requerimientos funcionales del sistema y reconocer las necesidades del sistema, se ha optado por diseñar un sistema general compuesto por los diferentes subsistemas del mismo.


\FloatBarrier
\subsubsection*{Diagrama sistema general}
Dicho sistema, estará compuesto por tres grupos de subsistemas: subsistema de acceso, subsistema de administración y subsistema de aplicación.

\begin{figure}[!htbp]
\centering

\includegraphics[
   keepaspectratio=true
]{./../Diagrams/uc__Sistema-General.png}
\caption{Diagrama casos de uso - Sistema general.}
\end{figure}


\FloatBarrier
\subsubsection*{Diagrama sistema de acceso}
\begin{figure}[!htbp]
\centering

\includegraphics[
   keepaspectratio=true
]{./../Diagrams/uc__Sistema-Acceso.png}
\caption{Diagrama casos de uso - Sistema general.}
\end{figure}


\FloatBarrier
\subsubsection*{Diagrama sistema aplicación móvil}
\begin{figure}[!htbp]
\centering

\includegraphics[
   keepaspectratio=true
]{./../Diagrams/Sistema-Aplicacion-Movil.png}
\caption{Diagrama casos de uso - Sistema general.}
\end{figure}


\FloatBarrier
\subsubsection*{Diagrama sistema aplicación web}
\begin{figure}[!htbp]
\centering

\includegraphics[
   keepaspectratio=true
]{./../Diagrams/Sistema-Aplicacion-Web.png}
\caption{Diagrama casos de uso - Sistema general.}
\end{figure}


\FloatBarrier
\subsubsection*{Diagrama sistema administración}
\begin{figure}[!htbp]
\centering

\includegraphics[
   keepaspectratio=true
]{./../Diagrams/Sistema-Administracion.png}
\caption{Diagrama casos de uso - Sistema general.}
\end{figure}


\FloatBarrier
\subsubsection*{Diagrama sistema gestión usuarios}
\begin{figure}[!htbp]
\centering

\includegraphics[
   keepaspectratio=true
]{./../Diagrams/Sistema-Gestion-Usuarios.png}
\caption{Diagrama casos de uso - Sistema general.}
\end{figure}


\FloatBarrier
\subsubsection*{Diagrama sistema gestión rutas}
\begin{figure}[!htbp]
\centering

\includegraphics[
   keepaspectratio=true
]{./../Diagrams/Sistema-Gestion-Rutas.png}
\caption{Diagrama casos de uso - Sistema general.}
\end{figure}


\FloatBarrier
\subsubsection*{Diagrama sistema gestión lugares}
\begin{figure}[!htbp]
\centering

\includegraphics[
   keepaspectratio=true
]{./../Diagrams/Sistema-Gestion-Lugares.png}
\caption{Diagrama casos de uso - Sistema general.}
\end{figure}


\FloatBarrier
\subsubsection*{Diagrama sistema gestión eventos}
\begin{figure}[!htbp]
\centering

\includegraphics[
   keepaspectratio=true
]{./../Diagrams/Sistema-Gestion-Eventos.png}
\caption{Diagrama casos de uso - Sistema general.}
\end{figure}


\FloatBarrier
\subsubsection*{Diagrama sistema gestión categorías}
\begin{figure}[!htbp]
\centering

\includegraphics[
   keepaspectratio=true
]{./../Diagrams/Sistema-Gestion-Categorias.png}
\caption{Diagrama casos de uso - Sistema general.}
\end{figure}



\newpage
\subsection{Especificación casos de uso}


\newpage
\subsubsection*{Caso de uso: Autenticarse}
\begin{longtable}{| p{4cm} | p{10cm} |}
\endfirsthead
\multicolumn{2}{c}{\textit{Continúa de la página anterior}}\\[12pt]
\hline
\endhead
\hline
\multicolumn{2}{c}{\textit{Continúa en la siguiente página}} \\
\endfoot
\hline
\caption{Caso de Uso: Autenticarse}\label{fig:1}\\
\endlastfoot


\hline
\multicolumn{2}{|c|}{\textbf{CU$<$01$>$ - Autenticarse}} \\

\hline
\textbf{Descripción} &
El usuario se identifica introduciendo las credenciales de acceso en el sistema \\

\hline
\textbf{Actores} &
Cliente Móvil\newline
Cliente Web\newline
Administrador\newline
Moderador\\

\hline
\textbf{Precondiciones} &
\\

\hline
\textbf{Secuencia Normal} &\mbox{}\par\vspace{-\baselineskip}
\begin{enumerate}[leftmargin=0.7cm, topsep=0.1cm]
\item El usuario introduce sus credenciales en la ventana de login. 
\item El usuario pulsa el botón de \textit{Acceder}.
\item El sistema valida las credenciales.
\item El usuario accede a la aplicación.
\end{enumerate}\\

\hline
\textbf{Excepciones} &\mbox{}\par\vspace{-\baselineskip}
\begin{enumerate}[leftmargin=0.9cm, topsep=0.1cm]
\item[3.] Los datos introducidos no son correctos.
	\begin{itemize}
	\item[1.] El sistema muestra un mensaje de error y regresa al \textit{Paso 1}.
	\end{itemize}

\end{enumerate}\\

\hline
\textbf{Postcondiciones} & 
El usuario queda autenticado en el sistema\\
\hline
\end{longtable}




\newpage
\subsubsection*{Caso de uso: Registrarse}
\begin{longtable}{| p{4cm} | p{10cm} |}
\endfirsthead
\multicolumn{2}{c}{\textit{Continúa de la página anterior}}\\[12pt]
\hline
\endhead
\hline
\multicolumn{2}{c}{\textit{Continúa en la siguiente página}} \\
\endfoot
\hline
\caption{Caso de Uso: Registrarse}\label{fig:1}\\
\endlastfoot


\hline
\multicolumn{2}{|c|}{\textbf{CU$<$02$>$ - Registrarse}} \\

\hline
\textbf{Descripción} &
El usuario introduce los datos para darse de alta en la aplicación. \\

\hline
\textbf{Actores} &
Cliente Móvil\\


\hline
\textbf{Precondiciones} &
N/A\\

\hline
\textbf{Secuencia Normal} &\mbox{}\par\vspace{-\baselineskip}
\begin{enumerate}[leftmargin=0.7cm, topsep=0.1cm]
\item El usuario selecciona la opción de registro.
\item El sistema muestra un formulario indicando los campos necesarios para realizar el registro.
\item El usuario rellena los campos y pulsa el botón de \textit{Registrarse}.
\item El sistema valida los datos introducidos por el usuario.
\item El usuario accede a la aplicación.
\end{enumerate}\\

\hline
\textbf{Excepciones} &\mbox{}\par\vspace{-\baselineskip}
\begin{enumerate}[leftmargin=0.9cm, topsep=0.1cm]
\item[3.] El usuario pulsa el botón de \textit{Cancelar}.
	\begin{itemize}
	\item[1.] El sistema cancela el registro y redirige al usuario a la pantalla de login.
	\end{itemize}
\item[4.] Los datos introducidos por el usuario no son válidos.
	\begin{itemize}
	\item[1.] El sistema muestra un mensaje de error y regresa al \textit{Paso 2}.
	\end{itemize}
\end{enumerate}\\

\hline
\textbf{Postcondiciones} & 
El usuario queda registrado y autenticado en el sistema\\
\hline
\end{longtable}




\newpage
\subsubsection*{Caso de uso: Crear ruta}
\begin{longtable}{| p{4cm} | p{10cm} |}
\endfirsthead
\multicolumn{2}{c}{\textit{Continúa de la página anterior}}\\[12pt]
\hline
\endhead
\hline
\multicolumn{2}{c}{\textit{Continúa en la siguiente página}} \\
\endfoot
\hline
\caption{Caso de Uso: Crear ruta}\label{fig:1}\\
\endlastfoot


\hline
\multicolumn{2}{|c|}{\textbf{CU$<$03$>$ - Crear Ruta}} \\

\hline
\textbf{Descripción} &
El usuario crea una ruta para una ciudad o lugar especificado. \\

\hline
\textbf{Actores} &
Cliente Móvil\\

\hline
\textbf{Precondiciones} &
El usuario está autenticado en la aplicación\\

\hline
\textbf{Secuencia Normal} &\mbox{}\par\vspace{-\baselineskip}
\begin{enumerate}[leftmargin=0.7cm, topsep=0.1cm]
\item El usuario selecciona la opción de crear una nueva ruta.
\item El sistema muestra buscador para que el usuario indique en qué ciudad o lugar desea crear dicha ruta.
\item El usuario rellena el buscador.
\item El sistema ayuda al usuario autocompletando con los datos de diferentes ciudades y lugares.
\item El usuario selecciona el lugar en la lista de autocompletado ofrecida por el sistema.
\item El sistema muestra un mapa indicando la ubicación del lugar seleccionado y permite al usuario completar el proceso de creación.
\item El usuario pulsa el botón para crear la ruta.
\end{enumerate}\\

\hline
\textbf{Excepciones} &\mbox{}\par\vspace{-\baselineskip}
\\

\hline
\textbf{Postcondiciones} & 
La ruta queda registrada en el sistema\\
\hline
\end{longtable}




\newpage
\subsubsection*{Caso de uso: Explorar rutas}
\begin{longtable}{| p{4cm} | p{10cm} |}
\endfirsthead
\multicolumn{2}{c}{\textit{Continúa de la página anterior}}\\[12pt]
\hline
\endhead
\hline
\multicolumn{2}{c}{\textit{Continúa en la siguiente página}} \\
\endfoot
\hline
\caption{Caso de Uso: Explorar rutas}\label{fig:1}\\
\endlastfoot


\hline
\multicolumn{2}{|c|}{\textbf{CU$<$04$>$ - Explorar Rutas}} \\

\hline
\textbf{Descripción} &
El usuario explora las diferentes rutas creadas por los demás usuarios.\\

\hline
\textbf{Actores} &
Cliente Móvil\newline
Cliente Web\\

\hline
\textbf{Precondiciones} &
El usuario está autenticado en la aplicación\\

\hline
\textbf{Secuencia Normal} &\mbox{}\par\vspace{-\baselineskip}
\begin{enumerate}[leftmargin=0.7cm, topsep=0.1cm]
\item El usuario selecciona la opción de explorar rutas.
\item El sistema muestra las diferentes rutas que hay en el sistema.
\end{enumerate}\\

\hline
\textbf{Excepciones} &\mbox{}\par\vspace{-\baselineskip}
\\

\hline
\textbf{Postcondiciones} & 
\\
\hline
\end{longtable}



\newpage
\subsubsection*{Caso de uso: Obtener rutas propias}
\begin{longtable}{| p{4cm} | p{10cm} |}
\endfirsthead
\multicolumn{2}{c}{\textit{Continúa de la página anterior}}\\[12pt]
\hline
\endhead
\hline
\multicolumn{2}{c}{\textit{Continúa en la siguiente página}} \\
\endfoot
\hline
\caption{Caso de Uso: Obtener Rutas propias}\label{fig:1}\\
\endlastfoot


\hline
\multicolumn{2}{|c|}{\textbf{CU$<$05$>$ - Obtener Rutas Propias}} \\

\hline
\textbf{Descripción} &
El usuario obtiene las rutas creadas por él.\\

\hline
\textbf{Actores} &
Cliente Móvil\newline
Cliente Web\\

\hline
\textbf{Precondiciones} &
El usuario está autenticado en la aplicación\\

\hline
\textbf{Secuencia Normal} &\mbox{}\par\vspace{-\baselineskip}
\begin{enumerate}[leftmargin=0.7cm, topsep=0.1cm]
\item El usuario selecciona la opción de obtener rutas propias.
\item El sistema muestra las rutas que tiene almacenadas en el sistema y.
\item El sistema clasifica las rutas del usuario en función de su progreso; las que aún no empezaron, las que están en curso y las que ya se realizaron.
\item El usuario selecciona una de las posibilidades ofrecidas por el sistema.
\item El sistema filtra las rutas del usuario según lo solicitado.
\end{enumerate}\\

\hline
\textbf{Excepciones} &\mbox{}\par\vspace{-\baselineskip}
\\

\hline
\textbf{Postcondiciones} & 
\\
\hline
\end{longtable}




\newpage
\subsubsection*{Caso de uso: Consultar ruta}
\begin{longtable}{| p{4cm} | p{10cm} |}
\endfirsthead
\multicolumn{2}{c}{\textit{Continúa de la página anterior}}\\[12pt]
\hline
\endhead
\hline
\multicolumn{2}{c}{\textit{Continúa en la siguiente página}} \\
\endfoot
\hline
\caption{Caso de Uso: Consultar ruta}\label{fig:1}\\
\endlastfoot


\hline
\multicolumn{2}{|c|}{\textbf{CU$<$06$>$ - Consultar Ruta}} \\

\hline
\textbf{Descripción} &
El usuario obtiene la información detallada de una ruta concreta.\\

\hline
\textbf{Actores} &
Cliente Móvil\newline
Cliente Web\\

\hline
\textbf{Precondiciones} &
\\

\hline
\textbf{Secuencia Normal} &\mbox{}\par\vspace{-\baselineskip}
\begin{enumerate}[leftmargin=0.7cm, topsep=0.1cm]
\item El usuario selecciona una ruta concreta.
\item El sistema obtiene los datos de la ruta.
\item El sistema muestra un panel con los datos detallados de la ruta.
\end{enumerate}\\

\hline
\textbf{Excepciones} &\mbox{}\par\vspace{-\baselineskip}
\begin{enumerate}[leftmargin=0.9cm, topsep=0.1cm]
\item[3.] El usuario pulsa el botón de \textit{Atrás}.
	\begin{itemize}
	\item[1.] El sistema devuelve al usuario a la vista anterior.
	\end{itemize}
\end{enumerate}
\\

\hline
\textbf{Postcondiciones} & 
\\
\hline
\end{longtable}



\newpage
\subsubsection*{Caso de uso: Modificar fechas de viaje}
\begin{longtable}{| p{4cm} | p{10cm} |}
\endfirsthead
\multicolumn{2}{c}{\textit{Continúa de la página anterior}}\\[12pt]
\hline
\endhead
\hline
\multicolumn{2}{c}{\textit{Continúa en la siguiente página}} \\
\endfoot
\hline
\caption{Caso de Uso: Modificar fechas de viaje}\label{fig:1}\\
\endlastfoot


\hline
\multicolumn{2}{|c|}{\textbf{CU$<$07$>$ - Modificar Fechas de Viaje}} \\

\hline
\textbf{Descripción} &
El usuario selecciona las fechas en las que se realizará la ruta que está planificando.\\

\hline
\textbf{Actores} &
Cliente Móvil\\

\hline
\textbf{Precondiciones} &
\\

\hline
\textbf{Secuencia Normal} &\mbox{}\par\vspace{-\baselineskip}
\begin{enumerate}[leftmargin=0.7cm, topsep=0.1cm]
\item El usuario selecciona la opción de \textit{Seleccionar Fechas}.
\item El sistema muestra una pantalla con las fechas del calendario.
\item El usuario selecciona la fecha de inicio y fin y pulsa el botón de \textit{Aceptar}.
\item El sistema modifica las fechas de la ruta.
\end{enumerate}\\

\hline
\textbf{Excepciones} &\mbox{}\par\vspace{-\baselineskip}
\begin{enumerate}[leftmargin=0.9cm, topsep=0.1cm]
\item[3.] El usuario pulsa el botón de \textit{Cancelar}.
	\begin{itemize}
	\item[1.] El sistema cancela la modificación de las fechas y cierra la pantalla.
	\end{itemize}
\end{enumerate}
\\

\hline
\textbf{Postcondiciones} & 
Las fechas de la ruta quedan actualizadas en el sistema.\\
\hline
\end{longtable}



\newpage
\subsubsection*{Caso de uso: Modificar privacidad}
\begin{longtable}{| p{4cm} | p{10cm} |}
\endfirsthead
\multicolumn{2}{c}{\textit{Continúa de la página anterior}}\\[12pt]
\hline
\endhead
\hline
\multicolumn{2}{c}{\textit{Continúa en la siguiente página}} \\
\endfoot
\hline
\caption{Caso de Uso: Modificar privacidad}\label{fig:1}\\
\endlastfoot


\hline
\multicolumn{2}{|c|}{\textbf{CU$<$08$>$ - Modificar Privacidad}} \\

\hline
\textbf{Descripción} &
El usuario puede alternar la privacidad de cada una de sus rutas permitiendo que sean visible para todos o solo para él.\\

\hline
\textbf{Actores} &
Cliente Móvil\\

\hline
\textbf{Precondiciones} &
\\

\hline
\textbf{Secuencia Normal} &\mbox{}\par\vspace{-\baselineskip}
\begin{enumerate}[leftmargin=0.7cm, topsep=0.1cm]
\item El usuario selecciona la opción de \textit{Detalles de la Ruta}.
\item El sistema muestra una pantalla con los detalles de la ruta.
\item El sistema incluye un botón que permite alternar entre los estados de privacidad.
\item El usuario pulsa el botón.
\item El sistema cambia el valor del botón y actualiza el nuevo valor de privacidad.
\end{enumerate}\\

\hline
\textbf{Excepciones} &\mbox{}\par\vspace{-\baselineskip}
\begin{enumerate}[leftmargin=0.9cm, topsep=0.1cm]
\item[4.] El usuario pulsa el botón de \textit{Atrás}.
	\begin{itemize}
	\item[1.] El sistema retorna a la pantalla anterior.
	\end{itemize}
\end{enumerate}
\\

\hline
\textbf{Postcondiciones} & 
La privacidad de la ruta queda actualizada.\\
\hline
\end{longtable}



\newpage
\subsubsection*{Caso de uso: Mostrar ruta en mapa}
\begin{longtable}{| p{4cm} | p{10cm} |}
\endfirsthead
\multicolumn{2}{c}{\textit{Continúa de la página anterior}}\\[12pt]
\hline
\endhead
\hline
\multicolumn{2}{c}{\textit{Continúa en la siguiente página}} \\
\endfoot
\hline
\caption{Caso de Uso: Mostrar Ruta en Mapa}\label{fig:1}\\
\endlastfoot


\hline
\multicolumn{2}{|c|}{\textbf{CU$<$09$>$ - Mostrar Ruta en Mapa}} \\

\hline
\textbf{Descripción} &
El usuario puede mostrar la información de la ruta en el mapa.\\

\hline
\textbf{Actores} &
Cliente Móvil\newline
Cliente Web\\

\hline
\textbf{Precondiciones} &
\\

\hline
\textbf{Secuencia Normal} &\mbox{}\par\vspace{-\baselineskip}
\begin{enumerate}[leftmargin=0.7cm, topsep=0.1cm]
\item El usuario selecciona la opción de \textit{Mostrar Ruta en Mapa}.
\item El sistema muestra un mapa con las marcas de los lugares asignados a la ruta en un día concreto de la ruta.
\item Si la ruta ya ha transcurrido en el tiempo y tiene información de su ejecución en tiempo real.
	\begin{itemize}
	\item[1.] El sistema incorpora al mapa la ruta real realizada por el usuario permitiendo compararla con la ruta planificada.
	\end{itemize}
\item El usuario puede cambiar entre los días haciendo click sobre ellos.
\end{enumerate}\\

\hline
\textbf{Excepciones} &\mbox{}\par\vspace{-\baselineskip}
\begin{enumerate}[leftmargin=0.9cm, topsep=0.1cm]
\item[3.] El usuario pulsa el botón de \textit{Atrás}.
	\begin{itemize}
	\item[1.] El sistema retorna a la pantalla anterior.
	\end{itemize}
\end{enumerate}
\\

\hline
\textbf{Postcondiciones} & \\
\hline
\end{longtable}



\newpage
\subsubsection*{Caso de uso: Consultar eventos}
\begin{longtable}{| p{4cm} | p{10cm} |}
\endfirsthead
\multicolumn{2}{c}{\textit{Continúa de la página anterior}}\\[12pt]
\hline
\endhead
\hline
\multicolumn{2}{c}{\textit{Continúa en la siguiente página}} \\
\endfoot
\hline
\caption{Caso de Uso: Cosnultar eventos}\label{fig:1}\\
\endlastfoot


\hline
\multicolumn{2}{|c|}{\textbf{CU$<$10$>$ - Consultar Eventos}} \\

\hline
\textbf{Descripción} &
El usuario obtiene los eventos dados de alta en el sistema\\

\hline
\textbf{Actores} &
Cliente Móvil\\

\hline
\textbf{Precondiciones} &
\\

\hline
\textbf{Secuencia Normal} &\mbox{}\par\vspace{-\baselineskip}
\begin{enumerate}[leftmargin=0.7cm, topsep=0.1cm]
\item El usuario selecciona la opción de \textit{Consultar Eventos}.
\item El sistema muestra una pantalla con las opciones: \textit{En el viaje} y \textit{Próximos}.
\item Si el usuario selecciona \textit{En el viaje}.
	\begin{itemize}
	\item[1.] El sistema muestra los eventos que coinciden durante las fechas de viaje del usuario.
	\end{itemize}
\item Si el usuario selecciona \textit{Próximos}.
	\begin{itemize}
	\item[1.] El sistema muestra los eventos futuros a las fechas de viaje del usuario.
	\end{itemize}
\end{enumerate}\\

\hline
\textbf{Excepciones} &\mbox{}\par\vspace{-\baselineskip}
\begin{enumerate}[leftmargin=0.9cm, topsep=0.1cm]
\item[3-4.] El usuario pulsa el botón de \textit{Atrás}.
	\begin{itemize}
	\item[1.] El sistema retorna a la pantalla anterior.
	\end{itemize}
\end{enumerate}
\\

\hline
\textbf{Postcondiciones} & \\
\hline
\end{longtable}



\newpage
\subsubsection*{Caso de uso: Mostrar evento en mapa }
\begin{longtable}{| p{4cm} | p{10cm} |}
\endfirsthead
\multicolumn{2}{c}{\textit{Continúa de la página anterior}}\\[12pt]
\hline
\endhead
\hline
\multicolumn{2}{c}{\textit{Continúa en la siguiente página}} \\
\endfoot
\hline
\caption{Caso de Uso: Mostrar Evento en Mapa}\label{fig:1}\\
\endlastfoot


\hline
\multicolumn{2}{|c|}{\textbf{CU$<$11$>$ - Mostrar Evento en Mapa}} \\

\hline
\textbf{Descripción} &
Muestra un evento concreto en el mapa.\\

\hline
\textbf{Actores} &
Cliente Móvil\\

\hline
\textbf{Precondiciones} &
\\

\hline
\textbf{Secuencia Normal} &\mbox{}\par\vspace{-\baselineskip}
\begin{enumerate}[leftmargin=0.7cm, topsep=0.1cm]
\item El usuario selecciona la opción de \textit{Mostrar Evento en Mapa}.
\item El sistema muestra un mapa indicando la ubicación del evento.
\item El sistema incluye la ubicación de los lugares asignados al día de la ruta que transcurre el mismo día del evento, permitiendo ubicar el evento en función de la ruta ya creada.
\end{enumerate}
\\
\hline
\textbf{Excepciones} &\mbox{}\par\vspace{-\baselineskip}
\begin{enumerate}[leftmargin=0.9cm, topsep=0.1cm]
\item[2-3.] El usuario pulsa el botón de \textit{Atrás}.
	\begin{itemize}
	\item[1.] El sistema retorna a la pantalla anterior.
	\end{itemize}
\end{enumerate}
\\

\hline
\textbf{Postcondiciones} & \\
\hline
\end{longtable}



\newpage
\subsubsection*{Caso de uso: Asignar eventos a ruta }
\begin{longtable}{| p{4cm} | p{10cm} |}
\endfirsthead
\multicolumn{2}{c}{\textit{Continúa de la página anterior}}\\[12pt]
\hline
\endhead
\hline
\multicolumn{2}{c}{\textit{Continúa en la siguiente página}} \\
\endfoot
\hline
\caption{Caso de Uso: Asignar evento a ruta}\label{fig:1}\\
\endlastfoot


\hline
\multicolumn{2}{|c|}{\textbf{CU$<$12$>$ - Asignar Evento a ruta}} \\

\hline
\textbf{Descripción} &
Permite incorporar o eliminar eventos a la ruta.\\

\hline
\textbf{Actores} &
Cliente Móvil\\

\hline
\textbf{Precondiciones} &
\\

\hline
\textbf{Secuencia Normal} &\mbox{}\par\vspace{-\baselineskip}
\begin{enumerate}[leftmargin=0.7cm, topsep=0.1cm]
\item Si el evento no está asignado a la ruta.
	\begin{itemize}
	\item[1.] El sistema muestra un botón para añadir dicho evento.
	 en la ruta del usuario.
	\end{itemize}
\item Si el evento ya está asignado a la ruta.
	\begin{itemize}
	\item[1.] El sistema muestra un botón para eliminar dicho evento de la ruta.
	\end{itemize}
\item El usuario clicka el botón correspondiente.
\item El sistema muestra un una ventana de confirmación.
\item El usuario pulsa en \textit{Aceptar}
\item El sistema incorpora o elimina el evento al día correspondiente.
\end{enumerate}


\\
\hline
\textbf{Excepciones} &\mbox{}\par\vspace{-\baselineskip}
\begin{enumerate}[leftmargin=0.9cm, topsep=0.1cm]
\item[2.] El usuario pulsa el botón de \textit{Atrás}.
	\begin{itemize}
	\item[1.] El sistema retorna a la pantalla anterior.
	\end{itemize}
\item[5.] El usuario pulso en \textit{Cancelar} en la ventana de confirmación.
	\begin{itemize}
	\item[1.] El sistema cancela la confirmación y retorna al \textit{Paso 1}
	\end{itemize}
\item[6.] El sistema no puede realizar la acción solicitada.
	\begin{itemize}
	\item[1.] El sistema cancela la acción y muestra un error al usuario.
	\end{itemize}
\end{enumerate}
\\

\hline
\textbf{Postcondiciones} & \\
\hline
\end{longtable}



\newpage
\subsubsection*{Caso de uso: Consultar itinerario ruta }
\begin{longtable}{| p{4cm} | p{10cm} |}
\endfirsthead
\multicolumn{2}{c}{\textit{Continúa de la página anterior}}\\[12pt]
\hline
\endhead
\hline
\multicolumn{2}{c}{\textit{Continúa en la siguiente página}} \\
\endfoot
\hline
\caption{Caso de Uso: Consultar itinerario ruta}\label{fig:1}\\
\endlastfoot


\hline
\multicolumn{2}{|c|}{\textbf{CU$<$13$>$ - Consultar Itinerario Ruta}} \\

\hline
\textbf{Descripción} &
Obtiene la información de la ruta desglosada por los días que la componen.\\

\hline
\textbf{Actores} &
Cliente Móvil\newline
Cliente Web\\

\hline
\textbf{Precondiciones} &
La ruta tiene unas fechas de viaje asignadas.\\

\hline
\textbf{Secuencia Normal} &\mbox{}\par\vspace{-\baselineskip}
\begin{enumerate}[leftmargin=0.7cm, topsep=0.1cm]
\item El usuario selecciona la opción \textit{Itinerario}.
\item El sistema muestra en la parte superior un selector con los días y en la inferior la información correspondiente a cada día. Para cada día, el sistema muestra:
	\begin{itemize}
	\item [1.] El conjunto de visitas a lugares y/o eventos asignadas por el usuario.
	\item [2.] La hora de comienzo de la ruta para el día determinado.
	\item [3.] La hora de llegada, estimada, a cada visita.
	\item [4.] El tiempo asignado, como parada, en cada visita.
	\item [5.] La hora de salida, estimada, para cada visita.
	\item [6.] El modo de viaje, junto a su duración y distancia, entre cada visita.
	\item [7.] Un botón para eliminar la visita.
	\end{itemize}
\item El usuario cambia de día haciendo uso del selector.
\end{enumerate}


\\
\hline
\textbf{Excepciones} &\mbox{}\par\vspace{-\baselineskip}
\begin{enumerate}[leftmargin=0.9cm, topsep=0.1cm]
\item[2-3.] El usuario pulsa el botón de \textit{Atrás}.
	\begin{itemize}
	\item[1.] El sistema retorna a la pantalla anterior.
	\end{itemize}
\end{enumerate}
\\

\hline
\textbf{Postcondiciones} & \\
\hline
\end{longtable}



\newpage
\subsubsection*{Caso de uso: Editar orden visitas }
\begin{longtable}{| p{4cm} | p{10cm} |}
\endfirsthead
\multicolumn{2}{c}{\textit{Continúa de la página anterior}}\\[12pt]
\hline
\endhead
\hline
\multicolumn{2}{c}{\textit{Continúa en la siguiente página}} \\
\endfoot
\hline
\caption{Caso de Uso: Editar orden visitas}\label{fig:1}\\
\endlastfoot


\hline
\multicolumn{2}{|c|}{\textbf{CU$<$14$>$ - Editar Orden Visitas}} \\

\hline
\textbf{Descripción} &
Modifica el orden de las visitas establecidas para un día.\\

\hline
\textbf{Actores} &
Cliente Móvil\\

\hline
\textbf{Precondiciones} &
\\

\hline
\textbf{Secuencia Normal} &\mbox{}\par\vspace{-\baselineskip}
\begin{enumerate}[leftmargin=0.7cm, topsep=0.1cm]
\item El usuario selecciona la opción \textit{Habilitar Edición}.
\item El sistema muestra la vista de edición para las visitas.
\item El usuario modifica el orden de las visitas.
\item El sistema actualiza el orden establecido y recalcula las distancias y tiempos de viaje para el nuevo orden de las visitas.
\item El usuario selecciona \textit{Terminar Edición}.
\item El sistema vuelve a la vista normal.
\end{enumerate}


\\
\hline
\textbf{Excepciones} &\mbox{}\par\vspace{-\baselineskip}
\begin{enumerate}[leftmargin=0.9cm, topsep=0.1cm]
\item[3-5.] El usuario pulsa el botón de \textit{Atrás}.
	\begin{itemize}
	\item[1.] El sistema retorna a la pantalla anterior.
	\end{itemize}
\end{enumerate}
\\

\hline
\textbf{Postcondiciones} & \\
\hline
\end{longtable}



\newpage
\subsubsection*{Caso de uso: Modificar hora de salida }
\begin{longtable}{| p{4cm} | p{10cm} |}
\endfirsthead
\multicolumn{2}{c}{\textit{Continúa de la página anterior}}\\[12pt]
\hline
\endhead
\hline
\multicolumn{2}{c}{\textit{Continúa en la siguiente página}} \\
\endfoot
\hline
\caption{Caso de Uso: Modificar hora de salida}\label{fig:1}\\
\endlastfoot


\hline
\multicolumn{2}{|c|}{\textbf{CU$<$15$>$ - Modificar Hora de Salida}} \\

\hline
\textbf{Descripción} &
Modifica la hora de salida para un día concreto de la ruta.\\

\hline
\textbf{Actores} &
Cliente Móvil\\

\hline
\textbf{Precondiciones} &
\\

\hline
\textbf{Secuencia Normal} &\mbox{}\par\vspace{-\baselineskip}
\begin{enumerate}[leftmargin=0.7cm, topsep=0.1cm]
\item El usuario selecciona la opción \textit{Modificar hora de salida}.
\item El sistema muestra un formulario para indicar la hora deseada.
\item El usuario selecciona la hora en el formulario y pulsa en \textit{Confirmar}.
\item El sistema valida la hora y la actualiza.
\end{enumerate}


\\
\hline
\textbf{Excepciones} &\mbox{}\par\vspace{-\baselineskip}
\begin{enumerate}[leftmargin=0.9cm, topsep=0.1cm]
\item[3.] El usuario pulsa el botón de \textit{Cancelar}.
	\begin{itemize}
	\item[1.] El sistema deshecha el formulario y cancela la acción.
	\end{itemize}
\item[4.] El sistema no puede realizar la acción solicitada.
	\begin{itemize}
	\item[1.] El sistema cancela la acción y muestra un error al usuario.
	\end{itemize}
\end{enumerate}
\\

\hline
\textbf{Postcondiciones} & 
La hora de salida queda actualizada en el sistema.\\
\hline
\end{longtable}



\newpage
\subsubsection*{Caso de uso: Eliminar visita }
\begin{longtable}{| p{4cm} | p{10cm} |}
\endfirsthead
\multicolumn{2}{c}{\textit{Continúa de la página anterior}}\\[12pt]
\hline
\endhead
\hline
\multicolumn{2}{c}{\textit{Continúa en la siguiente página}} \\
\endfoot
\hline
\caption{Caso de Uso: Eliminar visita}\label{fig:1}\\
\endlastfoot


\hline
\multicolumn{2}{|c|}{\textbf{CU$<$16$>$ - Eliminar Visita}} \\

\hline
\textbf{Descripción} &
El usuario elimina una visita de un día de la ruta.\\

\hline
\textbf{Actores} &
Cliente Móvil\\

\hline
\textbf{Precondiciones} &
\\

\hline
\textbf{Secuencia Normal} &\mbox{}\par\vspace{-\baselineskip}
\begin{enumerate}[leftmargin=0.7cm, topsep=0.1cm]
\item El usuario hace uso del botón para eliminar la visita.
\item El sistema muestra una pantalla de confirmación.
\item El usuario confirma la acción.
\item El sistema elimina la visita.
\end{enumerate}


\\
\hline
\textbf{Excepciones} &\mbox{}\par\vspace{-\baselineskip}
\begin{enumerate}[leftmargin=0.9cm, topsep=0.1cm]
\item[3.] El usuario pulsa el botón de \textit{Cancelar}.
	\begin{itemize}
	\item[1.] El sistema cancela la eliminación de la visita.
	\end{itemize}
\item[4.] El sistema no puede realizar la acción solicitada.
	\begin{itemize}
	\item[1.] El sistema cancela la acción y muestra un error al usuario.
	\end{itemize}
\end{enumerate}
\\

\hline
\textbf{Postcondiciones} & \\
\hline
\end{longtable}



\newpage
\subsubsection*{Caso de uso: Modificar tiempo en la visita }
\begin{longtable}{| p{4cm} | p{10cm} |}
\endfirsthead
\multicolumn{2}{c}{\textit{Continúa de la página anterior}}\\[12pt]
\hline
\endhead
\hline
\multicolumn{2}{c}{\textit{Continúa en la siguiente página}} \\
\endfoot
\hline
\caption{Caso de Uso: Modificar tiempo en la visita}\label{fig:1}\\
\endlastfoot


\hline
\multicolumn{2}{|c|}{\textbf{CU$<$17$>$ - Modificar Tiempo en la Visita}} \\

\hline
\textbf{Descripción} &
El usuario modifica el tiempo de parada en una determinada visita.\\

\hline
\textbf{Actores} &
Cliente Móvil\\

\hline
\textbf{Precondiciones} &
\\

\hline
\textbf{Secuencia Normal} &\mbox{}\par\vspace{-\baselineskip}
\begin{enumerate}[leftmargin=0.7cm, topsep=0.1cm]
\item El usuario selecciona la opción \textit{Editar Tiempo Visita}.
\item El sistema muestra un formulario para indicar el tiempo deseado.
\item El usuario selecciona indica el tiempo en el formulario y pulsa en \textit{Confirmar}.
\item El sistema actualiza el tiempo para la visita.

\end{enumerate}


\\
\hline
\textbf{Excepciones} &\mbox{}\par\vspace{-\baselineskip}
\begin{enumerate}[leftmargin=0.9cm, topsep=0.1cm]
\item[3.] El usuario pulsa el botón de \textit{Cancelar}.
	\begin{itemize}
	\item[1.] El sistema cancela la operación.
	\end{itemize}
\end{enumerate}
\\

\hline
\textbf{Postcondiciones} & \\
\hline
\end{longtable}



\newpage
\subsubsection*{Caso de uso: Modificar modo de viaje }
\begin{longtable}{| p{4cm} | p{10cm} |}
\endfirsthead
\multicolumn{2}{c}{\textit{Continúa de la página anterior}}\\[12pt]
\hline
\endhead
\hline
\multicolumn{2}{c}{\textit{Continúa en la siguiente página}} \\
\endfoot
\hline
\caption{Caso de Uso: Modificar modo de viaje}\label{fig:1}\\
\endlastfoot


\hline
\multicolumn{2}{|c|}{\textbf{CU$<$18$>$ - Modificar Modo de Viaje}} \\

\hline
\textbf{Descripción} &
El usuario modifica el modo de viaje entre dos visitas. Los modos de viaje habilitados son: \textit{En coche}, \textit{En bicicleta} y \textit{Andando}\\

\hline
\textbf{Actores} &
Cliente Móvil\\

\hline
\textbf{Precondiciones} &
\\

\hline
\textbf{Secuencia Normal} &\mbox{}\par\vspace{-\baselineskip}
\begin{enumerate}[leftmargin=0.7cm, topsep=0.1cm]
\item El usuario selecciona la opción \textit{Modificar Modo de Viaje}.
\item Si el modo de viaje era \textit{Andando}
	\begin{itemize}
	\item [1.] El sistema actualiza el modo de viaje a \textit{En coche}.
	\end{itemize}
\item Si el modo de viaje era \textit{En coche}
	\begin{itemize}
	\item [1.] El sistema actualiza el modo de viaje a \textit{En bicicleta}.
	\end{itemize}
\item Si el modo de viaje era \textit{En bicicleta}
	\begin{itemize}
	\item [1.] El sistema actualiza el modo de viaje a \textit{Andando}.
	\end{itemize}

\end{enumerate}


\\
\hline
\textbf{Excepciones} &\mbox{}\par\vspace{-\baselineskip}
\\

\hline
\textbf{Postcondiciones} & \\
\hline
\end{longtable}



\newpage
\subsubsection*{Caso de uso: Consultar lugares }
\begin{longtable}{| p{4cm} | p{10cm} |}
\endfirsthead
\multicolumn{2}{c}{\textit{Continúa de la página anterior}}\\[12pt]
\hline
\endhead
\hline
\multicolumn{2}{c}{\textit{Continúa en la siguiente página}} \\
\endfoot
\hline
\caption{Caso de Uso: Consultar lugares}\label{fig:1}\\
\endlastfoot


\hline
\multicolumn{2}{|c|}{\textbf{CU$<$19$>$ - Consultar lugares}} \\

\hline
\textbf{Descripción} &
El usuario consulta los diferentes lugares en función de diferentes criterios.\\

\hline
\textbf{Actores} &
Cliente Móvil\\

\hline
\textbf{Precondiciones} &
\\

\hline
\textbf{Secuencia Normal} &\mbox{}\par\vspace{-\baselineskip}
\begin{enumerate}[leftmargin=0.7cm, topsep=0.1cm]
\item El sistema muestra en la parte superior un selector y en la inferior los lugares más relevantes para la ciudad o lugar donde está creada la ruta.
\item Si el usuario selecciona la opción \textit{Lista} en el selector.
	\begin{itemize}
	\item[1.] El sistema muestra los lugares en una lista.
	\item[2.] Para cada lugar el sistema incluye.
		\begin{itemize}
		\item[1.] Una pequeña foto.
		\item[2.] El nombre del lugar.
		\item[3.] La categoría a la que pertenece.
		\item[4.] La dirección en la que se encuentra.
		\item[5.] Un indicador con el número de días en el que está asignado dicho lugar en la ruta del usuario.
		\item[6.] Un botón para asignar o desasignar el lugar a los días de la ruta.
		\end{itemize}
	\end{itemize}
\item Si el usuario selecciona la opción \textit{Mapa} en el selector.
	\begin{itemize}
	\item[1.] El sistema muestra los lugares en el mapa.
	\item[2.] Para cada lugar el sistema incluye.
		\begin{itemize}
		\item[1.] Una marca, que incluye nombre y categoría, en la ubicación exacta en el mapa.
		\item[2.] Un color diferente en función de si el lugar está asignado o no a la ruta.
		\item[3.] Un botón para asignar o desasignar el lugar a los días de la ruta.
		\end{itemize}
	\end{itemize}
\end{enumerate}


\\
\hline
\textbf{Excepciones} &\mbox{}\par\vspace{-\baselineskip}
\begin{enumerate}[leftmargin=0.9cm, topsep=0.1cm]
\item[2-3.] El usuario pulsa el botón de \textit{Atrás}.
	\begin{itemize}
	\item[1.] El sistema retorna a la pantalla anterior.
	\end{itemize}
\end{enumerate}
\\

\hline
\textbf{Postcondiciones} & \\
\hline
\end{longtable}


\newpage
\subsubsection*{Caso de uso: Asignar lugares a ruta }
\begin{longtable}{| p{4cm} | p{10cm} |}
\endfirsthead
\multicolumn{2}{c}{\textit{Continúa de la página anterior}}\\[12pt]
\hline
\endhead
\hline
\multicolumn{2}{c}{\textit{Continúa en la siguiente página}} \\
\endfoot
\hline
\caption{Caso de Uso: Asginar lugares a ruta}\label{fig:1}\\
\endlastfoot


\hline
\multicolumn{2}{|c|}{\textbf{CU$<$20$>$ - Asignar Lugares a Ruta}} \\

\hline
\textbf{Descripción} &
El usuario añade o elimina lugares a la ruta.\\

\hline
\textbf{Actores} &
Cliente Móvil\\

\hline
\textbf{Precondiciones} &
\\

\hline
\textbf{Secuencia Normal} &\mbox{}\par\vspace{-\baselineskip}
\begin{enumerate}[leftmargin=0.7cm, topsep=0.1cm]
\item El usuario selecciona la opción \textit{Añadir Lugar}.
\item El sistema implementa \textit{CU$<$19$>$ - Consultar Lugares}
\item El usuario hace click en el botón para asignar o desasignar el lugar.
\item El sistema muestra una ventana con el conjunto de días que forman la ruta de los cuáles, los que aparecen activados, indican que el lugar ya se encuentra asignado a ese día.
\item El usuario, seleccionando y deseleccionandao, indica los días que desea visitar determinado lugar y pulsa en \textit{Aceptar}.
\item El sistema confirma la acción y actualiza la ruta del usuario.

\end{enumerate}


\\
\hline
\textbf{Excepciones} &\mbox{}\par\vspace{-\baselineskip}
\begin{enumerate}[leftmargin=0.9cm, topsep=0.1cm]
\item[3.] El usuario pulsa el botón de \textit{Atrás}.
	\begin{itemize}
	\item[1.] El sistema retorna a la pantalla anterior.
	\end{itemize}
\item[5.] El usuario pulsa el botón de \textit{Cancelar}.
	\begin{itemize}
	\item[1.] El sistema cancela la acción y no actualiza la información de la ruta.
	\end{itemize}
\end{enumerate}
\\

\hline
\textbf{Postcondiciones} & \\
\hline
\end{longtable}



\newpage
\subsubsection*{Caso de uso: Modificar datos personales }
\begin{longtable}{| p{4cm} | p{10cm} |}
\endfirsthead
\multicolumn{2}{c}{\textit{Continúa de la página anterior}}\\[12pt]
\hline
\endhead
\hline
\multicolumn{2}{c}{\textit{Continúa en la siguiente página}} \\
\endfoot
\hline
\caption{Caso de Uso: Modificar datos personales}\label{fig:1}\\
\endlastfoot


\hline
\multicolumn{2}{|c|}{\textbf{CU$<$21$>$ - Modificar Datos Personales}} \\

\hline
\textbf{Descripción} &
El usuario modifica sus datos en el sistema.\\

\hline
\textbf{Actores} &
Cliente Móvil\\

\hline
\textbf{Precondiciones} &
El usuario está autenticado en el sistema.\\

\hline
\textbf{Secuencia Normal} &\mbox{}\par\vspace{-\baselineskip}
\begin{enumerate}[leftmargin=0.7cm, topsep=0.1cm]
\item El usuario selecciona la opción \textit{Editar Datos}.
\item El sistema muestra un formulario con los datos actuales del usuario.
\item El usuario modifica sus datos y pulsa el botón \textit{Aceptar}.
\item El sistema actualiza los datos del usuario.
\end{enumerate}


\\
\hline
\textbf{Excepciones} &\mbox{}\par\vspace{-\baselineskip}
\begin{enumerate}[leftmargin=0.9cm, topsep=0.1cm]
\item[3.] El usuario pulsa el botón de \textit{Cancelar}.
	\begin{itemize}
	\item[1.] El sistema cancela la acción y no actualiza la información del usuario.
	\end{itemize}
\end{enumerate}
\\

\hline
\textbf{Postcondiciones} & 
Los datos del usuario quedan actualizados en el sistema.\\
\hline
\end{longtable}



\newpage
\subsubsection*{Caso de uso: Consultar usuarios }
\begin{longtable}{| p{4cm} | p{10cm} |}
\endfirsthead
\multicolumn{2}{c}{\textit{Continúa de la página anterior}}\\[12pt]
\hline
\endhead
\hline
\multicolumn{2}{c}{\textit{Continúa en la siguiente página}} \\
\endfoot
\hline
\caption{Caso de Uso: Consultar usuarios}\label{fig:1}\\
\endlastfoot


\hline
\multicolumn{2}{|c|}{\textbf{CU$<$22$>$ - Consultar Usuarios}} \\

\hline
\textbf{Descripción} &
Muestra los diferentes usuarios registrados en el sistema.\\

\hline
\textbf{Actores} &
Administrador\\

\hline
\textbf{Precondiciones} &
El administrador está autenticado en el sistema.\\

\hline
\textbf{Secuencia Normal} &\mbox{}\par\vspace{-\baselineskip}
\begin{enumerate}[leftmargin=0.7cm, topsep=0.1cm]
\item El administrador selecciona la opción \textit{Consultar Usuarios}.
\item El sistema muestra una tabla con todos los usuarios del sistema.
\end{enumerate}


\\
\hline
\textbf{Excepciones} &\mbox{}\par\vspace{-\baselineskip}
\\

\hline
\textbf{Postcondiciones} & \\
\hline
\end{longtable}



\newpage
\subsubsection*{Caso de uso: Alta usuario }
\begin{longtable}{| p{4cm} | p{10cm} |}
\endfirsthead
\multicolumn{2}{c}{\textit{Continúa de la página anterior}}\\[12pt]
\hline
\endhead
\hline
\multicolumn{2}{c}{\textit{Continúa en la siguiente página}} \\
\endfoot
\hline
\caption{Caso de Uso: Alta usuario}\label{fig:1}\\
\endlastfoot


\hline
\multicolumn{2}{|c|}{\textbf{CU$<$23$>$ - Alta Usuario}} \\

\hline
\textbf{Descripción} &
Añade un nuevo usuario al sistema.\\

\hline
\textbf{Actores} &
Administrador\\

\hline
\textbf{Precondiciones} &
El administrador está autenticado en el sistema.\\

\hline
\textbf{Secuencia Normal} &\mbox{}\par\vspace{-\baselineskip}
\begin{enumerate}[leftmargin=0.7cm, topsep=0.1cm]
\item El administrador selecciona la opción \textit{Añadir Usuario}.
\item El sistema muestra un formulario con los datos a rellenar por el usuario.
\item El administrador introduce los datos y pulsa sobre el botón \textit{Confirmar}.
\item El sistema comprueba los datos y los inserta en el sistema.
\end{enumerate}


\\
\hline
\textbf{Excepciones} &\mbox{}\par\vspace{-\baselineskip}
\begin{enumerate}[leftmargin=0.9cm, topsep=0.1cm]
\item[3.] El administrador pulsa el botón de \textit{Cancelar}.
	\begin{itemize}
	\item[1.] El sistema cancela la acción y detiene el proceso de alta.
	\end{itemize}
\item[4.] El sistema no acepta los datos introducidos.
	\begin{itemize}
	\item[1.] El sistema cancela la acción y muestra un error al usuario.
	\end{itemize}
\end{enumerate}
\\

\hline
\textbf{Postcondiciones} & \\
\hline
\end{longtable}



\newpage
\subsubsection*{Caso de uso: Modificar usuario }
\begin{longtable}{| p{4cm} | p{10cm} |}
\endfirsthead
\multicolumn{2}{c}{\textit{Continúa de la página anterior}}\\[12pt]
\hline
\endhead
\hline
\multicolumn{2}{c}{\textit{Continúa en la siguiente página}} \\
\endfoot
\hline
\caption{Caso de Uso: Modificar usuario}\label{fig:1}\\
\endlastfoot


\hline
\multicolumn{2}{|c|}{\textbf{CU$<$24$>$ - Modificar Usuario}} \\

\hline
\textbf{Descripción} &
Modifica los datos de un usuario.\\

\hline
\textbf{Actores} &
Administrador\\

\hline
\textbf{Precondiciones} &
El administrador está autenticado en el sistema.\\

\hline
\textbf{Secuencia Normal} &\mbox{}\par\vspace{-\baselineskip}
\begin{enumerate}[leftmargin=0.7cm, topsep=0.1cm]
\item El administrador selecciona la opción de \textit{Modificar} sobre un usuario.
\item El sistema muestra un formulario con los datos actuales del usuario seleccionado.
\item El administrador modifica los datos que desea y pulsa sobre el botón \textit{Confirmar}.
\item El sistema comprueba los datos y los actualiza.
\end{enumerate}


\\
\hline
\textbf{Excepciones} &\mbox{}\par\vspace{-\baselineskip}
\begin{enumerate}[leftmargin=0.9cm, topsep=0.1cm]
\item[3.] El administrador pulsa el botón de \textit{Cancelar}.
	\begin{itemize}
	\item[1.] El sistema cancela la acción y no actualiza la información del usuario.
	\end{itemize}
\item[4.] El sistema no acepta los datos introducidos.
	\begin{itemize}
	\item[1.] El sistema cancela la acción y muestra un error al usuario.
	\end{itemize}
\end{enumerate}
\\

\hline
\textbf{Postcondiciones} & \\
\hline
\end{longtable}



\newpage
\subsubsection*{Caso de uso: Eliminar usuario }
\begin{longtable}{| p{4cm} | p{10cm} |}
\endfirsthead
\multicolumn{2}{c}{\textit{Continúa de la página anterior}}\\[12pt]
\hline
\endhead
\hline
\multicolumn{2}{c}{\textit{Continúa en la siguiente página}} \\
\endfoot
\hline
\caption{Caso de Uso: Eliminar usuario}\label{fig:1}\\
\endlastfoot


\hline
\multicolumn{2}{|c|}{\textbf{CU$<$25$>$ - Eliminar Usuario}} \\

\hline
\textbf{Descripción} &
Elimina un usuario del sistema.\\

\hline
\textbf{Actores} &
Administrador\\

\hline
\textbf{Precondiciones} &
El administrador está autenticado en el sistema.\\

\hline
\textbf{Secuencia Normal} &\mbox{}\par\vspace{-\baselineskip}
\begin{enumerate}[leftmargin=0.7cm, topsep=0.1cm]
\item El administrador selecciona la opción de \textit{Eliminar} sobre un usuario.
\item El sistema muestra una pantalla de confirmación.
\item El administrador pulsa sobre el botón de \textit{Confirmar}.
\item El sistema realiza la acción.
\end{enumerate}


\\
\hline
\textbf{Excepciones} &\mbox{}\par\vspace{-\baselineskip}
\begin{enumerate}[leftmargin=0.9cm, topsep=0.1cm]
\item[3.] El administrador pulsa el botón de \textit{Cancelar}.
	\begin{itemize}
	\item[1.] El sistema cancela la acción y detiene el proceso de eliminación.
	\end{itemize}
\end{enumerate}
\\

\hline
\textbf{Postcondiciones} & \\
\hline
\end{longtable}



\newpage
\subsubsection*{Caso de uso: Consultar rutas }
\begin{longtable}{| p{4cm} | p{10cm} |}
\endfirsthead
\multicolumn{2}{c}{\textit{Continúa de la página anterior}}\\[12pt]
\hline
\endhead
\hline
\multicolumn{2}{c}{\textit{Continúa en la siguiente página}} \\
\endfoot
\hline
\caption{Caso de Uso: Consultar rutas}\label{fig:1}\\
\endlastfoot


\hline
\multicolumn{2}{|c|}{\textbf{CU$<$26$>$ - Consultar Rutas}} \\

\hline
\textbf{Descripción} &
Muestra los diferentes rutas registradas en el sistema.\\

\hline
\textbf{Actores} &
Administrador\\

\hline
\textbf{Precondiciones} &
El usuario está autenticado en el sistema.\\

\hline
\textbf{Secuencia Normal} &\mbox{}\par\vspace{-\baselineskip}
\begin{enumerate}[leftmargin=0.7cm, topsep=0.1cm]
\item El usuario selecciona la opción \textit{Consultar Rutas}.
\item El sistema muestra una tabla con todas las rutas del sistema.
\end{enumerate}


\\
\hline
\textbf{Excepciones} &\mbox{}\par\vspace{-\baselineskip}
\\

\hline
\textbf{Postcondiciones} & \\
\hline
\end{longtable}



\newpage
\subsubsection*{Caso de uso: Alta ruta }
\begin{longtable}{| p{4cm} | p{10cm} |}
\endfirsthead
\multicolumn{2}{c}{\textit{Continúa de la página anterior}}\\[12pt]
\hline
\endhead
\hline
\multicolumn{2}{c}{\textit{Continúa en la siguiente página}} \\
\endfoot
\hline
\caption{Caso de Uso: Alta ruta}\label{fig:1}\\
\endlastfoot


\hline
\multicolumn{2}{|c|}{\textbf{CU$<$27$>$ - Alta Ruta}} \\

\hline
\textbf{Descripción} &
Añade una nueva ruta al sistema.\\

\hline
\textbf{Actores} &
Administrador\\

\hline
\textbf{Precondiciones} &
El usuario está autenticado en el sistema.\\

\hline
\textbf{Secuencia Normal} &\mbox{}\par\vspace{-\baselineskip}
\begin{enumerate}[leftmargin=0.7cm, topsep=0.1cm]
\item El usuario selecciona la opción \textit{Añadir Ruta}.
\item El sistema muestra un formulario con los datos a rellenar por el usuario.
\item El usuario introduce los datos y pulsa sobre el botón \textit{Confirmar}.
\item El sistema comprueba los datos y los inserta en el sistema.
\end{enumerate}


\\
\hline
\textbf{Excepciones} &\mbox{}\par\vspace{-\baselineskip}
\begin{enumerate}[leftmargin=0.9cm, topsep=0.1cm]
\item[3.] El usuario pulsa el botón de \textit{Cancelar}.
	\begin{itemize}
	\item[1.] El sistema cancela la acción y detiene el proceso de alta.
	\end{itemize}
\item[4.] El sistema no acepta los datos introducidos.
	\begin{itemize}
	\item[1.] El sistema cancela la acción y muestra un error al usuario.
	\end{itemize}
\end{enumerate}
\\

\hline
\textbf{Postcondiciones} & \\
\hline
\end{longtable}



\newpage
\subsubsection*{Caso de uso: Modificar ruta }
\begin{longtable}{| p{4cm} | p{10cm} |}
\endfirsthead
\multicolumn{2}{c}{\textit{Continúa de la página anterior}}\\[12pt]
\hline
\endhead
\hline
\multicolumn{2}{c}{\textit{Continúa en la siguiente página}} \\
\endfoot
\hline
\caption{Caso de Uso: Modificar ruta}\label{fig:1}\\
\endlastfoot


\hline
\multicolumn{2}{|c|}{\textbf{CU$<$28$>$ - Modificar Ruta}} \\

\hline
\textbf{Descripción} &
Modifica los datos de una ruta.\\

\hline
\textbf{Actores} &
Administrador\\

\hline
\textbf{Precondiciones} &
El usuario está autenticado en el sistema.\\

\hline
\textbf{Secuencia Normal} &\mbox{}\par\vspace{-\baselineskip}
\begin{enumerate}[leftmargin=0.7cm, topsep=0.1cm]
\item El usuario selecciona la opción de \textit{Modificar} sobre una ruta.
\item El sistema muestra un formulario con los datos actuales de la ruta seleccionada.
\item El usuario modifica los datos que desea y pulsa sobre el botón \textit{Confirmar}.
\item El sistema comprueba los datos y los actualiza.
\end{enumerate}


\\
\hline
\textbf{Excepciones} &\mbox{}\par\vspace{-\baselineskip}
\begin{enumerate}[leftmargin=0.9cm, topsep=0.1cm]
\item[3.] El usuario pulsa el botón de \textit{Cancelar}.
	\begin{itemize}
	\item[1.] El sistema cancela la acción y no actualiza la información de la ruta.
	\end{itemize}
\item[4.] El sistema no acepta los datos introducidos.
	\begin{itemize}
	\item[1.] El sistema cancela la acción y muestra un error al usuario.
	\end{itemize}
\end{enumerate}
\\

\hline
\textbf{Postcondiciones} & \\
\hline
\end{longtable}



\newpage
\subsubsection*{Caso de uso: Eliminar ruta }
\begin{longtable}{| p{4cm} | p{10cm} |}
\endfirsthead
\multicolumn{2}{c}{\textit{Continúa de la página anterior}}\\[12pt]
\hline
\endhead
\hline
\multicolumn{2}{c}{\textit{Continúa en la siguiente página}} \\
\endfoot
\hline
\caption{Caso de Uso: Eliminar ruta}\label{fig:1}\\
\endlastfoot


\hline
\multicolumn{2}{|c|}{\textbf{CU$<$29$>$ - Eliminar Ruta}} \\

\hline
\textbf{Descripción} &
Elimina una ruta del sistema.\\

\hline
\textbf{Actores} &
Administrador\\

\hline
\textbf{Precondiciones} &
El usuario está autenticado en el sistema.\\

\hline
\textbf{Secuencia Normal} &\mbox{}\par\vspace{-\baselineskip}
\begin{enumerate}[leftmargin=0.7cm, topsep=0.1cm]
\item El usuario selecciona la opción de \textit{Eliminar} sobre una ruta.
\item El sistema muestra una pantalla de confirmación.
\item El usuario pulsa sobre el botón de \textit{Confirmar}.
\item El sistema realiza la acción.
\end{enumerate}


\\
\hline
\textbf{Excepciones} &\mbox{}\par\vspace{-\baselineskip}
\begin{enumerate}[leftmargin=0.9cm, topsep=0.1cm]
\item[3.] El usuario pulsa el botón de \textit{Cancelar}.
	\begin{itemize}
	\item[1.] El sistema cancela la acción y detiene el proceso de eliminación.
	\end{itemize}
\end{enumerate}
\\

\hline
\textbf{Postcondiciones} & \\
\hline
\end{longtable}



\newpage
\subsubsection*{Caso de uso: Consultar lugares }
\begin{longtable}{| p{4cm} | p{10cm} |}
\endfirsthead
\multicolumn{2}{c}{\textit{Continúa de la página anterior}}\\[12pt]
\hline
\endhead
\hline
\multicolumn{2}{c}{\textit{Continúa en la siguiente página}} \\
\endfoot
\hline
\caption{Caso de Uso: Consultar lugares}\label{fig:1}\\
\endlastfoot


\hline
\multicolumn{2}{|c|}{\textbf{CU$<$30$>$ - Consultar Lugares}} \\

\hline
\textbf{Descripción} &
Muestra los diferentes lugares registrados en el sistema.\\

\hline
\textbf{Actores} &
Administrador\\

\hline
\textbf{Precondiciones} &
El usuario está autenticado en el sistema.\\

\hline
\textbf{Secuencia Normal} &\mbox{}\par\vspace{-\baselineskip}
\begin{enumerate}[leftmargin=0.7cm, topsep=0.1cm]
\item El usuario selecciona la opción \textit{Consultar Lugares}.
\item El sistema muestra una tabla con todos los lugares del sistema.
\end{enumerate}


\\
\hline
\textbf{Excepciones} &\mbox{}\par\vspace{-\baselineskip}
\\

\hline
\textbf{Postcondiciones} & \\
\hline
\end{longtable}



\newpage
\subsubsection*{Caso de uso: Modificar lugar }
\begin{longtable}{| p{4cm} | p{10cm} |}
\endfirsthead
\multicolumn{2}{c}{\textit{Continúa de la página anterior}}\\[12pt]
\hline
\endhead
\hline
\multicolumn{2}{c}{\textit{Continúa en la siguiente página}} \\
\endfoot
\hline
\caption{Caso de Uso: Modificar lugar}\label{fig:1}\\
\endlastfoot


\hline
\multicolumn{2}{|c|}{\textbf{CU$<$31$>$ - Modificar Lugar}} \\

\hline
\textbf{Descripción} &
Modifica los datos de un lugar.\\

\hline
\textbf{Actores} &
Administrador\\

\hline
\textbf{Precondiciones} &
El usuario está autenticado en el sistema.\\

\hline
\textbf{Secuencia Normal} &\mbox{}\par\vspace{-\baselineskip}
\begin{enumerate}[leftmargin=0.7cm, topsep=0.1cm]
\item El usuario selecciona la opción de \textit{Modificar} sobre un lugar.
\item El sistema muestra un formulario con los datos actuales del lugar seleccionado.
\item El usuario modifica los datos que desea y pulsa sobre el botón \textit{Confirmar}.
\item El sistema comprueba los datos y los actualiza.
\end{enumerate}


\\
\hline
\textbf{Excepciones} &\mbox{}\par\vspace{-\baselineskip}
\begin{enumerate}[leftmargin=0.9cm, topsep=0.1cm]
\item[3.] El usuario pulsa el botón de \textit{Cancelar}.
	\begin{itemize}
	\item[1.] El sistema cancela la acción y no actualiza la información del lugar.
	\end{itemize}
\item[4.] El sistema no acepta los datos introducidos.
	\begin{itemize}
	\item[1.] El sistema cancela la acción y muestra un error al usuario.
	\end{itemize}
\end{enumerate}
\\

\hline
\textbf{Postcondiciones} & \\
\hline
\end{longtable}



\newpage
\subsubsection*{Caso de uso: Eliminar lugar }
\begin{longtable}{| p{4cm} | p{10cm} |}
\endfirsthead
\multicolumn{2}{c}{\textit{Continúa de la página anterior}}\\[12pt]
\hline
\endhead
\hline
\multicolumn{2}{c}{\textit{Continúa en la siguiente página}} \\
\endfoot
\hline
\caption{Caso de Uso: Eliminar lugar}\label{fig:1}\\
\endlastfoot


\hline
\multicolumn{2}{|c|}{\textbf{CU$<$32$>$ - Eliminar Lugar}} \\

\hline
\textbf{Descripción} &
Elimina un lugar del sistema.\\

\hline
\textbf{Actores} &
Administrador\\

\hline
\textbf{Precondiciones} &
El usuario está autenticado en el sistema.\\

\hline
\textbf{Secuencia Normal} &\mbox{}\par\vspace{-\baselineskip}
\begin{enumerate}[leftmargin=0.7cm, topsep=0.1cm]
\item El usuario selecciona la opción de \textit{Eliminar} sobre un lugar.
\item El sistema muestra una pantalla de confirmación.
\item El usuario pulsa sobre el botón de \textit{Confirmar}.
\item El sistema realiza la acción.
\end{enumerate}


\\
\hline
\textbf{Excepciones} &\mbox{}\par\vspace{-\baselineskip}
\begin{enumerate}[leftmargin=0.9cm, topsep=0.1cm]
\item[3.] El usuario pulsa el botón de \textit{Cancelar}.
	\begin{itemize}
	\item[1.] El sistema cancela la acción y detiene el proceso de eliminación.
	\end{itemize}
\end{enumerate}
\\

\hline
\textbf{Postcondiciones} & \\
\hline
\end{longtable}



\newpage
\subsubsection*{Caso de uso: Consultar eventos }
\begin{longtable}{| p{4cm} | p{10cm} |}
\endfirsthead
\multicolumn{2}{c}{\textit{Continúa de la página anterior}}\\[12pt]
\hline
\endhead
\hline
\multicolumn{2}{c}{\textit{Continúa en la siguiente página}} \\
\endfoot
\hline
\caption{Caso de Uso: Consultar eventos}\label{fig:1}\\
\endlastfoot


\hline
\multicolumn{2}{|c|}{\textbf{CU$<$33$>$ - Consultar Eventos}} \\

\hline
\textbf{Descripción} &
Muestra los diferentes usuarios registrados en el sistema.\\

\hline
\textbf{Actores} &
Administrador\newline
Gestor de Eventos\\

\hline
\textbf{Precondiciones} &
El usuario está autenticado en el sistema.\\

\hline
\textbf{Secuencia Normal} &\mbox{}\par\vspace{-\baselineskip}
\begin{enumerate}[leftmargin=0.7cm, topsep=0.1cm]
\item El usuario selecciona la opción \textit{Consultar Eventos}.
\item El sistema muestra una tabla con todos los eventos del sistema.
\end{enumerate}


\\
\hline
\textbf{Excepciones} &\mbox{}\par\vspace{-\baselineskip}
\\

\hline
\textbf{Postcondiciones} & \\
\hline
\end{longtable}



\newpage
\subsubsection*{Caso de uso: Alta evento }
\begin{longtable}{| p{4cm} | p{10cm} |}
\endfirsthead
\multicolumn{2}{c}{\textit{Continúa de la página anterior}}\\[12pt]
\hline
\endhead
\hline
\multicolumn{2}{c}{\textit{Continúa en la siguiente página}} \\
\endfoot
\hline
\caption{Caso de Uso: Alta evento}\label{fig:1}\\
\endlastfoot


\hline
\multicolumn{2}{|c|}{\textbf{CU$<$34$>$ - Alta Evento}} \\

\hline
\textbf{Descripción} &
Añade un nuevo evento al sistema.\\

\hline
\textbf{Actores} &
Administrador\newline
Gestor de Eventos\\

\hline
\textbf{Precondiciones} &
El usuario está autenticado en el sistema.\\

\hline
\textbf{Secuencia Normal} &\mbox{}\par\vspace{-\baselineskip}
\begin{enumerate}[leftmargin=0.7cm, topsep=0.1cm]
\item El usuario selecciona la opción \textit{Añadir Evento}.
\item El sistema muestra un formulario con los datos a rellenar por el usuario.
\item El usuario introduce los datos y pulsa sobre el botón \textit{Confirmar}.
\item El sistema comprueba los datos y los inserta en el sistema.
\end{enumerate}


\\
\hline
\textbf{Excepciones} &\mbox{}\par\vspace{-\baselineskip}
\begin{enumerate}[leftmargin=0.9cm, topsep=0.1cm]
\item[3.] El usuario pulsa el botón de \textit{Cancelar}.
	\begin{itemize}
	\item[1.] El sistema cancela la acción y detiene el proceso de alta.
	\end{itemize}
\item[4.] El sistema no acepta los datos introducidos.
	\begin{itemize}
	\item[1.] El sistema cancela la acción y muestra un error al usuario.
	\end{itemize}
\end{enumerate}
\\

\hline
\textbf{Postcondiciones} & \\
\hline
\end{longtable}



\newpage
\subsubsection*{Caso de uso: Modificar evento }
\begin{longtable}{| p{4cm} | p{10cm} |}
\endfirsthead
\multicolumn{2}{c}{\textit{Continúa de la página anterior}}\\[12pt]
\hline
\endhead
\hline
\multicolumn{2}{c}{\textit{Continúa en la siguiente página}} \\
\endfoot
\hline
\caption{Caso de Uso: Modificar evento}\label{fig:1}\\
\endlastfoot


\hline
\multicolumn{2}{|c|}{\textbf{CU$<$35$>$ - Modificar Evento}} \\

\hline
\textbf{Descripción} &
Modifica los datos de un evento.\\

\hline
\textbf{Actores} &
Administrador\newline
Gestor de Eventos\\

\hline
\textbf{Precondiciones} &
El usuario está autenticado en el sistema.\\

\hline
\textbf{Secuencia Normal} &\mbox{}\par\vspace{-\baselineskip}
\begin{enumerate}[leftmargin=0.7cm, topsep=0.1cm]
\item El usuario selecciona la opción de \textit{Modificar} sobre un evento.
\item El sistema muestra un formulario con los datos actuales del evento seleccionado.
\item El usuario modifica los datos que desea y pulsa sobre el botón \textit{Confirmar}.
\item El sistema comprueba los datos y los actualiza.
\end{enumerate}


\\
\hline
\textbf{Excepciones} &\mbox{}\par\vspace{-\baselineskip}
\begin{enumerate}[leftmargin=0.9cm, topsep=0.1cm]
\item[3.] El usuario pulsa el botón de \textit{Cancelar}.
	\begin{itemize}
	\item[1.] El sistema cancela la acción y no actualiza la información del evento.
	\end{itemize}
\item[4.] El sistema no acepta los datos introducidos.
	\begin{itemize}
	\item[1.] El sistema cancela la acción y muestra un error al usuario.
	\end{itemize}
\end{enumerate}
\\

\hline
\textbf{Postcondiciones} & \\
\hline
\end{longtable}



\newpage
\subsubsection*{Caso de uso: Eliminar evento }
\begin{longtable}{| p{4cm} | p{10cm} |}
\endfirsthead
\multicolumn{2}{c}{\textit{Continúa de la página anterior}}\\[12pt]
\hline
\endhead
\hline
\multicolumn{2}{c}{\textit{Continúa en la siguiente página}} \\
\endfoot
\hline
\caption{Caso de Uso: Eliminar evento}\label{fig:1}\\
\endlastfoot


\hline
\multicolumn{2}{|c|}{\textbf{CU$<$36$>$ - Eliminar Evento}} \\

\hline
\textbf{Descripción} &
Elimina un evento del sistema.\\

\hline
\textbf{Actores} &
Administrador\newline
Gestor de Eventos\\

\hline
\textbf{Precondiciones} &
El usuario está autenticado en el sistema.\\

\hline
\textbf{Secuencia Normal} &\mbox{}\par\vspace{-\baselineskip}
\begin{enumerate}[leftmargin=0.7cm, topsep=0.1cm]
\item El usuario selecciona la opción de \textit{Eliminar} sobre un evento.
\item El sistema muestra una pantalla de confirmación.
\item El usuario pulsa sobre el botón de \textit{Confirmar}.
\item El sistema realiza la acción.
\end{enumerate}


\\
\hline
\textbf{Excepciones} &\mbox{}\par\vspace{-\baselineskip}
\begin{enumerate}[leftmargin=0.9cm, topsep=0.1cm]
\item[3.] El usuario pulsa el botón de \textit{Cancelar}.
	\begin{itemize}
	\item[1.] El sistema cancela la acción y detiene el proceso de eliminación.
	\end{itemize}
\end{enumerate}
\\

\hline
\textbf{Postcondiciones} & \\
\hline
\end{longtable}



\newpage
\subsubsection*{Caso de uso: Consultar categorías }
\begin{longtable}{| p{4cm} | p{10cm} |}
\endfirsthead
\multicolumn{2}{c}{\textit{Continúa de la página anterior}}\\[12pt]
\hline
\endhead
\hline
\multicolumn{2}{c}{\textit{Continúa en la siguiente página}} \\
\endfoot
\hline
\caption{Caso de Uso: Consultar categorías}\label{fig:1}\\
\endlastfoot


\hline
\multicolumn{2}{|c|}{\textbf{CU$<$37$>$ - Consultar Categorías}} \\

\hline
\textbf{Descripción} &
Muestra los diferentes categorías registradas en el sistema.\\

\hline
\textbf{Actores} &
Administrador\\

\hline
\textbf{Precondiciones} &
El usuario está autenticado en el sistema.\\

\hline
\textbf{Secuencia Normal} &\mbox{}\par\vspace{-\baselineskip}
\begin{enumerate}[leftmargin=0.7cm, topsep=0.1cm]
\item El usuario selecciona la opción \textit{Consultar Categorías}.
\item El sistema muestra una tabla con todas las categorías del sistema.
\end{enumerate}


\\
\hline
\textbf{Excepciones} &\mbox{}\par\vspace{-\baselineskip}
\\

\hline
\textbf{Postcondiciones} & \\
\hline
\end{longtable}



\newpage
\subsubsection*{Caso de uso: Cargar categorías }
\begin{longtable}{| p{4cm} | p{10cm} |}
\endfirsthead
\multicolumn{2}{c}{\textit{Continúa de la página anterior}}\\[12pt]
\hline
\endhead
\hline
\multicolumn{2}{c}{\textit{Continúa en la siguiente página}} \\
\endfoot
\hline
\caption{Caso de Uso: Cargar categorías}\label{fig:1}\\
\endlastfoot


\hline
\multicolumn{2}{|c|}{\textbf{CU$<$38$>$ - Cargar Categorías}} \\

\hline
\textbf{Descripción} &
Obtiene las categorías de una fuente externa y las actualiza en el sistema.\\

\hline
\textbf{Actores} &
Administrador\\

\hline
\textbf{Precondiciones} &
El usuario está autenticado en el sistema.\\

\hline
\textbf{Secuencia Normal} &\mbox{}\par\vspace{-\baselineskip}
\begin{enumerate}[leftmargin=0.7cm, topsep=0.1cm]
\item El usuario selecciona la opción \textit{Cargar Categorías}.
\item El sistema muestra una pantalla de confirmación.
\item El usuario pulsa sobre el botón de \textit{Confirmar}.
\item El sistema realiza la acción.
\end{enumerate}


\\
\hline
\textbf{Excepciones} &\mbox{}\par\vspace{-\baselineskip}
\begin{enumerate}[leftmargin=0.9cm, topsep=0.1cm]
\item[3.] El usuario pulsa el botón de \textit{Cancelar}.
	\begin{itemize}
	\item[1.] El sistema cancela la acción y detiene el proceso de alta.
	\end{itemize}
\item[4.] El sistema no puede realizar la acción.
	\begin{itemize}
	\item[1.] El sistema cancela la acción y muestra un error al usuario.
	\end{itemize}
\end{enumerate}
\\

\hline
\textbf{Postcondiciones} & \\
\hline
\end{longtable}



\newpage
\subsubsection*{Caso de uso: Modificar categoría }
\begin{longtable}{| p{4cm} | p{10cm} |}
\endfirsthead
\multicolumn{2}{c}{\textit{Continúa de la página anterior}}\\[12pt]
\hline
\endhead
\hline
\multicolumn{2}{c}{\textit{Continúa en la siguiente página}} \\
\endfoot
\hline
\caption{Caso de Uso: Modificar categoría}\label{fig:1}\\
\endlastfoot


\hline
\multicolumn{2}{|c|}{\textbf{CU$<$39$>$ - Modificar Categoría}} \\

\hline
\textbf{Descripción} &
Modifica los datos de una categoría.\\

\hline
\textbf{Actores} &
Administrador\\

\hline
\textbf{Precondiciones} &
El usuario está autenticado en el sistema.\\

\hline
\textbf{Secuencia Normal} &\mbox{}\par\vspace{-\baselineskip}
\begin{enumerate}[leftmargin=0.7cm, topsep=0.1cm]
\item El usuario selecciona la opción de \textit{Modificar} sobre una categoría.
\item El sistema muestra un formulario con los datos actuales de la categoría seleccionada.
\item El usuario modifica los datos que desea y pulsa sobre el botón \textit{Confirmar}.
\item El sistema comprueba los datos y los actualiza.
\end{enumerate}


\\
\hline
\textbf{Excepciones} &\mbox{}\par\vspace{-\baselineskip}
\begin{enumerate}[leftmargin=0.9cm, topsep=0.1cm]
\item[3.] El usuario pulsa el botón de \textit{Cancelar}.
	\begin{itemize}
	\item[1.] El sistema cancela la acción y no actualiza la información de la categoría.
	\end{itemize}
\item[4.] El sistema no acepta los datos introducidos.
	\begin{itemize}
	\item[1.] El sistema cancela la acción y muestra un error al usuario.
	\end{itemize}
\end{enumerate}
\\

\hline
\textbf{Postcondiciones} & \\
\hline
\end{longtable}



\newpage
\subsubsection*{Caso de uso: Eliminar categoría }
\begin{longtable}{| p{4cm} | p{10cm} |}
\endfirsthead
\multicolumn{2}{c}{\textit{Continúa de la página anterior}}\\[12pt]
\hline
\endhead
\hline
\multicolumn{2}{c}{\textit{Continúa en la siguiente página}} \\
\endfoot
\hline
\caption{Caso de Uso: Eliminar categoría}\label{fig:1}\\
\endlastfoot


\hline
\multicolumn{2}{|c|}{\textbf{CU$<$40$>$ - Eliminar Categoría}} \\

\hline
\textbf{Descripción} &
Elimina una ruta del sistema.\\

\hline
\textbf{Actores} &
Administrador\\

\hline
\textbf{Precondiciones} &
El usuario está autenticado en el sistema.\\

\hline
\textbf{Secuencia Normal} &\mbox{}\par\vspace{-\baselineskip}
\begin{enumerate}[leftmargin=0.7cm, topsep=0.1cm]
\item El usuario selecciona la opción de \textit{Eliminar} sobre una ruta.
\item El sistema muestra una pantalla de confirmación.
\item El usuario pulsa sobre el botón de \textit{Confirmar}.
\item El sistema realiza la acción.
\end{enumerate}


\\
\hline
\textbf{Excepciones} &\mbox{}\par\vspace{-\baselineskip}
\begin{enumerate}[leftmargin=0.9cm, topsep=0.1cm]
\item[3.] El usuario pulsa el botón de \textit{Cancelar}.
	\begin{itemize}
	\item[1.] El sistema cancela la acción y detiene el proceso de eliminación.
	\end{itemize}
\end{enumerate}
\\

\hline
\textbf{Postcondiciones} & \\
\hline
\end{longtable}

\chapter[Dise\~no]{
  \label{chp:diseno}
  DISE\~NO
}
\thispagestyle{numberingStyle}
\pagestyle{numberingStyle}


\section{Arquitectura del sistema}
La arquitectura empleada en nuestro sistema será una arquitectura en capas, una de las técnicas de diseño más usadas en las ciencias de la computación. La arquitectura basada en capas es una especialización de la arquitectura cliente-servidor donde la carga de trabajo se divide en diferentes capas con un reparto claro de las funciones.

En la arquitectura basada en capas, una capa inferior proporciona un servicio a otra copa superior. El servicio ofrecido se define mediante un contrato de servicio. De esta forma, se consigue independizar el software de ambas capas y los cambios de implementación en una de ellas, no tiene repercusión sobre las demás.

Partiendo de que la aplicación será accesible desde dispositivo móvil y navegador web, se presentará una solución al diseño basada en dos alternativas de la arquitectura basada en capas: la arquitectura en 3 capas y la arquitectura en 4 capas.


\subsection{Arquitectura en 3 capas}
En este sistema de arquitectura, se diferencian tres capas, donde `capa' significa conjunto de máquinas que cumplen una función diferente.
\\
\\ 

\begin{figure}[!h]

\includegraphics[
   keepaspectratio=true
]{./05_Diseno/arq3capas.png}
\caption{Diagrama arquitectura en tres capas}
\end{figure}

\begin{itemize} [label={}]
	\item \textbf{Capa Servidor de Datos: } Es la capa encargada de gestionar el almacenamiento de los datos.
	\item \textbf{Capa Servidor Aplicación:} Formada por la capa de  servicios y el modelo. La capa de servicios, sirve de enlace entre la interfaz de usuario y el modelo mientras que el modelo, comúnmente, se subdivide en dos: la subcapa de acceso a datos y la subcapa de lógica de negocio.
	En la primera de ellas, se lleva acabo todas las acciones relacionadas con el acceso a los datos y es la que mantiene la comunicación con el servidor de datos.
	Por su parte, la subcapa de lógica de negocio, se encarga de llevar a cabo la implementación de las funcionalidades de la aplicación.
	\item \textbf{Capa Interfaz de Usuario: } Corresponde con la interfaz gráfica que se instala en las máquinas clientes y dispositivos finales. 
\end{itemize}



\subsection{Arquitectura en 4 capas}
En esta alternativa, se añade una capa intermedia entre el cliente y el modelo que actúe de servidor de aplicaciones y que proporciones la interfaz web para clientes que accedan desde navegador web.

\begin{figure}[!h]

\includegraphics[
   keepaspectratio=true
]{./05_Diseno/arq4capas.png}
\caption{Diagrama arquitectura en cuatro capas}
\end{figure}

Un navegador, para acceder a la aplicación, necesitará de un servidor de aplicaciones que le proporciones la interfaz web. Incorporar esta interfaz dentro del servidor de aplicaciones visto en la arquitectura anterior haría que este y el modelo estén fuertemente acoplados, impidiendo que puedan ser construidos con tecnologías diferentes.

Por ello, con esta arquitectura se pretende hacer ese desacople consiguiendo que múltiples aplicaciones pueden invocar al modelo, independientemente de que sean con interfaz gráfica o mediante navegador, sin necesidad de replicar el código del modelo en cada aplicación.

En consecuente, analizados los requisitos del sistema y conociendo las necesidades del mismo, se diseñará un sistema basado en una arquitectura de cuatro capas.


\subsection{Arquitectura completa del sistema}
A continuación, se presenta el diseño completo que se elaborará en la aplicación.

\begin{figure}[!h]

\includegraphics[
   keepaspectratio=true
]{./05_Diseno/arqsistema.png}
\caption{Diagrama arquitectura completa del sistema}
\end{figure}

Como se puede observar en el diagrama, el sistema sigue una arquitectura en cuatro capas.

Al tener la aplicación modelo desacoplada de la aplicación web, la capa de servicios debe servir de enlace entre la capa modelo y la interfaz de usuario. Ese enlace lo ofrece mediante unos servicios REST, que exponen a la capa superior, las funcionalidades implementadas en la capa modelo.

Por su parte, las aplicaciones cliente, tanto la interfaz de usuario del cliente móvil como la interfaz web del servidor de aplicaciones, siguen el patrón de arquitectura Modelo-Vista-Controlador (MVC). En ellas, el modelo no se encuentra implementado en la propia aplicación, sino que es accedido mediante un cliente REST que consume los servicios ofrecidos por la capa de servicios del modelo. De esta forma, ambas aplicaciones finales, invocan al modelo sin necesidad de tenerlo replicado.


\section{Capa Modelo}
Esta capa es la encargada de implementar la lógica de negocio de la aplicación, lo que implica el manejo de las entidades persistentes y el acceso a los datos. Como podemos observar en la Figura 6.4, y debido a la arquitectura establecida, el modelo estará compuesto por una subcapa de acceso a datos, otra de lógica de negocio y una última de servicios REST, la que permitirá acceder remotamente a las funcionalidades de la aplicación independientemente del tipo de  cliente final.

\begin{figure}[H]
\includegraphics[
   keepaspectratio=true
]{./05_Diseno/disenomodelo.png}
\caption{Diagrama diseño modelo}
\end{figure}


\newpage
\subsection{Diagrama clases persistentes}
\begin{figure}[H]
\includegraphics[
   keepaspectratio=true
]{./05_Diseno/disenoclases.png}
\caption{Diagrama diseño modelo}
\end{figure}

En el diagrama se muestran las clases persistentes que manejará la aplicación. A continuación, se detallará brevemente, el significado y funcionalidad de cada una:

\begin{itemize}
	\item \textbf{User: } Es la clase encargada de gestionar la información de los usuarios registrados en la aplicación.
	\item \textbf{Role: } Enumerado con los roles disponibles para un usuario: \textit{ADMIN, MODERATOR y USER}.
	\item \textbf{Route: } Clase encargada de guardar la información sobre las rutas creadas por los usuarios. Las rutas están compuestas por \textit{RouteDays}.
	\item \textbf{RouteDay: } Clase con una relación fuerte de composición con la clase \textit{Route}. Su tiempo de vida está condicionada por la vida de la clase que la incluye. Mantiene la información para cada uno de los días que componen la duración de una ruta.
	\item \textbf{RouteState: } Enumerado con los diferentes estados por los que pasa una ruta en el tiempo: \textit{PENDING, IN\_PROGRESS y COMPLETED}.
	\item \textbf{Stay: } Entidad con la funcionalidad de gestionar las visitas que decida hacer un usuario en un día de una ruta determinada. La visita, puede ser a lugares obtenidos de una fuente externa (\textit{Foursquare}) o a eventos gestionados por la propia aplicación.
	\item \textbf{Place: } Entidad que registra y almacena los detalles sobre los lugares extraídos de la fuente externa.
	\item \textbf{Event: } Es la clase encargada de gestionar los eventos dados de alto en el sistema. Los eventos están compuestos por \textit{EventDays}.
	\item \textbf{EventDay: } Clase con una relación fuerte de composición con la clase \textit{Event}. Su tiempo de vida está condicionada por la vida de la clase que la incluye. Mantiene la información para cada uno de los días que componen al evento.
	\item \textbf{EventPlace: } Es la clase donde se maneja toda la información sobre las distintas ubicaciones y actividades que incluye un día determinado del evento.
	\item \textbf{Category: } Clase que almacena la información relevante a las categorías sobre las que se filtran los lugares obtenidos de \textit{Foursquare}.
	\item \textbf{SubCategory: } Establece una jerarquía con la clase anterior. Almacena las categorías que son un subtipo de una categoría determinada.
\end{itemize}

\subsection{Diseño módulo acceso a datos}
En esta capa se hará uso del patrón de diseño Data-Access-Object (DAO).

\subsubsection*{Patrón de diseño DAO}
Este patrón de diseño intenta desacoplar el acceso a los datos de su almacenamiento subyacente. Los datos persistentes, actualmente, dependen en gran medida del tipo de base de datos utilizada: base de datos relacional, base de datos orientada a objetos, archivos planos... siendo las bases de datos relacionales las más utilizadas. Lamentablemente, se acceden a estes tipos de bases de datos de maneras muy diferentes y sería preferible elegir el tipo de base de datos utilizada en la fase de implementación en lugar de en la fase de diseño.

Utilizar este patrón en lugar de acceder directamente a la fuente de datos nos permite pasar de un tipo de fuente de datos a otro diferente sin tener que realizar modificaciones en la lógica.


\begin{figure}[H]
\includegraphics[
   keepaspectratio=true
]{./05_Diseno/patrondao.jpg}
\caption{Diagrama diseño modelo}
\end{figure}

En la Figura 6.6 se muestra un pequeño diagrama con los elementos participantes en este patrón de diseño.

\begin{itemize}
	\item \textbf{Business Object: } Representa la clase con la lógica de negocio. Es la responsable de saber qué y cómo modificar el contenido de los datos pero no como almacenarlo.
	\item \textbf{Data Access Object: } Se encarga de ocultar la fuente de datos real de manera que el objeto con la lógica de negocio (Business Object) se comunica con este en vez de hacerlo directamente con el objeto de acceso a los datos.
	\item \textbf{DataSource: } Es la fuente real de datos y el que realiza la conexión con la base de datos. En la mayoría de casos, existe una base de datos relacional a la que se accede a través de SQL.
	\item \textbf{Transfer Object: } Es el objeto que se utiliza para transferir el contenido de los datos reales del \textit{Data Access Object} al objeto de negocio \textit{Business Object}. Representa los datos almacenados en la base de datos.
\end{itemize}

\subsection{Diseño módulo lógica de negocio}
\subsection{Diseño módulo servicios REST}
\section{Controlador}
\subsection{Aplicación web}
\subsection{Aplicación móvil}

\section{Diseño físco de los datos}
\subsection{Modelo Entidad-Relación}
\subsection{Modelo Relacional}








\chapter[Implementación]{
  \label{chp:implementacion}
  IMPLEMENTACIÓN
}
\thispagestyle{numberingStyle}
\pagestyle{numberingStyle}

\chapter[Pruebas]{
  \label{chp:pruebas}
  PRUEBAS
}
\thispagestyle{numberingStyle}
\pagestyle{numberingStyle}

\chapter[Conclusiones y líneas futuras]{
  \label{chp:conclusiones}
  CONCLUSIONES Y LÍNEAS FUTURAS
}
\thispagestyle{numberingStyle}
\pagestyle{numberingStyle}

\section{Conclusiones}

El objetivo primordial del desarrollo de este proyecto era la elaboración de una aplicación para la gestión y planificación de rutas turísticas y, el resultado obtenido, cumple con todos los requisitos establecidos en el anteproyecto.

A lo largo del desarrollo se fueron interponiendo por el camino muchas dificultades. El no uso previo de algunas de las tecnologías empleadas  y las dificultadas surgidas con el uso de las APIs externas, han supuesto los mayores problemas durante el desarrollo. Esto supuso realizar una gran inversión de tiempo en conocimiento y comprendimiento de las nuevas tecnologías a utilizar, como por ejemplo, con el framework Ionic, herramienta utilizada por primera vez. Con respecto a las APIs, en concreto la API de Foursquare, se ofrecía una librería Java obsoleta. Ha sido necesario modificar y compilar de nuevo dicha librería para poder emplearla más fácilmente en el proyecto. Con la otra API externa utilizada, la API de Google, surgieron problemas de configuración tratando de emplear dicha API como elemento nativo en la aplicación móvil.

Por otra parte, partiendo de que la aplicación será consumida por usuarios finales con, probablemente, escaso nivel informático, se ha conseguido elaborar una aplicación móvil atractiva, clara y de fácil uso; uno de los requisitos más valorados por los usuarios finales.

Personalmente, uno de los objetivos a la hora de realizar este proyecto era conocer y poder trabajar con una de las herramientas más utilizadas hoy en día, el framework Ionic. Como consecuencia, se valora el conocimiento adquirido con esta herramienta y que sirve de punto de partida en el mundo del desarrollo de aplicaciones móviles.

Finalmente, la realización de este proyecto, plasma muchos de los conocimientos teóricos adquiridos en los últimos años. Gracias a estos conocimientos, se ha podido desarrollar un software de calidad, fácilmente escalable, que supone una aplicación base estable para el continuo proceso de desarrollo y mejora.


\section{Líneas futuras}
La aplicación creada supone una base inicial para seguir trabajando y mejorando su funcionalidad. Como trabajo futuro, se plantean dos metas a seguir, unas a corto plazo que permiten una mejora actual y continua del sistema, y otras a largo plazo; prestando más ambición en el futuro de la aplicación.

A corto plazo se consideran las siguientes metas:

\begin{itemize}

	\item \textbf{Mejorar interfaz}. El usuario final será muy crítico con la interfaz de la aplicación. Será necesario prestar atención al feedback de los usuarios para poder mejorar aquellos aspectos más críticos. Otro tema importante, es mejorar la visualización de las rutas en los mapas. Para ello, sería conveniente profundizar en la API de Google para mostrar mapas y hacerlos más atractivos para los usuarios.
	
	\item \textbf{Obtención de los datos de geolocalización}. Actualmente, estos datos se obtiene en segundo plano cuando el usuario solicita registrarlos para una ruta concreta. A mayores, estos datos se transmiten al sistema conforme se van obteniendo, de manera que si no se dispone de una conexión de internet en determinado momento, no será posible registrar dicha información. 
	La idea futura es solucionar estos inconvenientes pretendiendo obtener dichos datos de forma automática, sin necesidad de que el usuario confirme esta acción, únicamente tendrá la opción de escoger, previamente, si deseará registrar estos datos o no. De esta forma, cuando la ruta planificada se encuentre en el día concreto, el sistema activará automáticamente la geolocalización en segundo plano, y a mayores, almacenará los datos internamente y los remitirá al servidor cuando exista conexión de red, permitiendo evitar la perdida de esa información.

\end{itemize}

Con respecto al trabajo a más a largo, se consideran los siguientes objetivos.

\begin{itemize}
	\item \textbf{Avance hacia red social}. La idea reside en ampliar esta aplicación hasta convertirla en una pequeña red social. Con lo realizado, la interacción social de la aplicación solo permite la consulta de rutas de otros usuarios. La idea consistiría en poder incorporar ciertas funcionalidades como copiar rutas de otros usuarios a tus rutas, permitir comentarios sobre las rutas, o poder crear grupos de usuarios que realicen una misma ruta, permitiendo hacer una aplicación más social.
	
	\item \textbf{Gestión de eventos}. Actualmente, los eventos de la aplicación son gestionados por usuarios con permisos específicos. El objetivo sería eliminar este tipo de usuario encargado de gestionar los eventos y hacer que esta tarea resida en los usuarios finales de la aplicación. Para ello, todos podrían crear eventos o modificarlos, pero sería necesario incorporar un sistema de validación entre usuarios, evitando por ejemplo, que se den de alta eventos no reales. Otra solución, que podría mejorar la gestión de eventos, sería obtenerlos de fuentes externas como podría ser Facebook, lo que permitiría aprovechar también las funcionalidades sociales de esta aplicación. Ambas propuestas no son excluyentes, y podrían coexistir las dos en la aplicación.

	\item \textbf{Incorporar funcionalidades}. El objetivo sería, partiendo de las funcionalidades ya implementadas, incorporar más posibilidades de personalización en las rutas. Esto consistiría en poder crear rutas que se ubiquen en lugares o ciudades diferentes; estableciendo métodos de viaje entre estos lugares, ya sea indicando tren, avión o método de transporte que se use y permitir hacer pagos o reservas en lugares o eventos que se visiten (museos, conciertos...). También se plantea aumentar la fuente de datos, intentado recabar información de lugares o ubicaciones no contempladas en Foursquare, o que simplemente enriquezcan la información obtenida de este fuente externa, permitiendo caracterizar las diferentes rutas, como por ejemplo, en turísticas, de aventura o de ocio.

\end{itemize}














\begin{thebibliography}{9}

\thispagestyle{numberingStyle}
\pagestyle{numberingStyle}

\bibitem{java} 
"java.com: Java + You", \textit{Java.com}, 2018. [Online]. Available: https://www.java.com/en/. [Accessed: 05- Jun- 2018].

\bibitem{ajax} 
M. Firtman, \textit{Ajax}, 2nd ed. Barcelona: Marcombo, 2011.

\bibitem{ionic} 
A. Ravulavaru and M. Hartington, \textit{Learning Ionic : Build real-time and hybrid mobile applications with Ionic}. Birmingham, England: Packt Publishing, 2015.

\bibitem{jaxrs} 
B. Burke, \textit{RESTful Java with JAX-RS 2.0}, 2nd ed. Sebastopol: O'Reilly, 2013.

\bibitem{rup} 
I. Jacobson, G. Booch and J. Rumbaugh, \textit{El proceso unificado de desarrollo de software}. Madrid: Addison Wesley, 2000.

\bibitem{htmlcssjs} 
J. Meloni, \textit{HTML5, CSS3 y JavaScript}, 2nd ed. Madrid: Anaya Multimedia, 2015.

\bibitem{patterns} 
E. Gamma, \textit{Design patterns}. Reading, Mass.: Addison-Wesley, 1995.

\bibitem{jpa}
M. Keith and M. Schnicariol, \textit{Pro JPA 2}. New York: Apress, 2009.

\end{thebibliography}



\appendix
\chapter[Diagramas]{
  \label{chp:diagramas}
  Diagramas
}
\thispagestyle{numberingStyle}
\pagestyle{numberingStyle}


\subsection{Diagrama Entidad Relación}
\begin{sidewaysfigure}[]
\includegraphics[
   keepaspectratio=true
]{./11_Apendice/Apendice_A/img/ER.png}
\caption{Diagrama ER con atributos}
\end{sidewaysfigure}



\newpage
\subsection{Diagramas DAOs}

\subsubsection*{DAO entidad - Category}
\begin{figure}[H]
\centering
\includegraphics[
   keepaspectratio=true
]{./11_Apendice/Apendice_A/img/categorydao.png}
\caption{Diagrama DAO entidad \textit{Category}}
\end{figure}


\subsubsection*{DAO entidad - SubCategory}
\begin{figure}[H]
\centering
\includegraphics[
   keepaspectratio=true
]{./11_Apendice/Apendice_A/img/subcategorydao.png}
\caption{Diagrama DAO entidad \textit{SubCategory}}
\end{figure}


\subsubsection*{DAO entidad - User}
\begin{figure}[H]
\centering
\includegraphics[
   keepaspectratio=true
]{./11_Apendice/Apendice_A/img/userdao.png}
\caption{Diagrama DAO entidad \textit{User}}
\end{figure}


\subsubsection*{DAO entidad - Route}
\begin{figure}[H]
\centering
\includegraphics[
   keepaspectratio=true
]{./11_Apendice/Apendice_A/img/routedao.png}
\caption{Diagrama DAO entidad \textit{Route}}
\end{figure}


\subsubsection*{DAO entidad - RouteDay}
\begin{figure}[H]
\centering
\includegraphics[
   keepaspectratio=true
]{./11_Apendice/Apendice_A/img/routedaydao.png}
\caption{Diagrama DAO entidad \textit{RouteDay}}
\end{figure}



\subsubsection*{DAO entidad - Stay}
\begin{figure}[H]
\centering
\includegraphics[
   keepaspectratio=true
]{./11_Apendice/Apendice_A/img/staydao.png}
\caption{Diagrama DAO entidad \textit{Stay}}
\end{figure}



\subsubsection*{DAO entidad - Event}
\begin{figure}[H]
\centering
\includegraphics[
   keepaspectratio=true
]{./11_Apendice/Apendice_A/img/eventdao.png}
\caption{Diagrama DAO entidad \textit{Event}}
\end{figure}



\subsubsection*{DAO entidad - EventDay}
\begin{figure}[H]
\centering
\includegraphics[
   keepaspectratio=true
]{./11_Apendice/Apendice_A/img/eventdaydao.png}
\caption{Diagrama DAO entidad \textit{EventDay}}
\end{figure}



\subsubsection*{DAO entidad - EventPlace}
\begin{figure}[H]
\centering
\includegraphics[
   keepaspectratio=true
]{./11_Apendice/Apendice_A/img/eventplacedao.png}
\caption{Diagrama DAO entidad \textit{EventPlace}}
\end{figure}



\subsubsection*{DAO entidad - Place}
\begin{figure}[H]
\centering
\includegraphics[
   keepaspectratio=true
]{./11_Apendice/Apendice_A/img/placedao.png}
\caption{Diagrama DAO entidad \textit{Place}}
\end{figure}

\chapter[Manual de usuario]{
  \label{chp:manualdeusuario}
  MANUAL DE USUARIO
}
\thispagestyle{numberingStyle}
\pagestyle{numberingStyle}


\section{Manual de usario aplicación móvil}

\subsection*{Acceso a la aplicación}
\begin{figure}[H]
\centering
\includegraphics[
   keepaspectratio=true
]{./11_Apendice/Apendice_B/img/Ionic-1-Login.png}
\caption{Pantalla acceso - Aplicación móvil.}
\end{figure}

Cuando se accede a la aplicación por primera vez se mostrará la pantalla que aparece en la parte izquierda de la figura anterior. El usuario, si ya tiene una cuenta registrada, deberá ingresar los datos de acceso y darle al botón de \textit{Entrar} para acceder a la aplicación. Si el usuario no se encuentra registrado deberá registrarse haciendo click sobre el botón \textit{Registrarse}. Al registrarse, se mostrará el formulario que aparece a la derecha en la figura, solicitando los datos de acceso necesarios. El sistema validará los datos de entrada y procederá a la autenticación del usuario.


\subsection*{Pantalla principal}

La pantalla principal de la aplicación estará formada por tres pestañas, que son las siguientes:


\begin{figure}[H]
\centering
\includegraphics[
   keepaspectratio=true
]{./11_Apendice/Apendice_B/img/Ionic-2-Tabs.png}
\caption{Pantalla acceso - Aplicación móvil.}
\end{figure}

De izquierda a derecha, y mediante el selector que aparece en la parte inferior, se puede alternar entre las pestañas de \textit{Explorar Rutas}, \textit{Crear Ruta} y \textit{Mis Datos}.


\newpage
\subsubsection*{Explorar Rutas}

La pestaña de explorar rutas permitirá poder obtener las rutas de los demás usuarios.

\begin{figure}[H]
\centering
\includegraphics[
   keepaspectratio=true
]{./11_Apendice/Apendice_B/img/Ionic-3-ExploreRoutes.png}
\caption{Pantalla acceso - Aplicación móvil.}
\end{figure}

En el cuerpo de la pestaña aparece el listado con las rutas de los usuarios. Para cada una de ellas, se muestra una imagen de fondo del sitio, el nombre de la ciudad y las fechas. En la parte superior derecha de la pestaña existe la opción de aplicar un filtro sobre las rutas. Al hacer click sobre dicho botón se muestra el formulario de la imagen en el que se podrá indicar ciudad, estado, número de días, distancia máxima o duración máxima para filtrar.

Los filtros son acumulativos y se pueden ver los filtros activos en la parte superior de la pantalla.


\subsubsection*{Crear Ruta}
\begin{figure}[H]
\centering
\includegraphics[
   keepaspectratio=true
]{./11_Apendice/Apendice_B/img/Ionic-4-AddRoute.png}
\caption{Pantalla acceso - Aplicación móvil.}
\end{figure}


La pantalla para crear rutas ofrecerá, en la parte superior, un buscador de ciudades. Al buscar una ciudad, el sistema ayudará autocompletando con las ciudades disponibles. Al seleccionar una de ellas, se mostrará un mapa, indicando la ubicación geográfica de dicha ciudad.

Haciendo click en la flecha de la parte inferior de la pantalla, se da de alta ruta en el sistema, y si no se produce ningún error, se redirige al usuario a la vista encargada de mostrar la información detallada de la ruta.

\newpage
\subsubsection*{Mis Datos}
En esta pestaña, se muestran los datos del usuario conectado a la aplicación. En la parte superior de la pantalla aparece un botón de opciones que permite modificar los datos del usuario y desconectarse de la aplicación. Ambas acciones, mostrarán respectivamente los formularios, situados a la derecha en la figura siguiente.

\begin{figure}[H]
\centering
\includegraphics[
   keepaspectratio=true
]{./11_Apendice/Apendice_B/img/Ionic-5-MyData.png}
\caption{Pantalla acceso - Aplicación móvil.}
\end{figure}


Para modificar los datos de usuario será necesario indicar correo electrónico y contraseña a modificar. Si el usuario confirma los datos, el sistema realizará las validaciones y actualizará los datos del usuario en el sistema.

En la parte central de la pantalla aparece la información sobre las rutas creadas por dicho usuario. En el listado, se puede hacer uso del selector que permite clasificar dichas rutas en función de su estado (Pendientes, en curso o completadas). Con un click sobre las rutas, se podrá acceder y consultar detalladamente la ruta seleccionada.


\newpage
\subsection*{Panel de viaje}
Una vez creada una ruta o cuando se consulte una, se mostrará el siguiente panel que permitirá personalizar las rutas.


\begin{figure}[H]
\centering
\includegraphics[
   keepaspectratio=true
]{./11_Apendice/Apendice_B/img/Ionic-6-RoutePanel.png}
\caption{Pantalla acceso - Aplicación móvil.}
\end{figure}

Este panel se compone de:

\begin{itemize}
	\item \textbf{Itinerario. }Deshabilitado mientras la ruta no tenga unas fechas de viaje establecidas. Cuando se asigne las fechas, esta opción permitirá consultar la distribución por días de la ruta.
	
	\item \textbf{Fechas. }Acción para asignar una rango de fechas a la ruta.
	
	\item \textbf{Mapa. }Muestra el itinerario de la ruta haciendo uso de mapas.
	
	\item \textbf{Eventos. }Permite consultar los eventos disponibles en la ciudad donde se va realizar el viaje. 
	
	\item \textbf{Track. }Activa la geolocalización, que permitirá guardar la información en tiempo real de la ruta.
	
	\item \textbf{Info. }Muestra los detalles de la ruta.
\end{itemize}

A continuación, se explicarán cada uno de los elementos mencionados.


\subsubsection*{Fechas}

Lo primero a realizar es asignar unas fechas a nuestro viaje.

\begin{figure}[H]
\centering
\includegraphics[
   keepaspectratio=true
]{./11_Apendice/Apendice_B/img/Ionic-7-Dates.png}
\caption{Pantalla acceso - Aplicación móvil.}
\end{figure}

Al hacer click sobre la opción \textit{Fechas} del panel se abre una ventana como si fuese un calendario. En esta ventana, se escoge el rango de fechas en las que se va realizar el viaje. Una vez asignadas las fechas, en el panel, la opción \textit{Itinerario} ya estará disponible.


\subsection*{Itinerario}
La página del itinerario será la encargada de mostrar las visitas asignadas a los días de la ruta. En el siguiente ejemplo se mostrarán dos días de la ruta con los correspondientes visitas asignadas.

\begin{figure}[H]
\centering
\includegraphics[
   keepaspectratio=true
]{./11_Apendice/Apendice_B/img/Ionic-8-Itinerario.png}
\caption{Pantalla acceso - Aplicación móvil.}
\end{figure}

El selector de la parte superior permite obtener las visitas asignadas a los diferentes días de la ruta. En el cuerpo de la página, aparece el listado con las visitas y la información  relevante en cada una de ellas.

\begin{itemize}
	\item \textbf{1. }Hora de llegada al lugar, en la primera visita corresponde con la hora de comienzo del día de la ruta, valor modificable por el usuario. 	
	
	\item \textbf{2. }Datos relevantes del lugar en concreto. Al final del bloque, incluye un apartado denominado \textit{Parada}, donde se especificará el tiempo a que se pasará en dicho lugar. 
	
	\item \textbf{3. }Indica la hora de salida del sitio, calculada en función del tiempo que desea pasar el usuario en el sitio.
	
	\item \textbf{4. }Acción que permite obtener la distancia y duración que hay entre dos visitas.
	
	\item \textbf{5. }Acción para eliminar una visita concreta del itinerario.
\end{itemize}

\subsubsection*{Calcular duración y distancia}

Al hacer click en el botón para obtener la distancia entre dos visitas obtenemos el siguiente resultado.

\begin{figure}[H]
\centering
\includegraphics[
   keepaspectratio=true
]{./11_Apendice/Apendice_B/img/Ionic-9.png}
\caption{Pantalla acceso - Aplicación móvil.}
\end{figure}

Como se puede observar en la imagen, ahora aparecen calculados los tiempos de desplazamiento y la duración entre las dos visitas. Haciendo ahora click sobre esa información obtenida, el sistema alterna los métodos de transporte disponibles, junto con la información obtenida para cada uno. En la figura se muestran los tres métodos de transporte que se utilizan, junto con la información de cada uno de ellos.


\subsubsection*{Modificar tiempo parada}
El tiempo a parar en cada visita se puede modificar haciendo click sobre el icono en forma de lápiz que aparece dentro de la información de cada visita.

\begin{figure}[H]
\centering
\includegraphics[
   keepaspectratio=true
]{./11_Apendice/Apendice_B/img/Ionic-10.png}
\caption{Pantalla acceso - Aplicación móvil.}
\end{figure}

En la ventana emergente se podrá seleccionar el tiempo a pasar en la visita seleccionada, indicando las horas  en la columna de la izquierda y los minutos en la de la derecha. 

Una vez actualizada dicha información, se mostrará el tiempo indicado en la información de la visita y se recalcularán los tiempos de, salida y llegada, en las visitas posteriores.


\subsubsection*{Opciones itinerario}
En la parte superior aparece un icono con tres puntos verticales que nos permitirán acceder a las opciones del día concreto de la ruta.

\begin{figure}[H]
\centering
\includegraphics[
   keepaspectratio=true
]{./11_Apendice/Apendice_B/img/Ionic-11.png}
\caption{Pantalla acceso - Aplicación móvil.}
\end{figure}


\begin{itemize}
	\item \textbf{Habilitar Edición. }Permitirá editar el orden de las visitas en el día concreto.
	
	\begin{figure}[H]
\centering
\includegraphics[
   keepaspectratio=true
]{./11_Apendice/Apendice_B/img/Ionic-12.png}
\caption{Pantalla acceso - Aplicación móvil.}
\end{figure}

	Al seleccionar la opción de edición, aparecerá al lado de cada visita un selector que permitirá coger y arrastrar cada visita a la posición deseada.	
	
	\item \textbf{Modificar Hora Salida. } Esta opción permitirá modificar la hora de comienzo de la primera visita del día.
	
		\begin{figure}[H]
\centering
\includegraphics[
   keepaspectratio=true
]{./11_Apendice/Apendice_B/img/Ionic-13.png}
\caption{Pantalla acceso - Aplicación móvil.}
\end{figure}

	Al igual que ocurría con los tiempos de parada en las visitas, para asignar la hora de comienzo sucede algo similar, permitiendo escoger la hora deseada.
	
	\item \textbf{Ver en Mapa. }Permitirá consultar el día de la ruta en el mapa.
	
	
	
\begin{figure}[H]
\centering
\includegraphics[
   keepaspectratio=true
]{./11_Apendice/Apendice_B/img/Ionic-14.png}
\caption{Pantalla acceso - Aplicación móvil.}
\end{figure}
\end{itemize}

El mapa podrá ser consultado directamente desde el itinerario como desde el panel principal de la ruta. Si se hace desde la pantalla \textit{Itinerario}, se mostrará el día seleccionado como primera opción. Si se consulta desde el panel de la ruta, se mostrará de inicio el mapa del primer día de la ruta. Dentro de la pantalla, se podrá alternar los días mediante el selector superior, al igual que se hacía en la pantalla de \textit{Itinerario}.
	
	En el mapa aparecerán marcados las visitas a realizar en el día concreto. Si se hace click sobre dichas marcas, se abrirá una ventana con información más detallada de la visita, indicando su orden, nombre, dirección y tiempo de parada.
	
	En la parte inferior, se encuentra una lista con todos las visitas para el día determinado. Mediante desplazamiento horizontal, se podrán alternar entre las diferentes visitas. Al cambiar a otra visita, automáticamente se mostrará su ventana de información en el mapa.
	
	
En la parte superior derecha de la imagen anterior, aparece un botón seleccionable que permite incorporar al mapa los datos reales de la ruta, si es que los hubiese.

** Simular datos reales **





\newpage
\subsection*{Eventos}
En la pantalla \textit{Panel de viaje} se podrán consultar los eventos disponibles.

\begin{figure}[H]
\centering
\includegraphics[
   keepaspectratio=true
]{./11_Apendice/Apendice_B/img/Ionic-15.png}
\caption{Pantalla acceso - Aplicación móvil.}
\end{figure}

Haciendo click sobre \textit{Eventos}, abriremos la pantalla de eventos donde, mediante el selector, podremos obtener los eventos coincidentes en nuestro viaje así como los eventos futuros que sucedan en la misma ciudad. 

\begin{figure}[H]
\centering
\includegraphics[
   keepaspectratio=true
]{./11_Apendice/Apendice_B/img/Ionic-16.png}
\caption{Pantalla acceso - Aplicación móvil.}
\end{figure}


Al desplegar un evento en concreto de los mostrados, se muestran las diferentes actividades, localizaciones y demás que componen dicho evento. Seleccionando el botón con el signo `+' indicado en la figura, se añadirá dicha actividad al día correspondiente de la ruta. Una vez añadida la actividad, aparecerá con un símbolo como una `V', indicando que esa actividad ya está asignada. Volviendo hacer click sobre ella podremos eliminarla de la ruta.

La opción \textit{Ver Ubicación} permitirá mostrar la ubicación de la actividad en el mapa y compararla con la ruta elaborada. 

\begin{figure}[H]
\centering
\includegraphics[
   keepaspectratio=true
]{./11_Apendice/Apendice_B/img/Ionic-17.png}
\caption{Pantalla acceso - Aplicación móvil.}
\end{figure}

La imagen de la izquierda de la figura anterior muestra la pantalla donde se puede consultar el evento en el mapa mientras que la de la derecha muestra como se representaría un evento en el itinerario de un día de la ruta. 


\newpage
\subsection*{Activar geolocalización}
En el panel de viaje, podremos activar la geolocalización cuando la ruta se encuentre en algunos de los días para los que está planificado. Haciendo click sobre el botón que pone \textit{Track}, el sistema activará automáticamente la geolocalización en segundo plano. Se sabrá que está activa por el cambio de color en el icono y, simplemente haciendo click de nuevo sobre el icono, se puede desactivar.


\begin{figure}[H]
\centering
\includegraphics[
   keepaspectratio=true
]{./11_Apendice/Apendice_B/img/Ionic-18.png}
\caption{Pantalla acceso - Aplicación móvil.}
\end{figure}

Será necesario otorgar permisos de acceso al GPS para su correcto funcionamiento.

\newpage
\subsection*{Pantalla información}
Desde el panel principal se podrá acceder a la información relevante de la ruta a través de la opción \textit{Info}.


\begin{figure}[H]
\centering
\includegraphics[
   keepaspectratio=true
]{./11_Apendice/Apendice_B/img/Ionic-19.png}
\caption{Pantalla acceso - Aplicación móvil.}
\end{figure}

En esta pantalla se muestra toda la información relacionada con la ruta y, haciendo uso de la opción \textit{Privada}, se puede alternar la privacidad establecida para la ruta.


\newpage
\subsection*{Añadir lugares}
Para añadir lugares a visitar en una ruta, será necesario hacerlo desde la pantalla de \textit{Itinerario}. ****





\newpage
\section{Manual de usario aplicación web}

\subsection{Acceso a la aplicación}

Tanto la aplicación web de usuario como la aplicación web de administración tendrán el mismo punto de acceso, que será el siguiente:

\begin{figure}[H]
\centering
\includegraphics[
   keepaspectratio=true
]{./11_Apendice/Apendice_B/img/WebLogin.png}
\caption{Pantalla acceso - Aplicación web.}
\end{figure}

Se mostrará un pequeño formulario en el que será necesario indicar usuario y contraseña para poder acceder a la aplicación. En función del rol del usuario, se accederá al panel de administración o a la propia aplicación de usuario. 

\subsection{Aplicación de usuario}

\subsubsection*{Pantalla principal}
Si se accede a la aplicación de usuario se mostrará la siguiente pantalla:

\begin{figure}[H]
\centering
\includegraphics[
   keepaspectratio=true
]{./11_Apendice/Apendice_B/img/WebIndex.png}
\caption{Pantalla principal - Aplicación usuario}
\end{figure}

La barra de navegación será fija para todas las pantallas de la aplicación de usuario y estará formada por:

\begin{itemize}
	\item \textbf{1 - \textit{Inicio}}. Lleva al usuario a esta página.
	\item \textbf{2 - \textit{Explorar}}. Dirige al usuario a la pantalla donde podrá ezplorar las rutas, de los demás usuarios, existentes en la aplicación.
	\item \textbf{3 - \textit{Mis Rutas}}. Dirige al usuario a la pantalla donde podrá consultar las rutas creadas por él mismo.
	\item \textbf{4 - \textit{Desconectarse}}. Permite al usuario desloguearse de la aplicación. Al lado, aparece el nombre de usuario que está actualmente conectado.
\end{itemize}
	
	
En el cuerpo de la página aparecen detalladas las funcionalidades que puede realizar el usuario. En este caso, esas funcionalidades son las mismas que se encuentran en los puntos 2 y 3 de la barra de navegación, \textit{Explorar} y \textit{Mis Rutas}, respectivamente.


\subsubsection*{Explorar rutas}
\begin{figure}[H]
\centering
\includegraphics[
   keepaspectratio=true
]{./11_Apendice/Apendice_B/img/WebExploreRoutes.png}
\caption{Pantalla explorar rutas - Aplicación usuario}
\end{figure}

La pantalla de explorar rutas permitirá al usuario obtener las rutas públicas creadas por los demás usuarios de la aplicación. Indicará un listado de las rutas con una pequeña información sobre cada una ellas. Cada ruta incluirá la opción de consultar, que permitirá acceder a la vista de detalles. 

Esta pantalla sigue, visualmente, un estilo similar a la pantalla de \textit{Mis Rutas}, que veremos a continuación más detalladamente. 


\subsubsection*{Mis rutas}

Dentro de la página \textit{Mis Rutas}, el usuario podrá obtener las rutas creadas por él, clasificadas en función de su estado.

\begin{figure}[H]
\centering
\includegraphics[
   keepaspectratio=true
]{./11_Apendice/Apendice_B/img/WebMyRoutes.png}
\caption{Pantalla mis rutas - Aplicación usuario}
\end{figure}

\begin{itemize}
	\item \textbf{Zona 1.} Selector que permite alternar entre las rutas, clasificadas por los diferentes estados en los que se encuentran.
	\item \textbf{Zona 2.} Bloque que representa la información básica de la ruta. Incluye foto, nombre de la ciudad, fechas, número de días, número de visitas asignadas y distancia y duración totales.
	\item \textbf{Zona 3.} Acciones a realizar sobre determinada ruta.
	\begin{itemize}
		\item Si el usuario selecciona \textit{Eliminar}, se mostrará una alerta, indicando al usuario si desea confirmar o no la acción solicitada.
		\begin{figure}[H]
			\centering
			\includegraphics[
			   keepaspectratio=true
			]{./11_Apendice/Apendice_B/img/WebMyRoutesDelete.png}
			\caption{Pantalla eliminar ruta - Aplicación usuario}
		\end{figure}


			
		\item Si el usuario selecciona \textit{Consultar}, el sistema lo dirigirá a la pantalla de \textit{Detalles}, donde podrá obtener la información, detallada por días, de la ruta seleccionada.
	\end{itemize}			
\end{itemize}


\subsubsection*{Detalles ruta}
\begin{figure}[H]
\centering
\includegraphics[
	keepaspectratio=true
]{./11_Apendice/Apendice_B/img/WebDetailRoute.png}
\caption{Pantalla detalles ruta - Aplicación usuario}
\end{figure}

En la página de detalles de la ruta tenemos tres zonas diferenciadas:

\begin{itemize}
	\item \textbf{Zona 1. }Selector que permite navegar por los días de la ruta.
	\item \textbf{Zona 2. }Listado, por orden,con los lugares a visitar en determinado día. Cada visita incluye el tiempo de llegada y una pequeña descripción con el nombre del lugar o evento, dirección y tiempo de parada. Se hace distinción por colores, en azul se muestran las visitas a eventos, en rojo las visitas a lugares y en amarilla se indica la información de distancia y tiempo entre cada lugar.
	\item \textbf{Zona 3. }Mapa en el que se muestran las visitas del listado anterior. Haciendo click en cada marca, se abre una ventana de información, en la que se indica el orden de que ocupa dicha visita en la ruta, el nombre y la dirección. Al tratarse de un mapa de Google, se puede interactuar con las funcionalidades que este ofrece, como son la vista en satélite o en mapa, hacer uso del StreetView, entre otras.
\end{itemize}


\subsection{Panel de administración}

\subsubsection*{Barra de navegación}

\begin{figure}[H]
\centering
\includegraphics[
   keepaspectratio=true
]{./11_Apendice/Apendice_B/img/WebPanelNavBar.png}
\caption{Barra de navegación - Panel de administración}
\end{figure}

La barra de navegación está formada por:

\begin{itemize}
	\item \textbf{1 - Página principal} Dirige al usuario a la página principal.
	\item \textbf{2 - Información de usuario}. Se mostrará el nombre del usuario conectado junto al rol que desempeña. A la derecha de todo se incluye la opción para desconectarse de la aplicación.
\end{itemize}


\subsubsection*{Pantalla principal}
Si el usuario que accede a la aplicación, tiene rol de administrador o de gestor de eventos, se mostrarán respectivamente las siguientes pantallas.

\begin{figure}[H]
\centering
\includegraphics[
   keepaspectratio=true
]{./11_Apendice/Apendice_B/img/WebPanelIndexAdmin.png}
\caption{Pantalla principal administrador - Panel de administración}
\end{figure}


\begin{figure}[H]
\centering
\includegraphics[
   keepaspectratio=true
]{./11_Apendice/Apendice_B/img/WebPanelIndexMod.png}
\caption{Pantalla principal gestor de eventos - Panel de administración}
\end{figure}

En las figuras se diferencian las siguientes zonas.

\begin{itemize}
	\item \textbf{Zona 1. }Redirige al usuario a la administración de usuarios. Está formada por la gestión de la entidad Usuarios.
	\item \textbf{Zona 2. }Redirige al usuario a la administración de las rutas. Está formada por rutas, días y visitas.
	\item \textbf{Zona 3. }Redirige al usuario a la administración de los eventos. Está formada por eventos, días y lugares de evento. Está funcionalidad es la única ofrecida a los usuarios con rol de gestor de eventos.
	\item \textbf{Zona 4. }Redirige al usuario a la gestión de datos de Foursquare. Está formada por los lugares, categorías y subcategorías obtenidas de está fuente externa.
\end{itemize}


\subsubsection*{Pantalla administración usuarios}
Se muestran las entidades que forman la administración de usuarios. En este caso, los usuarios están gestionados mediante una única entidad. Si hubiese más entidades involucradas en dicha gestión aparecerían en pantalla en forma de listado.

\begin{figure}[H]
\centering
\includegraphics[
   keepaspectratio=true
]{./11_Apendice/Apendice_B/img/WebPanelUsuarios1.png}
\caption{Pantalla administración usuarios - Panel de administración}
\end{figure}

Haciendo click sobre el botón que aparece la derecha de la entidad, se mostrarán los correspondientes datos almacenados. Se mostrarán en formato tabla, siendo las columnas los atributos de la entidad.


\begin{figure}[H]
\centering
\includegraphics[
   keepaspectratio=true
]{./11_Apendice/Apendice_B/img/WebPanelUsuarios2.png}
\caption{Pantalla administración usuarios - Usuarios - Panel de administración}
\end{figure}

En la figura se diferencian cuatro zonas.


\begin{itemize}
	
	\item \textbf{Zona 1. }Botón que permite dar de alta un nuevo elemento a la entidad. Muestra la siguiente ventana emergente con el formulario a cubrir.
	
\begin{figure}[H]
\centering
\includegraphics[
   keepaspectratio=true
]{./11_Apendice/Apendice_B/img/WebPanelUsuariosAdd.png}
\caption{Pantalla añadir usuarios - Panel de administración}
\end{figure}

El administrador indicará los datos necesarios y confirmará la acción.
	
	\item \textbf{Zona 2. }Muestra el conjunto de acciones a realizar sobre cada elemento de la tabla. Incluye:
	\begin{itemize}
		\item \textbf{Editar. }Botón con el icono de un lápiz. Muestra una ventana con los datos de dicho elemento, permitiendo realizar modificaciones sobre ellos. Los campos ensombrecidos no pueden ser modificados.
		
		\begin{figure}[H]
		\centering
		\includegraphics[
   		keepaspectratio=true
		]{./11_Apendice/Apendice_B/img/WebPanelUsuariosEdit.png}
		\caption{Pantalla añadir usuarios - Panel de administración}
		\end{figure}
		
		\item \textbf{Eliminar. }Icono con el botón de una aspa. Muestra una ventana pidiendo confirmación para eliminar dicho elemento.
		\begin{figure}[H]
		\centering
		\includegraphics[
   		keepaspectratio=true
		]{./11_Apendice/Apendice_B/img/WebPanelUsuariosDel.png}
		\caption{Pantalla añadir usuarios - Panel de administración}
		\end{figure}
	\end{itemize}		
	
	\item \textbf{Zona 3. }Formada por un selector, un input y un botón de \textit{Aplicar}. En el selector se selecciona el atributo de la entidad sobre el cual se aplicará un filtro, en el input se indicará el valor por el que filtrar y accionando el botón se aplicará dicho filtro sobre la tabla.
	
	\item \textbf{Zona 4. }Indica la paginación de la tabla. Haciendo uso de las flechas, adelante y atrás, se podrán obtener los datos de la entidad de forma paginada.
\end{itemize}







\end{document}