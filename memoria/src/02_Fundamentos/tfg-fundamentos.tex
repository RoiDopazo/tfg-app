\chapter[Fundamentos tecnológicos]{
  \label{chp:fundamentos}
  FUNDAMENTOS TECNOLÓGICOS
}

\thispagestyle{numberingStyle}
\pagestyle{numberingStyle}


\section{Lenguajes utilizados}
\subsection{Java}
Java es un lenguaje de programación de propósito general, concurrente, orientado a objetos. Fue diseñado para tener tan pocas dependencias de implementación como fuera posible tal que permitiera a los desarrolladores escribir el programa una vez y ejecutarlo en cualquier dispositivo sin necesidad de recompilarlo.

Fue originalmente desarrollado por James Gosling, de Sun Microsystems, la cual fue adquirida por la compañía Oracle.

Puede ejecutarse en cualquier máquina virtual Java (JVM) sin importar la arquitectura de la computadora subyacente y su sintaxis deriva en gran medida de lenguajes como C y C++, pero con menos utilidades de bajo nivel.

\subsection{JavaScript}
TypeScript es un lenguaje de programación interpretado, dialecto del estándar ECMAScript, orientado a objetos, basado en prototipos, imperativo, débilmente tipado y dinámico.

Se utiliza principalmente del lado del cliente, implementado como parte de un navegador web, permitiendo mejoras en la interfaz de usuario y páginas web dinámicas.

JavaScript se diseñó con una sintaxis similar a C, aunque adopta nombres y convenciones del lenguaje de programación Java. Sin embargo, Java y JavaScript tienen semánticas y propósitos diferentes.

\subsection{TypeScript}
TypeScript es un lenguaje de programación libre y de código abierto desarrollado y mantenido por Microsoft.

Este lenguaje es un superset del ya conocido JavaScript y que está pensado para grandes proyectos, los cuáles a traves de un compilador de TypeScript se traducen a código JavaScript original.

Un aspecto característico de TypeScript es su sistema de tipos. Permite a los desarrolladores definir variables y funciones tipadas sin perder la esencia de JavaScript gracias a una representación estática de los tipos dinámicos. Definir tipos durante el diseño, nos ayudará a evitar errores en tiempo de ejecución.

\subsection{HyperText Markup Language}
HyperText Markup Language, conocido comúnmente por sus siglas, HTML, es un lenguaje de marcado cuya finalidad es la elaboración de páginas web. Define una estructura básica y un código (denominado código HTML) para la definición de contenido de una página web. 

Es un estándar a cargo del World Wide Web Consortium (W3C), organización dedicada a la estandarización de casi todas las tecnologías ligadas a la web.

\subsection{Cascading StyleSheets}
Las hojas de estilo en cascada (o CSS, por sus siglas en ingles.) es un lenguaje de diseño gráfico para definir y crear la presentación de un documento estructurado escrito en un lenguaje de marcado. Especifica como se mostrarán por pantalla, o otro tipo de media, los denominados elementos HTML.

Junto con HTML y JavaScript, CSS es una tecnología usada por muchos sitios web para crear páginas web visualmente atractivas, interfaces de usuario de aplicaciones web y GUIs para muchas aplicaciones móviles.

Su principal objeto es mantener la separación del contenido del documento de su forma de presentación. Con las hojas de estilo se puede prescindir del uso de formatos de estilo dentro de la propia página HTML, de manera que se pueda modificar el estilo de toda una web modificando un único archivo CSS.




\section{Frameworks, librerías y técnicas de desarrollo}
\subsection{Spring}
Spring es un framework cuya finalidad es facilitar el desarrollo de aplicaciones desarrolladas en Java. Es de código abierto y la primera versión fue elaborada por Rod Johnson. A pesar de que no impone ningún modelo de programación en particular, este framework se ha vuelto popular en la comunidad al ser considerado una alternativa, sustituto, e incluso un complemento al modelo Enterprise JavaBean (EJB)

Spring está compuesto de diversos módulos que se pueden agregar a nuestras aplicaciones, permitiendo a los desarrolladores agregar sólo los módulos que vayan usar. El único módulo necesario para trabajar con Spring es el Spring Core puesto que es el que contiene la DI (Inyección de Dependencias) y la configuración de uso de objetos Java.

\subsubsection*{Spring MVC}
Spring MVC es un framework de aplicaciones web basado en el patrón MVC(model-viewcontroller) y que alberga todas las ventajas del framework de Spring.
	\begin{itemize}
		 \item Separación clara de roles. Cada objeto controlador, validador, formulario, de modelo pueden ser realizados por objetos especializados.
		 \item Configuración potente y directa. Capacidad de configuración que permite una fácil referencia a través de contextos, como por ejemplo, desde controladores web a objetos de negocio.
		 \item Adaptabilidad, flexibilidad y no intrusividad. Definir un controlador usando une de las anotaciones de parámetros (@RequestParam, @PathVariable,…) para un escenario dado.
		 \item Código de negocio reutilizable. No existe necesidad de duplicación.
	\end{itemize}
	
Spring MVC es, como otros frameworks MVC, basado en solicitud (request-driven). Están diseñados en torno a un Servlet central que sirve las solicitudes a los controladores y ofrece unas funcionalidades que facilitan el desarrollo de las aplicaciones web. Sin embargo, el DispatcherServlet de Spring es más que eso, está completamente integrado con el contenedor Spring IoC y permite hacer uso de las características y funcionalidades de Spring.

\subsubsection*{Spring Security}
Spring Security ofrece exhaustivos servicios de seguridad para las aplicaciones empresariales basadas en Java EE. 
Las dos principales áreas en las que se enfoca Spring Security son la Autenticación y la Autorización, probablemente, los dos temas más relevantes en la seguridad de las aplicaciones.
	\begin{itemize}
		\item Autenticación es el proceso por el que se determina que uno es el que dice ser.
		\item Autorización hace referencia al proceso de determinar qué acción o acciones puede realizar en la aplicación.
	\end{itemize}
	
A nivel de autenticación, Spring Security soporta un amplio rango de modelos de autenticación. La mayoría de estos modelos de autenticación son proporcionados por terceros, o desarrollados por los organismos estándar pertinentes, como Internet Engineering Task Force. A mayores, Spring Security provee su propio conjunto de mecanismos de autenticación y soporta integración de autenticación con diferentes tecnologías


\subsection{Java Persistence API}
Java Persistence API, comúnmente conocida por sus siglas JPA, es la API que describe la gestión de datos relacionales en aplicaciones que utilicen Java. La primera especificación fue lanzada en mayo de 2006 como parte del trabajo del JSR 220.

JPA en sí mismo es solo una especificación, no un producto. Son un conjunto de interfaces que requieren una implementación. Existen implementaciones de JPA de código abierto y comerciales, y cualquier servidor de aplicaciones JAVA EE 5 debe proporcionar soporte para su uso.

El objetivo de esta API es no perder las ventajas de la orientación a objetos al interactuar con una base de datos y permitir usar objetos regulares, comúnmente conocidos como POJOs. 

\subsection{JAX-RS}
JAX-RS es la API de Java para la elaboración de servicios web RESTful que brinda soporte en la creación de servicios web de acuerdo con el patrón arquitectónico REST. Desde la versión 1.1, JAX-RS es una parte oficial de Java EE 6. Una característica notable de ser parte oficial de Java EE es que no es necesaria ninguna configuración para comenzar a utilizar JAX-RS.

Esta API utiliza anotaciones, introducidas en Java SE 5, para simplificar el desarrollo y la implementación de clientes y recursos web. 

De la misma forma que sucedía con JPA, JAX-RS no es más que una especificación, necesita un producto que la implemente. Jersey y RESTEasy son implementaciones de JAX-RS,

\subsection{Thymeleaf}
Thymeleaf es una librería Java que implementa un motor de plantillas válidas para entornos web como independientes. Es un software de código abierto creado originalmente por un ingeniero de software español llamado Daniel Fernández. No está hecho ni respaldado por ningún software de ninguna compañía y se ofrece al público de manera totalmente gratuita, tanto en formato binario como en código fuente, bajo licencia Apache.

Su objetivo principal es la creación de plantillas de una manera elegante y con un código bien formateado.

Thymeleaf ofrece una buena integración con Spring MVC a través de su dialecto SpringStandard, pero esta integración con Spring es completamente opcional y el dialecto estándar está destinado a usarse sin Spring.

\subsection{jQuery}
jQuery es una librería multiplataforma de JavaScript, creada incialmente por John Resig. Es un software libre y de código abierto, y posee un doble licenciamiento bajo la licencia MIT y la licencia Pública General de GNU, permitiendo su uso en proyectos libres y privados.

Su objetivo es la realización de funcionalidades basadas en JavaScript de forma rápida y sencilla. Permite realizar recorridos y manipulaciones de documentos HTML, manejar eventos, animaciones y usar AJAX  mucho más simple con una API fácil de usar que funciona en multitud de navegadores.

\subsection{Bootstrap}
Bootstrap es un kit de herramientas de código abierto para desarrollo web junto a HTML, CSS y JS.  

Originalmente creado por un diseñador y desarrollador de Twitter, Bootstrap es uno de los frameworks front-end y proyectos de código abierto más populares en el mundo. Antes de ser un framework de código abierto, Bootstrap era conocido como Twitter Blueprint.

Bootstrap incluye plantillas de diseño basadas en HTML y CSS para tipografías, formularios, botones, tablas,... así como complementos de JavaScript opcionales. También brinda la capacidad de crear fácilmente diseños receptivos.

\subsection{AJAX}
AJAX es una técnica de desarrollo web para crear aplicaciones interactivas. Es una tecnología asíncrona, en el sentido que los datos adicionales solicitados al servidor, se cargan en segundo plano sin inferir con la visualización ni el comportamiento de la página.

La funciones de llamada AJAX se efectúan, normalmente, bajo el lenguaje de programación JavaScript mientras que el acceso a los datos se realiza mediante XMLHttpRequest.

AJAX no constituye una tecnología en sí misma, sino que es un término que engloba a un grupo de éstas que trabajan conjuntamente.

	\begin{itemize}
		\item XHTML (o HTML) y CSS para el diseño que acompaña a la información.
		\item Document Object Model (DOM) accedido con un lenguaje de scripting por parte del usuario, generalmente, JavaScript.
		\item El objeto XMLHttpRequest para intercambiar datos de forma asíncrona con el servidor web.
		\item XML es el formato usado generalmente para la transferencia de datos solicitados al servidor.
	\end{itemize}
	

\section{Herramientas de desarrollo}
\subsection{Eclipse}
Eclipse es una plataforma software compuesto por un conjunto de herramientas de programación de código abierto multiplataforma.

Fue desarrollado originalmente por IBM como el sucesor de su familiar de herramientas VisualAge. Ahora, está siendo desarrollado por la Fundación Eclipse, una organización independiente, sin ánimo de lucro, que fomenta una comunidad de código abierto y un conjunto de productos complementarios.

\subsection{Maven}
Maven es una herramienta de software para la gestión y construcción de proyectos Java creada por Jason van Zyl en 2002 que tiene un modelo de configuración de construcción basado en formato XML.

Maven utiliza un Project Object Model, conocido como POM, para describir el proyecto software a construir, sus dependencias de otros módulos y componentes externos, y el orden de construcción de los elementos.

Una de sus características clave son su disponibilidad para usarse en la red puesto que el motor incluido en su núcleo puede dinámicamente descargar plugins de un repositorio.


\subsection{Git}
Git es un sistema de control de versiones distribuido de código abierto y gratuito diseñado para manejar todo, desde proyectos pequeños a muy grandes, con velocidad y eficiencia.

Originalmente fue diseñado como un motor de sistema de control de versiones de bajo nivel sobre el cual otros podían codificar interfaces frontales. Desde entonces hasta ahora, el núcleo del proyecto Git se ha vuelto un sistema de control de versiones completo, utilizable en forma directa.

Git es fácil de aprender y ofrece un rendimiento increíblemente rápido. Su principal objetivo es llevar el registro de cambios en archivos y coordinar el trabajo que varias personas realizan sobre archivos compartidos.


\subsection{Oracle SQL Developer}
Oracle SQL Developer es un entorno de desarrollo integrado y gratuito que simplifica el desarrollo y la administración de las bases de datos de Oracle, tanto de implementaciones tradicionales como en la nube.

Este software admite productos de Oracle y una grana variedad de complementos de terceros que los usuarios pueden implementar para conectarse a bases de datos que no sean de Oracle.

SQL Developer ofrece un desarrollo completo de extremo a extremo de sus aplicaciones PL / SQL, una hoja de trabajo para ejecutar consultas y scripts, una consola DBA para administración, una interfaz de informes y una solución completa de modelado de datos.



\section{Sistema de gestión de bases de datos}
\subsection{Oracle Database}
Oracle Database es un sistema de gestión de base de datos de tipo objeto-relacional, desarrollado por Oracle Corporation. Es la base de datos más popular para el procesamiento de transacciones el línea(OLTP) y almacenes de datos (Data warehousing).

La tecnología Oracle se encuentra prácticamente en todas las industrias alrededor y es el proveedor mundial líder de software para administración de información.

El producto Oracle para el sistema de base de datos cuenta con 7 ediciones diferentes de las cuales, una es completamente gratuita.

\section{Servidor de aplicaciones}
\subsection{Apache Tomcat}
Apache Tomcat funciona como un contenedor de servlets desarrollado por Apache Software Foundation. Es un una implementación de código abierto de las tecnologías Java Servlet, JavaServer Pages, Java Expression Language y Java WebSocket. Se publica bajo la versión 2 de Apache License.

Tomcat no es un servidor de aplicaciones, pero puede funcionar como servidor web por sí mismo. Actualmente es usado como servidor web autónomo en entornos con alto nivel de tráfico y alta disponibilidad.

Está escrito en Java, de manera que puede funcionar en cualquier sistema operativo que disponga de la máquina virtual Java.



\section{Sin especificar}
\subsection{Ionic}
Ionic es un completo SDK de código abierto diseñado para el desarrollo de aplicaciones móviles híbridas. La versión original fue lanzada en 2013 y construida sobre AngularJS y Apache Cordova. Las versiones más recientes, conocidas como Ionic 3, están construidas sobre Angular.

Proviene de herramientas y servicios para desarrollar aplicaciones móviles híbridas usando tecnologías web (CSS, HTML,...).

Ionic proviene de toda la funcionalidad que se puede encontrar en los SDK de desarrollo móvil. Las aplicaciones construidas se pueden personalizar en función del uso final, Android o iOS, por ejemplo.

Además del SDK, Ionic también brinda servicios que los desarrolladores pueden usar para habilitar funciones, como la tecnología push, test A/B, construcciones automáticas, etc...

Ionic también proporciona una poderosa interfaz de línea de comandos, lo que permite poder comenzar a desarrollar un proyecto con un simple comando. También permite a los desarrolladores agregar complementos de Cordova.


