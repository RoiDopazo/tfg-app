\chapter[Conclusiones y líneas futuras]{
  \label{chp:conclusiones}
  CONCLUSIONES Y LÍNEAS FUTURAS
}
\thispagestyle{numberingStyle}
\pagestyle{numberingStyle}

\section{Conclusiones}

El objetivo primordial del desarrollo de este proyecto era la elaboración de una aplicación para la gestión y planificación de rutas turísticas y, el resultado obtenido, cumple con todos los requisitos establecidos en el anteproyecto.

A lo largo del desarrollo se fueron interponiendo por el camino muchas dificultades. El no uso previo de algunas de las tecnologías empleadas  y las dificultadas surgidas con el uso de las APIs externas, han supuesto los mayores problemas durante el desarrollo. Esto supuso realizar una gran inversión de tiempo en conocimiento y comprendimiento de las nuevas tecnologías a utilizar, como por ejemplo, con el framework Ionic, herramienta utilizada por primera vez. Con respecto a las APIs, en concreto la API de Foursquare, se ofrecía una librería Java obsoleta. Ha sido necesario modificar y compilar de nuevo dicha librería para poder emplearla más fácilmente en el proyecto. Con la otra API externa utilizada, la API de Google, surgieron problemas de configuración tratando de emplear dicha API como elemento nativo en la aplicación móvil.

Por otra parte, partiendo de que la aplicación será consumida por usuarios finales con, probablemente, escaso nivel informático, se ha conseguido elaborar una aplicación móvil atractiva, clara y de fácil uso; uno de los requisitos más valorados por los usuarios finales.

Personalmente, uno de los objetivos a la hora de realizar este proyecto era conocer y poder trabajar con una de las herramientas más utilizadas hoy en día, el framework Ionic. Como consecuencia, se valora el conocimiento adquirido con esta herramienta y que sirve de punto de partida en el mundo del desarrollo de aplicaciones móviles.

Finalmente, la realización de este proyecto, plasma muchos de los conocimientos teóricos adquiridos en los últimos años. Gracias a estos conocimientos, se ha podido desarrollar un software de calidad, fácilmente escalable, que supone una aplicación base estable para el continuo proceso de desarrollo y mejora.


\section{Líneas futuras}
La aplicación creada supone una base inicial para seguir trabajando y mejorando su funcionalidad. Como trabajo futuro, se plantean dos metas a seguir, unas a corto plazo que permiten una mejora actual y continua del sistema, y otras a largo plazo; prestando más ambición en el futuro de la aplicación.

A corto plazo se consideran las siguientes metas:

\begin{itemize}

	\item \textbf{Mejorar interfaz}. El usuario final será muy crítico con la interfaz de la aplicación. Será necesario prestar atención al feedback de los usuarios para poder mejorar aquellos aspectos más críticos. Otro tema importante, es mejorar la visualización de las rutas en los mapas. Para ello, sería conveniente profundizar en la API de Google para mostrar mapas y hacerlos más atractivos para los usuarios.
	
	\item \textbf{Obtención de los datos de geolocalización}. Actualmente, estos datos se obtiene en segundo plano cuando el usuario solicita registrarlos para una ruta concreta. A mayores, estos datos se transmiten al sistema conforme se van obteniendo, de manera que si no se dispone de una conexión de internet en determinado momento, no será posible registrar dicha información. 
	La idea futura es solucionar estos inconvenientes pretendiendo obtener dichos datos de forma automática, sin necesidad de que el usuario confirme esta acción, únicamente tendrá la opción de escoger, previamente, si deseará registrar estos datos o no. De esta forma, cuando la ruta planificada se encuentre en el día concreto, el sistema activará automáticamente la geolocalización en segundo plano, y a mayores, almacenará los datos internamente y los remitirá al servidor cuando exista conexión de red, permitiendo evitar la perdida de esa información.

\end{itemize}

Con respecto al trabajo a más a largo, se consideran los siguientes objetivos.

\begin{itemize}
	\item \textbf{Avance hacia red social}. La idea reside en ampliar esta aplicación hasta convertirla en una pequeña red social. Con lo realizado, la interacción social de la aplicación solo permite la consulta de rutas de otros usuarios. La idea consistiría en poder incorporar ciertas funcionalidades como copiar rutas de otros usuarios a tus rutas, permitir comentarios sobre las rutas, o poder crear grupos de usuarios que realicen una misma ruta, permitiendo hacer una aplicación más social.
	
	\item \textbf{Gestión de eventos}. Actualmente, los eventos de la aplicación son gestionados por usuarios con permisos específicos. El objetivo sería eliminar este tipo de usuario encargado de gestionar los eventos y hacer que esta tarea resida en los usuarios finales de la aplicación. Para ello, todos podrían crear eventos o modificarlos, pero sería necesario incorporar un sistema de validación entre usuarios, evitando por ejemplo, que se den de alta eventos no reales. Otra solución, que podría mejorar la gestión de eventos, sería obtenerlos de fuentes externas como podría ser Facebook, lo que permitiría aprovechar también las funcionalidades sociales de esta aplicación. Ambas propuestas no son excluyentes, y podrían coexistir las dos en la aplicación.

	\item \textbf{Incorporar funcionalidades}. El objetivo sería, partiendo de las funcionalidades ya implementadas, incorporar más posibilidades de personalización en las rutas. Esto consistiría en poder crear rutas que se ubiquen en lugares o ciudades diferentes; estableciendo métodos de viaje entre estos lugares, ya sea indicando tren, avión o método de transporte que se use y permitir hacer pagos o reservas en lugares o eventos que se visiten (museos, conciertos...). También se plantea aumentar la fuente de datos, intentado recabar información de lugares o ubicaciones no contempladas en Foursquare, o que simplemente enriquezcan la información obtenida de este fuente externa, permitiendo caracterizar las diferentes rutas, como por ejemplo, en turísticas, de aventura o de ocio.

\end{itemize}












