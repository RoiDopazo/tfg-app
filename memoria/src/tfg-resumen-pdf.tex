\documentclass[a4paper,12pt,twoside]{book}
\usepackage{tfg}
\begin{document}
\cleardoublepage
\begin{flushright}
\large{
\textbf{Título: } Aplicación para la planificación y gestión
de viajes turísticos\\
\textbf{Autor: } Rodrigo Dopazo Iglesias\\
\textbf{Director: } Juan Raposo Santiago\\
\textbf{Convocatoria: } Julio\\
\hfill \break}
\end{flushright}

\noindent
\textbf{\LARGE{Resumen}}


Se diseñará y desarrollará una aplicación que permita la gestión y planificación de viajes turísticos. En un mundo totalmente globalizado, el turismo es uno de los principales motores económicos de la sociedad, lo que permitirá a esta aplicación estar destinada a cualquier tipo de usuario que quiera gestionar y planificar sus viajes, en lo que a visitar una ciudad se refiere.

Se empleará una metodología basada en el Proceso Unificado. Se pretenderá obtener un producto final de calidad, dividiendo el trabajo en pequeños proyectos que permitirán incrementar y mejorar el producto.

En cuanto a su funcionalidad, la aplicación permitirá obtener las diferentes ciudades del mundo sobre las cuales se podrán crear rutas. Para cada ciudad, se podrán consultar los lugares y sitos de interés, obtenidos de una de las fuentes de datos de geolocalización más importantes, Foursquare. Estos sitios o lugares, podrán ser incorporados a las rutas de los usuarios y establecer sobre ellos la planificación. De igual forma, se ofrecerán eventos de carácter temporal, gestionados por la propia aplicación, que podrán ser incorporados a las rutas. Con todos estos datos, el usuario podrá crear y personalizar sus rutas y elaborar una completa planificación en tiempo y distancia. Las rutas creadas podrán ser visualizadas más atractivamente mediante el uso de mapas, en los que a mayores, se podrán mostrar los datos reales de ejecución de la ruta gracias al uso de la geolocalización, lo que permitirá al usuario comparar la planificación realizada con la realmente ejecutada.

El objetivo será el desarrollo, tanto de una aplicación web como de una aplicación móvil. La aplicación móvil ofrecerá todas las funcionalidades definidas mientras que la aplicación web implementará funcionalidades reducidas. Por su contra, se elaborará un panel de administración de la aplicación que estará disponible, únicamente, desde la interfaz web. La aplicación web será implementada en la tecnología de desarrollo Java, mientras que en la aplicación móvil, se empleará el framework Ionic para el desarrollo de aplicaciones híbridas. Ambas aplicaciones, accederán a un servicio de datos web implementado en Java.

\
\

\noindent
\textbf{\LARGE{Palabras clave}}

Aplicación, Java, Web, Móvil, Ionic, Rest, JPA, Spring, MVC, MVVM, Viaje turístico, Planificador, Rutas, Itinerario, Lugares, Eventos, Geolocalización, Foursquare, Google Maps.

\end{document}