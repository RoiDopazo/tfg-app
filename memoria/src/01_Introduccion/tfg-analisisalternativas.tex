\chapter[Análisis de alternativas]{
  \label{chp:analisisdealternativas}
  ANÁLISIS DE ALTERNATIVAS
}
\thispagestyle{numberingStyle}
\pagestyle{numberingStyle}



Antes de comenzar el desarrollo de este proyecto, se realizó un breve estudio de algunas de las aplicaciones, de temática similar, disponibles en el mercado. 

La primera en ser analizada ha sido la propia aplicación de Foursquare que, como se comentó con anterioridad, será la fuente de datos de lugares de la aplicación. La aplicación de Foursquare, tanto en su versión web como móvil, ofrece la posibilidad de buscar lugares, sitios de interés y demás, según diferentes criterios, y que pueden ser guardados por el usuario en lo que la aplicación denomina listas. Sin embargo, Foursquare no permite establecer ningún tipo de planificación ni ruta entre dichos sitios.

Por otra parte, existen numerosas aplicaciones enfocadas principalmente a la planificación de rutas. Una de las más populares es la aplicación móvil Google Trips: Travel Planner, aplicación del grupo Google. Está aplicación ofrece gran nivel de detalle en lo que se refiere a la planificación de viajes y se puede considerar una aplicación todo en uno. Permite crear viajes en función de las reservas remitidas al correo electrónico de Google del usuario, añadir planes de viaje diarios, mostrar información sobre el transporte disponible, ofrecer los diferentes descuentos disponibles a determinados lugares, como pueden ser museos, y demás funcionalidades relacionadas.

En su contra, a la hora de establecer los planes diarios, esta aplicación no permite una completa personalización. El modo de transporte utilizado será escogido por la propia aplicación, siendo este el más eficiente en tiempo. Por otra parte, te recomienda el tiempo a pasar en cada visita en función de la media de los demás usuarios, pero no ofrece la posibilidad de establecer un tiempo específico. Estas limitaciones impiden conocer el tiempo total que empleará el usuario en el plan de ese día, a qué horas debería salir de un determinado sitio para llegar a otro antes de una hora específica, modificar el modo de viaje en función de sus necesidades, y demás.


A mayores de las aplicaciones mencionadas, existe gran variedad de ellas que ofrecen una funcionalidad principal similar, la planificación de viajes turísticos. Cada una de ellas presenta características diferentes, como pueden ser, la fuente de datos utilizada, el nivel de detalle en la planificación, el número de funcionalidades ofrecidas, etc. Estas características y la manera de cómo ofrecerlas, serán las que hagan únicas a las aplicaciones en el mercado y que harán que los usuarios se decanten por el uso de unas u otras.












