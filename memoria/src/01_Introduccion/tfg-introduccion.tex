\setcounter{page}{1}
\chapter[Introducción]{
  \label{chp:introduccion}
  INTRODUCCIÓN
}
\thispagestyle{numberingStyle}
\pagestyle{numberingStyle}


\section{Contextualización}
Durante las últimas décadas, el turismo ha experimentado un continuo crecimiento que lo ha llegado a convertirse en uno de los sectores económicos más importantes. Por su parte, el éxito de la telefonía móvil y el continuo uso de smartphones en la sociedad, promulga el desarrollo de las aplicaciones móviles, otro sector que continúa en pleno crecimiento. Es entre estos dos sectores donde se engloba el desarrollo de la aplicación.

Se pretende presentar a los usuarios finales una aplicación móvil, fácilmente accesible, que permita ofrecer un servicio que ayude a planificar rutas turísticas en un viaje determinado. La creación y planificación de rutas es el objetivo primordial de la aplicación pero no su única funcionalidad. Los usuarios podrán consultar los diferentes lugares del mundo, con un alto nivel de filtrado, permitiendo obtener lugares de una característica o estilo concreto. También podrán consultar eventos, que tendrán una duración de carácter temporal y estarán formados por un conjunto actividades.

Los lugares ofrecidos al usuario serán obtenidos de la base de datos de Foursquare. Foursquare es un servicio basado en localización nacido en 2009 como una red social donde los usuarios hacían registros y recomendaciones sobre los sitios y lugares que visitaban. Con el paso de los años, la situación de la empresa fue empeorando drásticamente hasta perder la popularidad que había conseguido en sus comienzos. En los últimos años, Foursquare se ha rejuvenecido y ahora es una compañía basada en inteligencia de localización. Posee una base de datos muy amplia y de gran nivel que le permite ofrecer esa información a muchas aplicaciones y empresas.







\section{Objetivos}

\section{Estructura de la memoria}
En este apartado se detallará el contenido de la memoria donde se incluirá una pequeña descripción para cada capítulo que la componen.

\begin{itemize}
	\item \textbf{Capitulo 2 - Análisis de alternativas}. 
	\item \textbf{Capitulo 3 - Fundamentos tecnológicos}. 
	\item \textbf{Capitulo 4 - Metodología}. 	
	\item \textbf{Capitulo 5 - Planificación y costes}. 
	\item \textbf{Capitulo 6 - Análisis}. 
	\item \textbf{Capitulo 7 - Diseño}. 
	\item \textbf{Capitulo 8 - Implementación}. 
	\item \textbf{Capitulo 9 - Pruebas}. 
	\item \textbf{Capitulo 10 - Conclusiones}. 
	\item \textbf{Capitulo 11 - Trabajo futuro}. 
	\item \textbf{Capitulo 12 - Referencias}. 
	\item \textbf{Apéndice A - Diagramas}.
	\item \textbf{Apéndice B - Manual de usuario}.  
\end{itemize}

















