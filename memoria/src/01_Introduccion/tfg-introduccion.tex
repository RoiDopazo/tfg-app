\setcounter{page}{1}
\chapter[Introducción]{
  \label{chp:introduccion}
  INTRODUCCIÓN
}
\thispagestyle{numberingStyle}
\pagestyle{numberingStyle}


\section{Contextualización}
Durante las últimas décadas, el turismo ha experimentado un continuo crecimiento que lo ha llevado a convertirse en uno de los sectores económicos más importantes. Por su parte, el éxito de la telefonía móvil y el continuo uso de smartphones en la sociedad, promulga el desarrollo de las aplicaciones móviles, otro sector que continúa en pleno crecimiento. Es entre estos dos sectores donde se engloba el desarrollo de la aplicación.

Se pretende presentar a los usuarios finales una aplicación móvil, fácilmente accesible, que permita ofrecer un servicio que ayude a planificar rutas turísticas en un viaje determinado. La creación y planificación de rutas es el objetivo primordial de la aplicación pero no su única funcionalidad. Los usuarios podrán consultar los diferentes lugares del mundo, con un alto nivel de filtrado, permitiendo obtener lugares de una característica o estilo concreto. También podrán consultar eventos, que tendrán una duración de carácter temporal y estarán formados por un conjunto actividades.

Los lugares ofrecidos al usuario serán obtenidos de la base de datos de Foursquare. Foursquare es un servicio basado en localización nacido en 2009 como una red social donde los usuarios hacían registros y recomendaciones sobre los sitios y lugares que visitaban. Con el paso de los años, la situación de la empresa fue empeorando drásticamente hasta perder la popularidad que había conseguido en sus comienzos. En los últimos años, Foursquare se ha rejuvenecido y ahora es una compañía basada en inteligencia de localización. Posee una base de datos muy amplia y de gran nivel que le permite ofrecer esa información a muchas aplicaciones y empresas, y que también servirá a esta aplicación.

También, se pretenderá otorgar a la aplicación un carácter de red social. Las aplicaciones de redes sociales son uno de los mayores atractivos para los usuarios de aplicaciones, y el número de usuarios que las usan aumentan diariamente. De esta forma, otorgando esta característica a nuestra aplicación, permitiendo compartir y consultar rutas hechas o planificadas por demás usuarios de la aplicación, conseguiremos otorgarle un valor añadido a nuestra aplicación.


\section{Objetivos}
El desarrollo de esta aplicación supone la consecución de los siguientes objetivos.

\begin{itemize}

	\item \textbf{Planificar rutas}. El objetivo principal es crear y planificar rutas. La planificación incluirá poder determinar la ciudad o lugar de destino y las fechas en las que se va a realizar dicho viaje. También incluye la posibilidad de buscar lugares de interés o eventos en determinada ciudad y especificar las diferentes horas que se quiere pasar en cada uno de ellos, así como, calcular las distancias que hay entre cada uno de los lugares o eventos incorporados en cada ruta.
	\item \textbf{Obtención datos externos}. Será necesario poder establecer una comunicación con las fuentes de datos externos que nos permita obtener los lugares y calcular las distancias. Este objetivo también incluye la gestión y almacenamiento de estos datos en nuestra aplicación.
	\item \textbf{Gestionar eventos}. Tiene como objetivo crear, modificar, eliminar y consultar los eventos utilizados por la aplicación. Estos eventos, serán consultados por los usuarios cuando realicen la planificación de sus rutas y podrán incorporarlos en ellas.
	\item \textbf{Gestionar de usuarios}. Se pretende conseguir una completa gestión de usuarios que les permita crear rutas propias y compartirlas con los demás usuarios. También, se pretende ofrecer que un usuario pueda marcar una ruta como privada, de forma que no pueda ser accedida por los demás.
	\item \textbf{Aplicación móvil}. El objetivo de la aplicación es poder ser usada mediante un smartphone o dispositivo móvil inteligente. En la aplicación móvil se pretende incorporar el uso de mapas, permitiendo una planificación de rutas de manera visual y simulada en el mapa. También, como objetivo secundario, se establece la elaboración de una aplicación web con funcionalidades reducidas con respecto a la aplicación móvil y que también permite la administración remota de los datos almacenados en la aplicación.
	\item \textbf{Datos en tiempo real}. El objetivo es poder obtener los datos de geolocalización reales para el día exacto en que se realice la ruta y consultarlos posteriormente en el mapa de la aplicación. De esta forma, los usuarios podrán comparar la planificación de sus rutas con su ejecución real.
	\item \textbf{Mapas de Google Maps}. El objetivo de mostrar la información de las rutas en un mapa pasará por el uso de Google Maps. Será necesario configurar, tanto la aplicación móvil como web, para que hagan uso de estos mapas y muestren, visualmente, tanto la planificación realizada como los datos reales obtenidos.
	
\end{itemize}


\section{Estructura de la memoria}
En este apartado se detallará el contenido de la memoria. Para ello, se incluirá una pequeña descripción para cada capítulo que la componen.

\begin{itemize}
	\item \textbf{Capitulo 2 - Análisis de alternativas}. Se comentarán y analizarán las aplicaciones existentes en el mercado actual que poseen una finalidad semejante a la desarrolla en este proyecto.  
	\item \textbf{Capitulo 3 - Fundamentos tecnológicos}. Capítulo en el que se describirán cada uno de los elementos y componentes, de carácter tecnológico, que sirvieron para realizar la aplicación.
	\item \textbf{Capitulo 4 - Metodología}. En este capítulo se detallará la metodología de software empleada y se explicarán sus características.
	\item \textbf{Capitulo 5 - Planificación y costes}. Se realizará la planificación del proyecto y se detallará el coste de cada uno de las fases que lo componen.
	\item \textbf{Capitulo 6 - Análisis}. Apartado donde se detallarán los requisitos necesarios para realizar la aplicación. Se incluirán los diagramas y especificaciones de los casos de uso definidos.
	\item \textbf{Capitulo 7 - Diseño}. En este capítulo se presentará el diseño elaborado para la realización de la aplicación. Se detallará la arquitectura del sistema y de cada uno de los módulos que forman la aplicación.
	\item \textbf{Capitulo 8 - Implementación}. Se explicará la estructura del proyecto y se mostrarán ejemplos de implementación de algunos módulos de la aplicación.
	\item \textbf{Capitulo 9 - Pruebas}. Se detallarán y explicarán las pruebas realizadas.
	\item \textbf{Capitulo 10 - Conclusiones y trabajo futuro}. Capítulo con las conclusiones finales adquiridas después de realizar el proyecto y con las líneas de trabajo futuro a seguir para la continua mejora de la aplicación.
	\item \textbf{Apéndice A - Diagramas}. Apéndice en el que se incluirán diagramas de la aplicación.
	\item \textbf{Apéndice B - Manual de usuario}. Apéndice en el que se explicará la guía de uso de la aplicación.
\end{itemize}

















