\chapter[Análisis]{
  \label{chp:analisis}
  ANÁLISIS
}
\thispagestyle{numberingStyle}
\pagestyle{numberingStyle}

En este apartado se explicará, detalladamente, el análisis realizado.

\section{Análisis de requerimientos}

\subsection{Requerimientos funcionales}
A continuación se muestran los requerimientos funcionales del sistema, clasificados en distintas áreas.

\subsubsection*{Acceso a la aplicación}
\begin{itemize}
\setlength\itemsep{1pt}
\item El sistema ofrecerá la posibilidad de que un usuario se registre en la aplicación.
\item El sistema ofrecerá la posibilidad de que el usuario se identifique en el sistema. Los usuarios deben ingresar al sistema con  nombre de usuario y contraseña.
\end{itemize}

\subsubsection*{Cliente Móvil}
\begin{itemize}
\setlength\itemsep{1pt}
\item El sistema ofrecerá la posibilidad de crear nuevas rutas.
\item El usuario podrá consultar las rutas, propias y de otros usuarios, según ciertos criterios de búsqueda.
\item El sistema permitirá a los usuarios autorizados eliminar las rutas propias que deseen.
\item El sistema permitirá a los usuarios autorizados consultar los detalles de las rutas.
\item Para las rutas, el sistema permitirá:
	\begin{itemize}
	\item Establecer las fechas de inicio y fin.
	\item Consultar, asignar y desasignar a la ruta, los eventos disponibles en esas fechas.
	\item Consultar el itinerario por días.
	\item Modificar la hora de comienzo establecida para cada día.
	\item Consultar, añadir y eliminar lugares de interés a cada día de la ruta.
	\item Editar el modo de viaje a realizar entre diferentes lugares.
	\item Mostrar el itinerario, por días y en total, en el mapa.
	\item Habilitar y deshabilitar el sistema de geolocalización para conocer la ruta hecha en tiempo real.
	\item Consultar y comparar, el itinerario definido con el obtenido a tiempo real.
	\item Editar los permisos de la ruta.
	\end{itemize}
\end{itemize}

\subsubsection*{Cliente Web}
\begin{itemize}
\setlength\itemsep{1pt}
\item El usuario podrá consultar las rutas existentes, propias y de otros usuarios, según ciertos criterios de búsqueda.
\item El sistema permitirá a los usuarios autorizados consultar los detalles de las rutas.
\item El sistema permitirá a los usuarios autorizados marcar las rutas propias como privadas, con el fin de no compartirlas con los demás usuarios.
\item El sistema solo ofrecerá la posibilidad de consulta sobre los detalles de una ruta, permitiendo ver el itinerario, si tiene datos en tiempo real guardados, etc...
\item El sistema permitirá a los usuarios autorizados eliminar las rutas propias que deseen.
\end{itemize}

\subsubsection*{Cliente Administración Web}
\begin{itemize}
\setlength\itemsep{1pt}
\item El sistema solo permitirá acceso a usuarios con permisos de administración.
\item El sistema permitirá a los usuarios con dichos permisos, dar de alto nuevos usuarios.
\item El sistema permitirá las altas, bajas, modificaciones y consultas de las entidades del sistema.
	\begin{itemize}
	\item El sistema ofrecerá la posibilidad de crear, eliminar, modificar y consultar datos de usuarios.
	\item El sistema ofrecerá la posibilidad de crear, eliminar, modificar y consultar datos de rutas.
	\item El sistema ofrecerá la posibilidad de crear, eliminar, modificar y consultar datos de lugares.
	\item El sistema ofrecerá la posibilidad de crear, eliminar, modificar y consultar datos de categorías.
	\item El sistema ofrecerá la posibilidad de crear, eliminar, modificar y consultar datos de eventos.
	\end{itemize}
\item El sistema permitirá la existencia de usuarios con capacidades para la administración y gestión, exclusivamente, de los eventos. Permitiendo así, altas, bajas, modificaciones y consultas de los mismos en el sistema.
\end{itemize}

\subsubsection*{Seguridad}
\begin{itemize}
\setlength\itemsep{1pt}
\item El sistema ofrecerá la posibilidad de que el usuario modifique sus datos de acceso al sistema.
\item El sistema solo permitirá acciones correctamente autenticadas, exceptuando las de acceso a la aplicación.
\item Los usuarios de la aplicación solo podrán modificar los datos para los que tengan autorización. Un usuario no podrá modificar la información de los recursos de los que no es propietario.
\end{itemize}


\subsection{Requerimientos no funcionales}


\subsubsection*{Rendimiento}
En condiciones normales, el tiempo de respuesta a las peticiones realizadas no deberá superar el máximo de 6 segundos.

\subsubsection*{Disponibilidad}
La proporción de tiempo que el sistema debe estar en condiciones funcionales deberá ser del 99\%.


\subsubsection*{Portabilidad}
El sistema diseñado y sus componentes deben ser capaces de ser transferidos entre plataformas GNU/Linux y Windows ofreciendo facilidad de adaptación e instalación.

\subsubsection*{Facilidad de Uso}
La capacidad del software para ser comprendido, aprendido y utilizado de forma amigable por el usuario es un requerimiento esencial. Por ello, se debe ofrecer al usuario una interfaz sencilla y atractiva, con un manual de uso que describa el funcionamiento y uso del sistema final.

\subsubsection*{Seguridad}
El sistema solo permitirá el acceso autorizado a la información y el intercambio de esta por la red será exclusivamente mediante el uso del protocolo encriptado https.

\section{Modelo de casos de uso}

\subsection{Actores del sistema}
Analizando los requerimientos funcionales del sistema, se detectan cuatro tipos de actores, que demandan una determinada funcionalidad en el sistema. Estos cuatro actores son, el cliente móvil, el cliente web, el administrador y el gestor de eventos.


\begin{figure}[!htbp]
\centering

\includegraphics[
   keepaspectratio=true
]{./../Diagrams/Actores.png}
\caption{Diagrama casos de uso - Actores.}
\end{figure}




\subsection{Diagrama de casos de uso}
Tras conocer los requerimientos funcionales del sistema y reconocer las necesidades del sistema, se ha optado por diseñar un sistema general compuesto por los diferentes subsistemas del mismo.


\FloatBarrier
\subsubsection*{Diagrama sistema general}
Dicho sistema, estará compuesto por tres grupos de subsistemas: subsistema de acceso, subsistema de administración y subsistema de aplicación.

\begin{figure}[!htbp]
\centering

\includegraphics[
   keepaspectratio=true
]{./../Diagrams/uc__Sistema-General.png}
\caption{Diagrama casos de uso - Sistema general.}
\end{figure}


\FloatBarrier
\subsubsection*{Diagrama sistema de acceso}
\begin{figure}[!htbp]
\centering

\includegraphics[
   keepaspectratio=true
]{./../Diagrams/uc__Sistema-Acceso.png}
\caption{Diagrama casos de uso - Sistema de acceso.}
\end{figure}


\FloatBarrier
\subsubsection*{Diagrama sistema aplicación móvil}
\begin{figure}[!htbp]
\centering

\includegraphics[
   keepaspectratio=true
]{./../Diagrams/Sistema-Aplicacion-Movil.png}
\caption{Diagrama casos de uso - Sistema aplicación móvil.}
\end{figure}


\FloatBarrier
\subsubsection*{Diagrama sistema aplicación web}
\begin{figure}[!htbp]
\centering

\includegraphics[
   keepaspectratio=true
]{./../Diagrams/Sistema-Aplicacion-Web.png}
\caption{Diagrama casos de uso - Sistema aplicación web.}
\end{figure}


\FloatBarrier
\subsubsection*{Diagrama sistema administración}
\begin{figure}[!htbp]
\centering

\includegraphics[
   keepaspectratio=true
]{./../Diagrams/Sistema-Administracion.png}
\caption{Diagrama Casos de Uso - Sistema Administración.}
\end{figure}


\FloatBarrier
\subsubsection*{Diagrama sistema gestión usuarios}
\begin{figure}[!htbp]
\centering

\includegraphics[
   keepaspectratio=true
]{./../Diagrams/Sistema-Gestion-Usuarios.png}
\caption{Diagrama casos de uso - Sistema gestión usuarios.}
\end{figure}


\FloatBarrier
\subsubsection*{Diagrama sistema gestión rutas}
\begin{figure}[!htbp]
\centering

\includegraphics[
   keepaspectratio=true
]{./../Diagrams/Sistema-Gestion-Rutas.png}
\caption{Diagrama casos de uso - Sistema gestión rutas.}
\end{figure}


\FloatBarrier
\subsubsection*{Diagrama sistema gestión visitas}
\begin{figure}[!htbp]
\centering

\includegraphics[
   keepaspectratio=true
]{./../Diagrams/Sistema-Gestion-Visitas.png}
\caption{Diagrama casos de uso - Sistema  gestión visitas.}
\end{figure}


\FloatBarrier
\subsubsection*{Diagrama sistema gestión lugares}
\begin{figure}[!htbp]
\centering

\includegraphics[
   keepaspectratio=true
]{./../Diagrams/Sistema-Gestion-Lugares.png}
\caption{Diagrama casos de uso - Sistema gestión lugares.}
\end{figure}


\FloatBarrier
\subsubsection*{Diagrama sistema gestión eventos}
\begin{figure}[!htbp]
\centering

\includegraphics[
   keepaspectratio=true
]{./../Diagrams/Sistema-Gestion-Eventos.png}
\caption{Diagrama casos de uso - Sistema gestión eventos.}
\end{figure}


\FloatBarrier
\subsubsection*{Diagrama sistema gestión categorías}
\begin{figure}[!htbp]
\centering

\includegraphics[
   keepaspectratio=true
]{./../Diagrams/Sistema-Gestion-Categorias.png}
\caption{Diagrama casos de uso - Sistema gestión categorías.}
\end{figure}



\newpage
\subsection{Especificación casos de uso}


\newpage
\subsubsection*{Caso de uso: Autenticarse}
\begin{longtable}{| p{4cm} | p{10cm} |}
\endfirsthead
\multicolumn{2}{c}{\textit{Continúa de la página anterior}}\\[12pt]
\hline
\endhead
\hline
\multicolumn{2}{c}{\textit{Continúa en la siguiente página}} \\
\endfoot
\hline
\caption{Caso de Uso: Autenticarse}\label{fig:1}\\
\endlastfoot


\hline
\multicolumn{2}{|c|}{\textbf{CU$<$01$>$ - Autenticarse}} \\

\hline
\textbf{Descripción} &
El usuario se identifica introduciendo las credenciales de acceso en el sistema \\

\hline
\textbf{Actores} &
Cliente Móvil\newline
Cliente Web\newline
Administrador\newline
Gestor de Eventos\\

\hline
\textbf{Punto de Extensión} &
\\

\hline
\textbf{Extiende} &
\\

\hline
\textbf{Incluye} &
\\

\hline
\textbf{Precondiciones} &
\\

\hline
\textbf{Secuencia Normal} &\mbox{}\par\vspace{-\baselineskip}
\begin{enumerate}[leftmargin=0.7cm, topsep=0.1cm]
\item El usuario introduce sus credenciales en la ventana de login. 
\item El usuario pulsa el botón de \textit{Acceder}.
\item El sistema valida las credenciales.
\item El usuario accede a la aplicación.
\end{enumerate}\\

\hline
\textbf{Secuencia Alternativa} &\mbox{}\par\vspace{-\baselineskip}
\begin{enumerate}[leftmargin=0.9cm, topsep=0.1cm]
\item[3.] Los datos introducidos no son correctos.
	\begin{itemize}
	\item[1.] El sistema muestra un mensaje de error y regresa al \textit{Paso 1}.
	\end{itemize}

\end{enumerate}\\

\hline
\textbf{Postcondiciones} & 
El usuario queda autenticado en el sistema\\
\hline
\end{longtable}




\newpage
\subsubsection*{Caso de uso: Registrarse}
\begin{longtable}{| p{4cm} | p{10cm} |}
\endfirsthead
\multicolumn{2}{c}{\textit{Continúa de la página anterior}}\\[12pt]
\hline
\endhead
\hline
\multicolumn{2}{c}{\textit{Continúa en la siguiente página}} \\
\endfoot
\hline
\caption{Caso de Uso: Registrarse}\label{fig:1}\\
\endlastfoot


\hline
\multicolumn{2}{|c|}{\textbf{CU$<$02$>$ - Registrarse}} \\

\hline
\textbf{Descripción} &
El usuario introduce los datos para darse de alta en la aplicación. \\

\hline
\textbf{Actores} &
Cliente Móvil\\

\hline
\textbf{Punto de Extensión} &
\\

\hline
\textbf{Extiende} &
\\

\hline
\textbf{Incluye} &
\\

\hline
\textbf{Precondiciones} &
\\

\hline
\textbf{Secuencia Normal} &\mbox{}\par\vspace{-\baselineskip}
\begin{enumerate}[leftmargin=0.7cm, topsep=0.1cm]
\item El usuario selecciona la opción de registro.
\item El sistema muestra un formulario indicando los campos necesarios para realizar el registro.
\item El usuario rellena los campos y pulsa el botón de \textit{Registrarse}.
\item El sistema valida los datos introducidos por el usuario.
\item El usuario accede a la aplicación.
\end{enumerate}\\

\hline
\textbf{Secuencia Alternativa} &\mbox{}\par\vspace{-\baselineskip}
\begin{enumerate}[leftmargin=0.7cm, topsep=0.1cm]
\item[3.] El usuario pulsa el botón de \textit{Cancelar}.
	\begin{itemize}
	\item[1.] El sistema cancela el registro y redirige al usuario a la pantalla de login.
	\end{itemize}
\end{enumerate}
\\ &\mbox{}\par\vspace{-\baselineskip}
\begin{enumerate}[leftmargin=0.7cm, topsep=-1cm]
\item[4.] Los datos introducidos por el usuario no son válidos.
	\begin{itemize}
	\item[1.] El sistema muestra un mensaje de error y regresa al \textit{Paso 2}.
	\end{itemize}
\end{enumerate}\\

\hline
\textbf{Postcondiciones} & 
El usuario queda registrado y autenticado en el sistema\\
\hline
\end{longtable}




\newpage
\subsubsection*{Caso de uso: Crear ruta}
\begin{longtable}{| p{4cm} | p{10cm} |}
\endfirsthead
\multicolumn{2}{c}{\textit{Continúa de la página anterior}}\\[12pt]
\hline
\endhead
\hline
\multicolumn{2}{c}{\textit{Continúa en la siguiente página}} \\
\endfoot
\hline
\caption{Caso de Uso: Crear ruta}\label{fig:1}\\
\endlastfoot


\hline
\multicolumn{2}{|c|}{\textbf{CU$<$03$>$ - Crear Ruta}} \\

\hline
\textbf{Descripción} &
El usuario crea una ruta para una ciudad o lugar especificado. \\

\hline
\textbf{Actores} &
Cliente Móvil\\

\hline
\textbf{Punto de Extensión} &
\\

\hline
\textbf{Extiende} &
\\

\hline
\textbf{Incluye} &
\\

\hline
\textbf{Precondiciones} &
El usuario está autenticado en el sistema\\

\hline
\textbf{Secuencia Normal} &\mbox{}\par\vspace{-\baselineskip}
\begin{enumerate}[leftmargin=0.7cm, topsep=0.1cm]
\item El usuario selecciona la opción de crear una nueva ruta.
\item El sistema muestra buscador para que el usuario indique en qué ciudad o lugar desea crear dicha ruta.
\item El usuario rellena el buscador.
\item El sistema ayuda al usuario autocompletando con los datos de diferentes ciudades y lugares.
\item El usuario selecciona el lugar en la lista de autocompletado ofrecida por el sistema.
\item El sistema muestra un mapa indicando la ubicación del lugar seleccionado y permite al usuario completar el proceso de creación.
\item El usuario pulsa el botón para crear la ruta.
\item El sistema ejecuta la acción.
\end{enumerate}\\

\hline
\textbf{Secuencia Alternativa} &\mbox{}\par\vspace{-\baselineskip}
\begin{enumerate}[leftmargin=1.2cm, topsep=0.1cm]
\item[3-5-7.] El usuario pulsa el botón de \textit{Atrás}.
	\begin{itemize}
	\item[1.] El sistema cancela la acción y regresa a la pantalla anterior.
	\end{itemize}
\item[8.] El sistema no puede dar de alta la ruta.
	\begin{itemize}
	\item[1.] El sistema muestra un error indicando que no se pudo realizar la acción.
	\end{itemize}
\end{enumerate}


\\

\hline
\textbf{Postcondiciones} & 
La ruta queda registrada en el sistema\\
\hline
\end{longtable}




\newpage
\subsubsection*{Caso de uso: Explorar rutas}
\begin{longtable}{| p{4cm} | p{10cm} |}
\endfirsthead
\multicolumn{2}{c}{\textit{Continúa de la página anterior}}\\[12pt]
\hline
\endhead
\hline
\multicolumn{2}{c}{\textit{Continúa en la siguiente página}} \\
\endfoot
\hline
\caption{Caso de Uso: Explorar rutas}\label{fig:1}\\
\endlastfoot


\hline
\multicolumn{2}{|c|}{\textbf{CU$<$04$>$ - Explorar Rutas}} \\

\hline
\textbf{Descripción} &
El usuario explora las diferentes rutas creadas por los demás usuarios.\\

\hline
\textbf{Actores} &
Cliente Móvil\newline
Cliente Web\\

\hline
\textbf{Punto de Extensión} &
CU$<$07$>$ - Consultar Ruta
\\

\hline
\textbf{Extiende} &
\\

\hline
\textbf{Incluye} &
\\

\hline
\textbf{Precondiciones} &
El usuario está autenticado en el sistema.\\

\hline
\textbf{Secuencia Normal} &\mbox{}\par\vspace{-\baselineskip}
\begin{enumerate}[leftmargin=0.7cm, topsep=0.1cm]
\item El usuario selecciona la opción de explorar rutas.
\item El sistema obtiene y muestra las rutas públicas de los demás usuarios.
\end{enumerate}\\

\hline
\textbf{Secuencia Alternativa} &\mbox{}\par\vspace{-\baselineskip}
\\

\hline
\textbf{Postcondiciones} & 
\\
\hline
\end{longtable}



\newpage
\subsubsection*{Caso de uso: Obtener rutas propias}
\begin{longtable}{| p{4cm} | p{10cm} |}
\endfirsthead
\multicolumn{2}{c}{\textit{Continúa de la página anterior}}\\[12pt]
\hline
\endhead
\hline
\multicolumn{2}{c}{\textit{Continúa en la siguiente página}} \\
\endfoot
\hline
\caption{Caso de Uso: Obtener rutas propias}\label{fig:1}\\
\endlastfoot


\hline
\multicolumn{2}{|c|}{\textbf{CU$<$05$>$ - Obtener Rutas Propias}} \\

\hline
\textbf{Descripción} &
El usuario obtiene las rutas creadas por él.\\

\hline
\textbf{Actores} &
Cliente Móvil\newline
Cliente Web\\

\hline
\textbf{Punto de Extensión} &
CU$<$07$>$ - Consultar Ruta
\\

\hline
\textbf{Extiende} &
\\

\hline
\textbf{Incluye} &
\\

\hline
\textbf{Precondiciones} &
El usuario está autenticado en el sistema.\\

\hline
\textbf{Secuencia Normal} &\mbox{}\par\vspace{-\baselineskip}
\begin{enumerate}[leftmargin=0.7cm, topsep=0.1cm]
\item El usuario selecciona la opción de obtener rutas propias.
\item El sistema muestra las rutas que el usuario tiene almacenadas en el sistema.
\item El sistema clasifica las rutas del usuario en función de su progreso; las que aún no empezaron, las que están en curso y las que ya se realizaron.
\item El usuario selecciona una de las posibilidades ofrecidas por el sistema.
\item El sistema filtra las rutas del usuario según lo solicitado.
\end{enumerate}\\

\hline
\textbf{Secuencia Alternativa} &\mbox{}\par\vspace{-\baselineskip}
\\

\hline
\textbf{Postcondiciones} & 
\\
\hline
\end{longtable}



\newpage
\subsubsection*{Caso de uso: Eliminar ruta}
\begin{longtable}{| p{4cm} | p{10cm} |}
\endfirsthead
\multicolumn{2}{c}{\textit{Continúa de la página anterior}}\\[12pt]
\hline
\endhead
\hline
\multicolumn{2}{c}{\textit{Continúa en la siguiente página}} \\
\endfoot
\hline
\caption{Caso de Uso: Obtener rutas propias}\label{fig:1}\\
\endlastfoot


\hline
\multicolumn{2}{|c|}{\textbf{CU$<$06$>$ - Eliminar Ruta}} \\

\hline
\textbf{Descripción} &
El usuario elimina una ruta propia del sistema\\

\hline
\textbf{Actores} &
Cliente Móvil\newline
Cliente Web\\

\hline
\textbf{Punto de Extensión} &
CU$<$05$>$ - Obtener Rutas Propias
\\

\hline
\textbf{Extiende} &
\\

\hline
\textbf{Incluye} &
\\

\hline
\textbf{Precondiciones} &
El usuario está autenticado en el sistema.\\

\hline
\textbf{Secuencia Normal} &\mbox{}\par\vspace{-\baselineskip}
\begin{enumerate}[leftmargin=0.7cm, topsep=0.1cm]
\item El usuario selecciona la opción de \textit{Eliminar} sobre una ruta.
\item El sistema muestra un mensaje solicitando al usuario confirmar la acción.
\item El usuario pulsa en el botón de \textit{Confirmar}.
\item El sistema realiza la acción
\end{enumerate}\\

\hline
\textbf{Secuencia Alternativa} &\mbox{}\par\vspace{-\baselineskip}
\begin{enumerate}[leftmargin=0.7cm, topsep=0.1cm]
\item[3.] El usuario pulsa el botón de \textit{Cancelar}.
	\begin{itemize}
	\item[1.] El sistema cancela la eliminación de la ruta.
	\end{itemize}
\item[4.] El usuario no tiene permisos necesarios para realizar determinada acción.
	\begin{itemize}
	\item[1.] El sistema muestra un mensaje de error indicando que no está autorizado a realizar dicha acción.
	\end{itemize}
\end{enumerate}
\\ &\mbox{}\par\vspace{-\baselineskip}
\begin{enumerate}[leftmargin=0.7cm, topsep=0.1cm]
\item[4.] La ruta que quiere eliminar no se encuentra en el sistema
	\begin{itemize}
	\item[1.] El sistema muestra un mensaje de error indicando que no es posible realizar la acción.
	\end{itemize}
\end{enumerate}

\\

\hline
\textbf{Postcondiciones} & 
La ruta queda eliminada del sistema\\
\hline
\end{longtable}


\newpage
\subsubsection*{Caso de uso: Consultar ruta}
\begin{longtable}{| p{4cm} | p{10cm} |}
\endfirsthead
\multicolumn{2}{c}{\textit{Continúa de la página anterior}}\\[12pt]
\hline
\endhead
\hline
\multicolumn{2}{c}{\textit{Continúa en la siguiente página}} \\
\endfoot
\hline
\caption{Caso de Uso: Consultar ruta}\label{fig:1}\\
\endlastfoot


\hline
\multicolumn{2}{|c|}{\textbf{CU$<$07$>$ - Consultar Ruta}} \\

\hline
\textbf{Descripción} &
El usuario obtiene la información detallada de una ruta concreta.\\

\hline
\textbf{Actores} &
Cliente Móvil\newline
Cliente Web\\

\hline
\textbf{Punto de Extensión} &
CU$<$08$>$ - Modificar Fechas de Viaje\newline
CU$<$09$>$ - Modificar Privacidad\newline
CU$<$10$>$ - Mostrar Ruta en Mapa\newline
CU$<$11$>$ - Consultar Eventos\newline
CU$<$14$>$ - Consultar Itinerario Ruta
\\

\hline
\textbf{Extiende} &
CU$<$04$>$ - Explorar Rutas\newline
CU$<$05$>$ - Obtener Rutas Propias
\\

\hline
\textbf{Incluye} &
\\

\hline
\textbf{Precondiciones} &
El usuario está autenticado en el sistema.\\

\hline
\textbf{Secuencia Normal} &\mbox{}\par\vspace{-\baselineskip}
\begin{enumerate}[leftmargin=0.7cm, topsep=0.1cm]
\item El usuario selecciona una ruta concreta.
\item El sistema obtiene y muestra un panel con los datos detallados de la ruta.
\end{enumerate}\\

\hline
\textbf{Secuencia Alternativa} &\mbox{}\par\vspace{-\baselineskip}
\begin{enumerate}[leftmargin=0.7cm, topsep=0.1cm]
\item[2.] La ruta solicitada no se encuentra en el sistema.
	\begin{itemize}
	\item[1.] El sistema muestra un mensaje de error indicando que no es posible acceder a la ruta solicitada.
	\end{itemize}
\end{enumerate}
\\ &\mbox{}\par\vspace{-\baselineskip}	
\begin{enumerate}[leftmargin=0.7cm, topsep=0.1cm]
\item[2.] El usuario no tiene permisos necesarios para realizar determinada acción.
	\begin{itemize}
	\item[1.] El sistema muestra un mensaje de error indicando que el usuario no está autorizado a realizar dicha acción.
	\end{itemize}
\end{enumerate}
\\

\hline
\textbf{Postcondiciones} & 
\\
\hline
\end{longtable}



\newpage
\subsubsection*{Caso de uso: Modificar fechas de viaje}
\begin{longtable}{| p{4cm} | p{10cm} |}
\endfirsthead
\multicolumn{2}{c}{\textit{Continúa de la página anterior}}\\[12pt]
\hline
\endhead
\hline
\multicolumn{2}{c}{\textit{Continúa en la siguiente página}} \\
\endfoot
\hline
\caption{Caso de Uso: Modificar fechas de viaje}\label{fig:1}\\
\endlastfoot


\hline
\multicolumn{2}{|c|}{\textbf{CU$<$08$>$ - Modificar Fechas de Viaje}} \\

\hline
\textbf{Descripción} &
El usuario selecciona las fechas en las que se realizará la ruta que está planificando.\\

\hline
\textbf{Actores} &
Cliente Móvil\\


\hline
\textbf{Punto de Extensión} &
\\

\hline
\textbf{Extiende} &
CU$<$07$>$ - Consultar Ruta
\\

\hline
\textbf{Incluye} &
\\

\hline
\textbf{Precondiciones} &
El usuario está autenticado en el sistema.\\

\hline
\textbf{Secuencia Normal} &\mbox{}\par\vspace{-\baselineskip}
\begin{enumerate}[leftmargin=0.7cm, topsep=0.1cm]
\item El usuario selecciona la opción de \textit{Seleccionar Fechas}.
\item El sistema muestra una pantalla con las fechas del calendario.
\item El usuario selecciona la fecha de inicio y fin y pulsa el botón de \textit{Aceptar}.
\item El sistema modifica las fechas de la ruta.
\end{enumerate}\\

\hline
\textbf{Secuencia Alternativa} &\mbox{}\par\vspace{-\baselineskip}
\begin{enumerate}[leftmargin=0.7cm, topsep=0.1cm]
\item[3.] El usuario pulsa el botón de \textit{Cancelar}.
	\begin{itemize}
	\item[1.] El sistema cancela la modificación de las fechas y cierra la pantalla.
	\end{itemize}
\item[4.] El usuario no tiene permisos necesarios para realizar determinada acción.
	\begin{itemize}
	\item[1.] El sistema muestra un mensaje de error indicando que no está autorizado a realizar dicha acción.
	\end{itemize}
\end{enumerate}
\\ &\mbox{}\par\vspace{-\baselineskip}	
\begin{enumerate}[leftmargin=0.7cm, topsep=0.1cm]
\item[4.] La ruta que quiere actualizar no se encuentra en el sistema
	\begin{itemize}
	\item[1.] El sistema muestra un mensaje de error indicando que no es posible realizar la acción.
	\end{itemize}
\end{enumerate}
\\

\hline
\textbf{Postcondiciones} & 
Las fechas de la ruta quedan actualizadas en el sistema.\\
\hline
\end{longtable}



\newpage
\subsubsection*{Caso de uso: Modificar privacidad}
\begin{longtable}{| p{4cm} | p{10cm} |}
\endfirsthead
\multicolumn{2}{c}{\textit{Continúa de la página anterior}}\\[12pt]
\hline
\endhead
\hline
\multicolumn{2}{c}{\textit{Continúa en la siguiente página}} \\
\endfoot
\hline
\caption{Caso de Uso: Modificar privacidad}\label{fig:1}\\
\endlastfoot


\hline
\multicolumn{2}{|c|}{\textbf{CU$<$09$>$ - Modificar Privacidad}} \\

\hline
\textbf{Descripción} &
El usuario puede alternar la privacidad de cada una de sus rutas permitiendo que sean visible para todos o solo para él.\\

\hline
\textbf{Actores} &
Cliente Móvil\\

\hline
\textbf{Punto de Extensión} &
\\

\hline
\textbf{Extiende} &
CU$<$07$>$ - Consultar Ruta
\\

\hline
\textbf{Incluye} &
\\

\hline
\textbf{Precondiciones} &
El usuario está autenticado en el sistema.\\

\hline
\textbf{Secuencia Normal} &\mbox{}\par\vspace{-\baselineskip}
\begin{enumerate}[leftmargin=0.7cm, topsep=0.1cm]
\item El usuario selecciona la opción de \textit{Detalles de la Ruta}.
\item El sistema muestra una pantalla con los detalles de la ruta.
\item El sistema incluye un botón que permite alternar entre los estados de privacidad (\textit{Privado} y \textit{Público}).
\item El usuario pulsa el botón.
\item Si la privacidad tenía el valor \textit{Privado}:
	\begin{itemize}
	\item[1.] El sistema actualiza la privacidad a \textit{Pública}.
	\end{itemize}
\item Si la privacidad tenía el valor \textit{Público}:
	\begin{itemize}
	\item[1.] El sistema actualiza la privacidad a \textit{Privado}.
	\end{itemize}
\end{enumerate}\\

\hline
\textbf{Secuencia Alternativa} &\mbox{}\par\vspace{-\baselineskip}
\begin{enumerate}[leftmargin=0.9cm, topsep=0.1cm]
\item[4.] El usuario pulsa el botón de \textit{Atrás}.
	\begin{itemize}
	\item[1.] El sistema retorna a la pantalla anterior.
	\end{itemize}
\item[5-6.] El usuario no tiene permisos necesarios para realizar determinada acción.
	\begin{itemize}
	\item[1.] El sistema muestra un mensaje de error indicando que no está autorizado a realizar dicha acción.
	\end{itemize}
\item[5-6.] La ruta que quiere actualizar no se encuentra en el sistema
	\begin{itemize}
	\item[1.] El sistema muestra un mensaje de error indicando que no es posible realizar la acción.
	\end{itemize}
\end{enumerate}
\\

\hline
\textbf{Postcondiciones} & 
La privacidad de la ruta queda actualizada.\\
\hline
\end{longtable}



\newpage
\subsubsection*{Caso de uso: Mostrar ruta en mapa}
\begin{longtable}{| p{4cm} | p{10cm} |}
\endfirsthead
\multicolumn{2}{c}{\textit{Continúa de la página anterior}}\\[12pt]
\hline
\endhead
\hline
\multicolumn{2}{c}{\textit{Continúa en la siguiente página}} \\
\endfoot
\hline
\caption{Caso de Uso: Mostrar ruta en mapa}\label{fig:1}\\
\endlastfoot


\hline
\multicolumn{2}{|c|}{\textbf{CU$<$10$>$ - Mostrar Ruta en Mapa}} \\

\hline
\textbf{Descripción} &
El usuario puede mostrar la información de la ruta en el mapa.\\

\hline
\textbf{Actores} &
Cliente Móvil\newline
Cliente Web\\

\hline
\textbf{Punto de Extensión} &
\\

\hline
\textbf{Extiende} &
CU$<$07$>$ - Consultar Ruta
\\

\hline
\textbf{Incluye} &
\\

\hline
\textbf{Precondiciones} &
El usuario está autenticado en el sistema.\\

\hline
\textbf{Secuencia Normal} &\mbox{}\par\vspace{-\baselineskip}
\begin{enumerate}[leftmargin=0.7cm, topsep=0.1cm]
\item El usuario selecciona la opción de \textit{Mostrar Ruta en Mapa}.
\item Para cada día de la ruta, el sistema muestra un mapa con las marcas de las visitas asignadas al día concreto.
\item Si la ruta ya ha transcurrido en el tiempo y tiene información de su ejecución en tiempo real.
	\begin{itemize}
	\item[1.] El sistema incorpora al mapa la ruta real realizada por el usuario permitiendo compararla con la ruta planificada.
	\end{itemize}
\item El usuario puede cambiar entre los días haciendo click sobre ellos.
\end{enumerate}\\

\hline
\textbf{Secuencia Alternativa} &\mbox{}\par\vspace{-\baselineskip}
\begin{enumerate}[leftmargin=0.9cm, topsep=0.1cm]
\item[3.] El usuario pulsa el botón de \textit{Atrás}.
	\begin{itemize}
	\item[1.] El sistema retorna a la pantalla anterior.
	\end{itemize}
\end{enumerate}
\\

\hline
\textbf{Postcondiciones} & \\
\hline
\end{longtable}



\newpage
\subsubsection*{Caso de uso: Consultar eventos}
\begin{longtable}{| p{4cm} | p{10cm} |}
\endfirsthead
\multicolumn{2}{c}{\textit{Continúa de la página anterior}}\\[12pt]
\hline
\endhead
\hline
\multicolumn{2}{c}{\textit{Continúa en la siguiente página}} \\
\endfoot
\hline
\caption{Caso de Uso: Cosnultar eventos}\label{fig:1}\\
\endlastfoot


\hline
\multicolumn{2}{|c|}{\textbf{CU$<$11$>$ - Consultar Eventos}} \\

\hline
\textbf{Descripción} &
El usuario obtiene los eventos dados de alta en el sistema\\

\hline
\textbf{Actores} &
Cliente Móvil\\

\hline
\textbf{Punto de Extensión} &
CU$<$12$>$ - Mostrar Evento en Mapa
CU$<$13$>$ - Asignar Evento a Ruta
\\

\hline
\textbf{Extiende} &
CU$<$07$>$ - Consultar Ruta
\\

\hline
\textbf{Incluye} &
\\

\hline
\textbf{Precondiciones} &
El usuario está autenticado en el sistema.\\

\hline
\textbf{Secuencia Normal} &\mbox{}\par\vspace{-\baselineskip}
\begin{enumerate}[leftmargin=0.7cm, topsep=0.1cm]
\item El usuario selecciona la opción de \textit{Consultar Eventos}.
\item El sistema muestra una pantalla con las opciones: \textit{En el viaje} y \textit{Próximos}.
\item Si el usuario selecciona \textit{En el viaje}.
	\begin{itemize}
	\item[1.] El sistema muestra los eventos que coinciden durante las fechas de viaje del usuario.
	\end{itemize}
\item Si el usuario selecciona \textit{Próximos}.
	\begin{itemize}
	\item[1.] El sistema muestra los eventos futuros a las fechas de viaje del usuario.
	\end{itemize}
\end{enumerate}\\

\hline
\textbf{Secuencia Alternativa} &\mbox{}\par\vspace{-\baselineskip}
\begin{enumerate}[leftmargin=0.9cm, topsep=0.1cm]
\item[3-4.] El usuario pulsa el botón de \textit{Atrás}.
	\begin{itemize}
	\item[1.] El sistema retorna a la pantalla anterior.
	\end{itemize}
\end{enumerate}
\\

\hline
\textbf{Postcondiciones} & \\
\hline
\end{longtable}



\newpage
\subsubsection*{Caso de uso: Mostrar evento en mapa }
\begin{longtable}{| p{4cm} | p{10cm} |}
\endfirsthead
\multicolumn{2}{c}{\textit{Continúa de la página anterior}}\\[12pt]
\hline
\endhead
\hline
\multicolumn{2}{c}{\textit{Continúa en la siguiente página}} \\
\endfoot
\hline
\caption{Caso de Uso: Mostrar evento en mapa}\label{fig:1}\\
\endlastfoot


\hline
\multicolumn{2}{|c|}{\textbf{CU$<$12$>$ - Mostrar Evento en Mapa}} \\

\hline
\textbf{Descripción} &
Muestra un evento concreto en el mapa.\\

\hline
\textbf{Actores} &
Cliente Móvil\\

\hline
\textbf{Punto de Extensión} &
\\

\hline
\textbf{Extiende} &
CU$<$11$>$ - Consultar Eventos
\\

\hline
\textbf{Incluye} &
\\

\hline
\textbf{Precondiciones} &
El usuario está autenticado en el sistema.\\

\hline
\textbf{Secuencia Normal} &\mbox{}\par\vspace{-\baselineskip}
\begin{enumerate}[leftmargin=0.7cm, topsep=0.1cm]
\item El usuario selecciona la opción de \textit{Mostrar Evento en Mapa}.
\item El sistema muestra un mapa indicando la ubicación del evento incluyendo la ubicación de los lugares asignados al día de la ruta que transcurre el mismo día del evento, permitiendo ubicar el evento en función de la ruta ya creada.
\end{enumerate}
\\
\hline
\textbf{Secuencia Alternativa} &\mbox{}\par\vspace{-\baselineskip}
\begin{enumerate}[leftmargin=0.9cm, topsep=0.1cm]
\item[2.] El usuario pulsa el botón de \textit{Atrás}.
	\begin{itemize}
	\item[1.] El sistema retorna a la pantalla anterior.
	\end{itemize}
\end{enumerate}
\\

\hline
\textbf{Postcondiciones} & \\
\hline
\end{longtable}



\newpage
\subsubsection*{Caso de uso: Asignar evento a ruta }
\begin{longtable}{| p{4cm} | p{10cm} |}
\endfirsthead
\multicolumn{2}{c}{\textit{Continúa de la página anterior}}\\[12pt]
\hline
\endhead
\hline
\multicolumn{2}{c}{\textit{Continúa en la siguiente página}} \\
\endfoot
\hline
\caption{Caso de Uso: Asignar evento a ruta}\label{fig:1}\\
\endlastfoot


\hline
\multicolumn{2}{|c|}{\textbf{CU$<$13$>$ - Asignar Evento a ruta}} \\

\hline
\textbf{Descripción} &
Permite incorporar o eliminar eventos a la ruta.\\

\hline
\textbf{Actores} &
Cliente Móvil\\

\hline
\textbf{Punto de Extensión} &
\\

\hline
\textbf{Extiende} &
CU$<$11$>$ - Consultar Eventos
\\

\hline
\textbf{Incluye} &
\\

\hline
\textbf{Precondiciones} &
El usuario está autenticado en el sistema.\\

\hline
\textbf{Secuencia Normal} &\mbox{}\par\vspace{-\baselineskip}
\begin{enumerate}[leftmargin=0.7cm, topsep=0.1cm]
\item Si el evento no está asignado a la ruta.
	\begin{itemize}
	\item[1.] El sistema muestra un botón para añadirlo en la ruta del usuario.
	\end{itemize}
\item Si el evento ya está asignado a la ruta.
	\begin{itemize}
	\item[1.] El sistema muestra un botón para eliminarlo de la ruta.
	\end{itemize}
\item El usuario hace click en el botón correspondiente.
\item El sistema muestra un una ventana de confirmación.
\item El usuario pulsa en \textit{Aceptar}
\item El sistema incorpora o elimina el evento al día correspondiente.
\end{enumerate}


\\
\hline
\textbf{Secuencia Alternativa} &\mbox{}\par\vspace{-\baselineskip}
\begin{enumerate}[leftmargin=0.7cm, topsep=0.1cm]
\item[2.] El usuario pulsa el botón de \textit{Atrás}.
	\begin{itemize}
	\item[1.] El sistema retorna a la pantalla anterior.
	\end{itemize}
\end{enumerate}
\\ &\mbox{}\par\vspace{-\baselineskip}	
\begin{enumerate}[leftmargin=0.7cm, topsep=0.1cm]
\item[5.] El usuario pulso en \textit{Cancelar} en la ventana de confirmación.
	\begin{itemize}
	\item[1.] El sistema cancela la confirmación y retorna al \textit{Paso 1}
	\end{itemize}
\item[6.] El usuario no tiene permisos necesarios para realizar determinada acción.
	\begin{itemize}
	\item[1.] El sistema muestra un mensaje de error indicando que no está autorizado a realizar dicha acción.
	\end{itemize}
\item[6.] La ruta a la que quiere asignar el evento no se encuentra en el sistema
	\begin{itemize}
	\item[1.] El sistema muestra un mensaje de error indicando que no es posible realizar la acción.
	\end{itemize}
\end{enumerate}
\\

\hline
\textbf{Postcondiciones} & 
El evento queda asignado como visita en la ruta.\\
\hline
\end{longtable}



\newpage
\subsubsection*{Caso de uso: Consultar itinerario ruta }
\begin{longtable}{| p{4cm} | p{10cm} |}
\endfirsthead
\multicolumn{2}{c}{\textit{Continúa de la página anterior}}\\[12pt]
\hline
\endhead
\hline
\multicolumn{2}{c}{\textit{Continúa en la siguiente página}} \\
\endfoot
\hline
\caption{Caso de Uso: Consultar itinerario ruta}\label{fig:1}\\
\endlastfoot


\hline
\multicolumn{2}{|c|}{\textbf{CU$<$14$>$ - Consultar Itinerario Ruta}} \\

\hline
\textbf{Descripción} &
Obtiene la información de la ruta desglosada por los días que la componen.\\

\hline
\textbf{Actores} &
Cliente Móvil\newline
Cliente Web\\

\hline
\textbf{Punto de Extensión} &
CU$<$15$>$ - Editar Orden Visitas\newline
CU$<$16$>$ - Modificar Hora Salida\newline
CU$<$17$>$ - Eliminar Visita\newline
CU$<$18$>$ - Modificar Tiempo Visita\newline
CU$<$19$>$ - Modificar Modo de Viaje\newline
CU$<$21$>$ - Asignar Lugares a Ruta
\\

\hline
\textbf{Extiende} &
CU$<$07$>$ - Consultar Ruta
\\

\hline
\textbf{Incluye} &
\\

\hline
\textbf{Precondiciones} &
El usuario está autenticado en el sistema.\newline
La ruta tiene unas fechas de viaje asignadas.\\

\hline
\textbf{Secuencia Normal} &\mbox{}\par\vspace{-\baselineskip}
\begin{enumerate}[leftmargin=0.7cm, topsep=0.1cm]
\item El usuario selecciona la opción \textit{Itinerario}.
\item El sistema muestra en la parte superior un selector con los días y en la inferior, la información correspondiente a cada día. Para cada día, el sistema muestra:
	\begin{itemize}
	\item [1.] El conjunto de visitas a lugares y/o eventos asignadas por el usuario.
	\item [2.] La hora de comienzo de la ruta para el día determinado.
	\item [3.] La hora de llegada, estimada, a cada visita.
	\item [4.] El tiempo asignado, como parada, en cada visita.
	\item [5.] La hora de salida, estimada, para cada visita.
	\item [6.] El modo de viaje, junto a su duración y distancia, entre cada visita.
	\item [7.] Un botón para eliminar cada visita.
	\end{itemize}
\item El usuario cambia de día haciendo uso del selector.
\item El sistema actualiza la pantalla con los datos del día seleccionado.
\end{enumerate}


\\
\hline
\textbf{Secuencia Alternativa} &\mbox{}\par\vspace{-\baselineskip}
\begin{enumerate}[leftmargin=0.9cm, topsep=0.1cm]
\item[3.] El usuario pulsa el botón de \textit{Atrás}.
	\begin{itemize}
	\item[1.] El sistema retorna a la pantalla anterior.
	\end{itemize}
\end{enumerate}
\\

\hline
\textbf{Postcondiciones} & \\
\hline
\end{longtable}



\newpage
\subsubsection*{Caso de uso: Editar orden visitas }
\begin{longtable}{| p{4cm} | p{10cm} |}
\endfirsthead
\multicolumn{2}{c}{\textit{Continúa de la página anterior}}\\[12pt]
\hline
\endhead
\hline
\multicolumn{2}{c}{\textit{Continúa en la siguiente página}} \\
\endfoot
\hline
\caption{Caso de Uso: Editar orden visitas}\label{fig:1}\\
\endlastfoot


\hline
\multicolumn{2}{|c|}{\textbf{CU$<$15$>$ - Editar Orden Visitas}} \\

\hline
\textbf{Descripción} &
Modifica el orden de las visitas establecidas para un día.\\

\hline
\textbf{Actores} &
Cliente Móvil\\

\hline
\textbf{Punto de Extensión} &
\\

\hline
\textbf{Extiende} &
CU$<$14$>$ - Consultar Itinerario Ruta
\\

\hline
\textbf{Incluye} &
\\

\hline
\textbf{Precondiciones} &
La ruta tiene visitas asignadas a sus días\newline
El usuario está autenticado en el sistema.\\

\hline
\textbf{Secuencia Normal} &\mbox{}\par\vspace{-\baselineskip}
\begin{enumerate}[leftmargin=0.7cm, topsep=0.1cm]
\item El usuario selecciona la opción \textit{Habilitar Edición}.
\item El sistema muestra la vista de edición para las visitas.
\item El usuario modifica el orden de las visitas.
\item El sistema actualiza el orden establecido y recalcula las distancias y tiempos de viaje para el nuevo orden de las visitas.
\item El usuario selecciona \textit{Terminar Edición}.
\item El sistema vuelve a la vista normal.
\end{enumerate}


\\
\hline
\textbf{Secuencia Alternativa} &\mbox{}\par\vspace{-\baselineskip}
\begin{enumerate}[leftmargin=0.9cm, topsep=0.1cm]
\item[3-5.] El usuario pulsa el botón de \textit{Atrás}.
	\begin{itemize}
	\item[1.] El sistema retorna a la pantalla anterior.
	\end{itemize}
\end{enumerate}
\\ &\mbox{}\par\vspace{-\baselineskip}	
\begin{enumerate}[leftmargin=0.7cm, topsep=0.1cm]
\item[4.] El usuario no tiene permisos necesarios para realizar determinada acción.
	\begin{itemize}
	\item[1.] El sistema muestra un mensaje de error indicando que no está autorizado a realizar dicha acción.
	\end{itemize}
\item[4.] La visitas que quiere actualizar no se encuentra en el sistema
	\begin{itemize}
	\item[1.] El sistema muestra un mensaje de error indicando que no es posible realizar la acción.
	\end{itemize}
\end{enumerate}
\\

\hline
\textbf{Postcondiciones} & 
El nuevo orden de las visitas queda actualizado en el sistema.\\
\hline
\end{longtable}



\newpage
\subsubsection*{Caso de uso: Modificar hora de salida }
\begin{longtable}{| p{4cm} | p{10cm} |}
\endfirsthead
\multicolumn{2}{c}{\textit{Continúa de la página anterior}}\\[12pt]
\hline
\endhead
\hline
\multicolumn{2}{c}{\textit{Continúa en la siguiente página}} \\
\endfoot
\hline
\caption{Caso de Uso: Modificar hora de salida}\label{fig:1}\\
\endlastfoot


\hline
\multicolumn{2}{|c|}{\textbf{CU$<$16$>$ - Modificar Hora de Salida}} \\

\hline
\textbf{Descripción} &
Modifica la hora de salida para un día concreto de la ruta.\\

\hline
\textbf{Actores} &
Cliente Móvil\\

\hline
\textbf{Punto de Extensión} &
\\

\hline
\textbf{Extiende} &
CU$<$14$>$ - Consultar Itinerario Ruta
\\

\hline
\textbf{Incluye} &
\\

\hline
\textbf{Precondiciones} &
El usuario está autenticado en el sistema.\\

\hline
\textbf{Secuencia Normal} &\mbox{}\par\vspace{-\baselineskip}
\begin{enumerate}[leftmargin=0.7cm, topsep=0.1cm]
\item El usuario selecciona la opción \textit{Modificar hora de salida}.
\item El sistema muestra un formulario para indicar la hora deseada.
\item El usuario selecciona la hora en el formulario y pulsa en \textit{Confirmar}.
\item El sistema valida la hora y la actualiza.
\end{enumerate}


\\
\hline
\textbf{Secuencia Alternativa} &\mbox{}\par\vspace{-\baselineskip}
\begin{enumerate}[leftmargin=0.7cm, topsep=0.1cm]
\item[3.] El usuario pulsa el botón de \textit{Cancelar}.
	\begin{itemize}
	\item[1.] El sistema deshecha el formulario y cancela la acción.
	\end{itemize}
\item[4.] El usuario no tiene permisos necesarios para realizar determinada acción.
	\begin{itemize}
	\item[1.] El sistema muestra un mensaje de error indicando que no está autorizado a realizar dicha acción.
	\end{itemize}
\end{enumerate}
\\ &\mbox{}\par\vspace{-\baselineskip}	
\begin{enumerate}[leftmargin=0.7cm, topsep=0.1cm]
\item[4.] La ruta sobre la que quiere actualizar la hora de salida ya no se encuentra en el sistema.
	\begin{itemize}
	\item[1.] El sistema muestra un mensaje de error indicando que no es posible realizar la acción.
	\end{itemize}
\end{enumerate}
\\

\hline
\textbf{Postcondiciones} & 
La hora de salida queda actualizada en el sistema.\\
\hline
\end{longtable}



\newpage
\subsubsection*{Caso de uso: Eliminar visita }
\begin{longtable}{| p{4cm} | p{10cm} |}
\endfirsthead
\multicolumn{2}{c}{\textit{Continúa de la página anterior}}\\[12pt]
\hline
\endhead
\hline
\multicolumn{2}{c}{\textit{Continúa en la siguiente página}} \\
\endfoot
\hline
\caption{Caso de Uso: Eliminar visita}\label{fig:1}\\
\endlastfoot


\hline
\multicolumn{2}{|c|}{\textbf{CU$<$17$>$ - Eliminar Visita}} \\

\hline
\textbf{Descripción} &
El usuario elimina una visita de un día de la ruta.\\

\hline
\textbf{Actores} &
Cliente Móvil\\

\hline
\textbf{Punto de Extensión} &
\\

\hline
\textbf{Extiende} &
CU$<$14$>$ - Consultar Itinerario Ruta
\\

\hline
\textbf{Incluye} &
\\

\hline
\textbf{Precondiciones} &
El usuario está autenticado en el sistema.\\

\hline
\textbf{Secuencia Normal} &\mbox{}\par\vspace{-\baselineskip}
\begin{enumerate}[leftmargin=0.7cm, topsep=0.1cm]
\item El usuario hace uso del botón para eliminar la visita.
\item El sistema muestra una pantalla de confirmación.
\item El usuario confirma la acción.
\item El sistema elimina la visita.
\end{enumerate}


\\
\hline
\textbf{Secuencia Alternativa} &\mbox{}\par\vspace{-\baselineskip}
\begin{enumerate}[leftmargin=0.7cm, topsep=0.1cm]
\item[3.] El usuario pulsa el botón de \textit{Cancelar}.
	\begin{itemize}
	\item[1.] El sistema cancela la eliminación de la visita.
	\end{itemize}
\item[4.] El usuario no tiene permisos necesarios para realizar determinada acción.
	\begin{itemize}
	\item[1.] El sistema muestra un mensaje de error indicando que no está autorizado a realizar dicha acción.
	\end{itemize}
\end{enumerate}
\\ &\mbox{}\par\vspace{-\baselineskip}	
\begin{enumerate}[leftmargin=0.7cm, topsep=0.1cm]
\item[4.] La visita que quiere eliminar no se encuentra en el sistema
	\begin{itemize}
	\item[1.] El sistema muestra un mensaje de error indicando que no es posible realizar la acción.
	\end{itemize}
\end{enumerate}
\\

\hline
\textbf{Postcondiciones} & 
La visita queda eliminada del sistema.\\
\hline
\end{longtable}



\newpage
\subsubsection*{Caso de uso: Modificar tiempo en la visita }
\begin{longtable}{| p{4cm} | p{10cm} |}
\endfirsthead
\multicolumn{2}{c}{\textit{Continúa de la página anterior}}\\[12pt]
\hline
\endhead
\hline
\multicolumn{2}{c}{\textit{Continúa en la siguiente página}} \\
\endfoot
\hline
\caption{Caso de Uso: Modificar tiempo en la visita}\label{fig:1}\\
\endlastfoot


\hline
\multicolumn{2}{|c|}{\textbf{CU$<$18$>$ - Modificar Tiempo en la Visita}} \\

\hline
\textbf{Descripción} &
El usuario modifica el tiempo de parada en una determinada visita.\\

\hline
\textbf{Actores} &
Cliente Móvil\\

\hline
\textbf{Punto de Extensión} &
\\

\hline
\textbf{Extiende} &
CU$<$14$>$ - Consultar Itinerario Ruta
\\

\hline
\textbf{Incluye} &
\\

\hline
\textbf{Precondiciones} &
La ruta tiene visitas asignadas a sus días\newline
El usuario está autenticado en el sistema.\\

\hline
\textbf{Secuencia Normal} &\mbox{}\par\vspace{-\baselineskip}
\begin{enumerate}[leftmargin=0.7cm, topsep=0.1cm]
\item El usuario selecciona la opción \textit{Editar Tiempo Visita}.
\item El sistema muestra un formulario para indicar el tiempo deseado.
\item El usuario selecciona indica el tiempo en el formulario y pulsa en \textit{Confirmar}.
\item El sistema actualiza el tiempo para la visita.

\end{enumerate}


\\
\hline
\textbf{Secuencia Alternativa} &\mbox{}\par\vspace{-\baselineskip}
\begin{enumerate}[leftmargin=0.7cm, topsep=0.1cm]
\item[3.] El usuario pulsa el botón de \textit{Cancelar}.
	\begin{itemize}
	\item[1.] El sistema cancela la operación.
	\end{itemize}
\item[4.] El usuario no tiene permisos necesarios para realizar determinada acción.
	\begin{itemize}
	\item[1.] El sistema muestra un mensaje de error indicando que no está autorizado a realizar dicha acción.
	\end{itemize}
\end{enumerate}
\\ &\mbox{}\par\vspace{-\baselineskip}	
\begin{enumerate}[leftmargin=0.7cm, topsep=0.1cm]
\item[4.] La visita que quiere actualizar no se encuentra en el sistema.
	\begin{itemize}
	\item[1.] El sistema muestra un mensaje de error indicando que no es posible realizar la acción.
	\end{itemize}
\end{enumerate}
\\

\hline
\textbf{Postcondiciones} & 
El tiempo de la visita queda actualizado en el sistema.\\
\hline
\end{longtable}



\newpage
\subsubsection*{Caso de uso: Modificar modo de viaje }
\begin{longtable}{| p{4cm} | p{10cm} |}
\endfirsthead
\multicolumn{2}{c}{\textit{Continúa de la página anterior}}\\[12pt]
\hline
\endhead
\hline
\multicolumn{2}{c}{\textit{Continúa en la siguiente página}} \\
\endfoot
\hline
\caption{Caso de Uso: Modificar modo de viaje}\label{fig:1}\\
\endlastfoot


\hline
\multicolumn{2}{|c|}{\textbf{CU$<$19$>$ - Modificar Modo de Viaje}} \\

\hline
\textbf{Descripción} &
El usuario modifica el modo de viaje entre dos visitas. Los modos de viaje habilitados son: \textit{En coche}, \textit{En bicicleta} y \textit{Andando}\\

\hline
\textbf{Actores} &
Cliente Móvil\\

\hline
\textbf{Punto de Extensión} &
\\

\hline
\textbf{Extiende} &
CU$<$14$>$ - Consultar Itinerario Ruta
\\

\hline
\textbf{Incluye} &
\\

\hline
\textbf{Precondiciones} &
La ruta tiene al menos dos visitas para un día concreto\newline
El usuario está autenticado en el sistema.\\

\hline
\textbf{Secuencia Normal} &\mbox{}\par\vspace{-\baselineskip}
\begin{enumerate}[leftmargin=0.7cm, topsep=0.1cm]
\item El usuario selecciona la opción \textit{Modificar Modo de Viaje}.
\item Si el modo de viaje era \textit{Andando}.
	\begin{itemize}
	\item [1.] El sistema actualiza el modo de viaje a \textit{En coche}.
	\end{itemize}
\item Si el modo de viaje era \textit{En coche}.
	\begin{itemize}
	\item [1.] El sistema actualiza el modo de viaje a \textit{En bicicleta}.
	\end{itemize}
\item Si el modo de viaje era \textit{En bicicleta}.
	\begin{itemize}
	\item [1.] El sistema actualiza el modo de viaje a \textit{Andando}.
	\end{itemize}

\end{enumerate}


\\
\hline
\textbf{Secuencia Alternativa} &\mbox{}\par\vspace{-\baselineskip}

\begin{enumerate}[leftmargin=1.2cm, topsep=0.1cm]
\item[2-3-4.] El usuario no tiene permisos necesarios para realizar determinada acción.
	\begin{itemize}
	\item[1.] El sistema muestra un mensaje de error indicando que no está autorizado a realizar dicha acción.
	\end{itemize}
\item[2-3-4.] La visitas en las que quiere actualizar el modo de viaje no se encuentra en el sistema.
	\begin{itemize}
	\item[1.] El sistema muestra un mensaje de error indicando que no es posible realizar la acción.
	\end{itemize}
\end{enumerate}

\\
\hline
\textbf{Postcondiciones} & 
El modo de viaje queda actualizado en el sistema.\\
\hline
\end{longtable}



\newpage
\subsubsection*{Caso de uso: Consultar lugares }
\begin{longtable}{| p{4cm} | p{10cm} |}

\endfirsthead
\multicolumn{2}{c}{\textit{Continúa de la página anterior}}\\[12pt]
\hline
\endhead
\hline
\multicolumn{2}{c}{\textit{Continúa en la siguiente página}} \\
\endfoot
\hline
\caption{Caso de Uso: Modificar modo de viaje}\label{fig:1}\\
\endlastfoot
\hline
\caption{Caso de Uso: Consultar lugares}\label{fig:1}\\
\endlastfoot


\hline
\multicolumn{2}{|c|}{\textbf{CU$<$20$>$ - Consultar lugares}} \\

\hline
\textbf{Descripción} &
El usuario consulta los diferentes lugares en función de diferentes criterios.\\

\hline
\textbf{Actores} &
Cliente Móvil\\

\hline
\textbf{Punto de Extensión} &
\\

\hline
\textbf{Extiende} &
\\

\hline
\textbf{Incluye} &
\\

\hline
\textbf{Precondiciones} &
El usuario está autenticado en el sistema.\\

\hline
\textbf{Secuencia Normal} &\mbox{}\par\vspace{-\baselineskip}
\begin{enumerate}[leftmargin=0.7cm, topsep=0.1cm]
\item El sistema muestra en la parte superior un selector y en la inferior los lugares más relevantes para la ciudad o lugar donde está creada la ruta.
\item Si el usuario selecciona la opción \textit{Lista} en el selector.
	\begin{itemize}
	\item[1.] El sistema muestra los lugares en una lista.
	\item[2.] Para cada lugar el sistema incluye.
		\begin{itemize}
		\item[1.] Foto, nombre, categoría a la que pertenece y dirección en la que se encuentra.
		\item[2.] Un indicador con el número de días en el que está asignado dicho lugar en la ruta del usuario.
		\item[3.] Un botón para asignar o desasignar el lugar a los días de la ruta.
		\end{itemize}
	\end{itemize}
\end{enumerate}
\\ &\mbox{}\par\vspace{-\baselineskip}	
\begin{enumerate}[leftmargin=0.7cm, topsep=0.1cm]
\setcounter{enumi}{2}
\item Si el usuario selecciona la opción \textit{Mapa} en el selector.
	\begin{itemize}
	\item[1.] El sistema muestra los lugares en el mapa.
	\item[2.] Para cada lugar el sistema incluye.
		\begin{itemize}
		\item[1.] Una marca, que incluye nombre y categoría, en la ubicación exacta en el mapa.
		\item[2.] Un color diferente en función de si el lugar está asignado o no a la ruta.
		\item[3.] Un botón para asignar o desasignar el lugar a los días de la ruta.
		\end{itemize}
	\end{itemize}
\end{enumerate}


\\
\hline
\textbf{Secuencia Alternativa} &\mbox{}\par\vspace{-\baselineskip}
\begin{enumerate}[leftmargin=0.9cm, topsep=0.1cm]
\item[2-3.] El usuario pulsa el botón de \textit{Atrás}.
	\begin{itemize}
	\item[1.] El sistema retorna a la pantalla anterior.
	\end{itemize}
\end{enumerate}
\\

\hline
\textbf{Postcondiciones} & \\
\hline
\end{longtable}



\newpage
\subsubsection*{Caso de uso: Asignar lugares a ruta }
\begin{longtable}{| p{4cm} | p{10cm} |}
\endfirsthead
\multicolumn{2}{c}{\textit{Continúa de la página anterior}}\\[12pt]
\hline
\endhead
\hline
\multicolumn{2}{c}{\textit{Continúa en la siguiente página}} \\
\endfoot
\hline
\caption{Caso de Uso: Asginar lugares a ruta}\label{fig:1}\\
\endlastfoot


\hline
\multicolumn{2}{|c|}{\textbf{CU$<$21$>$ - Asignar Lugares a Ruta}} \\

\hline
\textbf{Descripción} &
El usuario añade o elimina lugares a la ruta.\\

\hline
\textbf{Actores} &
Cliente Móvil\\

\hline
\textbf{Punto de Extensión} &
\\

\hline
\textbf{Extiende} &
\\

\hline
\textbf{Incluye} &
CU$<$20$>$ - Consultar Lugares
\\

\hline
\textbf{Precondiciones} &
El usuario está autenticado en el sistema.\\

\hline
\textbf{Secuencia Normal} &\mbox{}\par\vspace{-\baselineskip}
\begin{enumerate}[leftmargin=0.7cm, topsep=0.1cm]
\item El usuario selecciona la opción \textit{Añadir Lugar}.
\item El sistema implementa \textit{CU$<$20$>$ - Consultar Lugares}
\item El usuario hace click en el botón para asignar o desasignar el lugar.
\item El sistema muestra una ventana con el conjunto de días que forman la ruta.
\item Si el día aparece \textit{Activado}.
	\begin{itemize}
	\item[1.] Dicho día ya tiene incorporado el lugar a la ruta.
	\end{itemize}
\item Si el día aparece \textit{Desactivado}.
	\begin{itemize}
	\item[1.] Dicho día no tiene incorporado el lugar a la ruta.
	\end{itemize}
\item El usuario, activando y desactivando, indica los días en los que desea incluir dicho lugar en la ruta y pulsa el botón de \textit{Aceptar}.
\end{enumerate}
\\ &\mbox{}\par\vspace{-\baselineskip}	
\begin{enumerate}[leftmargin=0.7cm, topsep=0.1cm]
\item[8.] El sistema confirma la acción y actualiza la ruta del usuario.

\end{enumerate}


\\
\hline
\textbf{Secuencia Alternativa} &\mbox{}\par\vspace{-\baselineskip}
\begin{enumerate}[leftmargin=0.9cm, topsep=0.1cm]
\item[3.] El usuario pulsa el botón de \textit{Atrás}.
	\begin{itemize}
	\item[1.] El sistema retorna a la pantalla anterior.
	\end{itemize}
\item[4-7.] El usuario pulsa el botón de \textit{Cancelar}.
	\begin{itemize}
	\item[1.] El sistema cancela la acción y no actualiza la información de la ruta.
	\end{itemize}
\item[8.] El usuario no tiene permisos necesarios para realizar determinada acción.
	\begin{itemize}
	\item[1.] El sistema muestra un mensaje de error indicando que no está autorizado a realizar dicha acción.
	\end{itemize}
\item[8.] La ruta en la que quiere incorporar los lugares no se encuentra en el sistema.
	\begin{itemize}
	\item[1.] El sistema muestra un mensaje de error indicando que no es posible realizar la acción.
	\end{itemize}
\end{enumerate}
\\

\hline
\textbf{Postcondiciones} & 
El lugar queda incluido en los días que el usuario indica.\\
\hline
\end{longtable}



\newpage
\subsubsection*{Caso de uso: Modificar datos personales }
\begin{longtable}{| p{4cm} | p{10cm} |}
\endfirsthead
\multicolumn{2}{c}{\textit{Continúa de la página anterior}}\\[12pt]
\hline
\endhead
\hline
\multicolumn{2}{c}{\textit{Continúa en la siguiente página}} \\
\endfoot
\hline
\caption{Caso de Uso: Modificar datos personales}\label{fig:1}\\
\endlastfoot


\hline
\multicolumn{2}{|c|}{\textbf{CU$<$22$>$ - Modificar Datos Personales}} \\

\hline
\textbf{Descripción} &
El usuario modifica sus datos en el sistema.\\

\hline
\textbf{Actores} &
Cliente Móvil\\

\hline
\textbf{Punto de Extensión} &
\\

\hline
\textbf{Extiende} &
\\

\hline
\textbf{Incluye} &
\\

\hline
\textbf{Precondiciones} &
El usuario está autenticado en el sistema.\\

\hline
\textbf{Secuencia Normal} &\mbox{}\par\vspace{-\baselineskip}
\begin{enumerate}[leftmargin=0.7cm, topsep=0.1cm]
\item El usuario selecciona la opción \textit{Editar Datos}.
\item El sistema muestra un formulario con los datos actuales del usuario.
\item El usuario modifica sus datos y pulsa el botón \textit{Aceptar}.
\item El sistema actualiza los datos del usuario.
\end{enumerate}


\\
\hline
\textbf{Secuencia Alternativa} &\mbox{}\par\vspace{-\baselineskip}
\begin{enumerate}[leftmargin=0.9cm, topsep=0.1cm]
\item[3.] El usuario pulsa el botón de \textit{Cancelar}.
	\begin{itemize}
	\item[1.] El sistema cancela la acción y no actualiza la información del usuario.
	\end{itemize}
\item[4.] El usuario no tiene permisos necesarios para realizar determinada acción.
	\begin{itemize}
	\item[1.] El sistema muestra un mensaje de error indicando que no está autorizado a realizar dicha acción.
	\end{itemize}
\end{enumerate}
\\

\hline
\textbf{Postcondiciones} & 
Los datos del usuario quedan actualizados en el sistema.\\
\hline
\end{longtable}



\newpage
\subsubsection*{Caso de uso: Consultar usuarios }
\begin{longtable}{| p{4cm} | p{10cm} |}
\endfirsthead
\multicolumn{2}{c}{\textit{Continúa de la página anterior}}\\[12pt]
\hline
\endhead
\hline
\multicolumn{2}{c}{\textit{Continúa en la siguiente página}} \\
\endfoot
\hline
\caption{Caso de Uso: Consultar usuarios}\label{fig:1}\\
\endlastfoot


\hline
\multicolumn{2}{|c|}{\textbf{CU$<$23$>$ - Consultar Usuarios}} \\

\hline
\textbf{Descripción} &
Muestra los diferentes usuarios registrados en el sistema.\\

\hline
\textbf{Actores} &
Administrador\\

\hline
\textbf{Punto de Extensión} &
\\

\hline
\textbf{Extiende} &
\\

\hline
\textbf{Incluye} &
\\

\hline
\textbf{Precondiciones} &
El administrador está autenticado en el sistema.\\

\hline
\textbf{Secuencia Normal} &\mbox{}\par\vspace{-\baselineskip}
\begin{enumerate}[leftmargin=0.7cm, topsep=0.1cm]
\item El administrador selecciona la opción \textit{Consultar Usuarios}.
\item El sistema muestra una tabla con todos los usuarios del sistema.
\end{enumerate}


\\
\hline
\textbf{Secuencia Alternativa} &\mbox{}\par\vspace{-\baselineskip}
\\

\hline
\textbf{Postcondiciones} & \\
\hline
\end{longtable}



\newpage
\subsubsection*{Caso de uso: Alta usuario }
\begin{longtable}{| p{4cm} | p{10cm} |}
\endfirsthead
\multicolumn{2}{c}{\textit{Continúa de la página anterior}}\\[12pt]
\hline
\endhead
\hline
\multicolumn{2}{c}{\textit{Continúa en la siguiente página}} \\
\endfoot
\hline
\caption{Caso de Uso: Alta usuario}\label{fig:1}\\
\endlastfoot


\hline
\multicolumn{2}{|c|}{\textbf{CU$<$24$>$ - Alta Usuario}} \\

\hline
\textbf{Descripción} &
Añade un nuevo usuario al sistema.\\

\hline
\textbf{Actores} &
Administrador\\

\hline
\textbf{Punto de Extensión} &
\\

\hline
\textbf{Extiende} &
\\

\hline
\textbf{Incluye} &
\\

\hline
\textbf{Precondiciones} &
El administrador está autenticado en el sistema.\\

\hline
\textbf{Secuencia Normal} &\mbox{}\par\vspace{-\baselineskip}
\begin{enumerate}[leftmargin=0.7cm, topsep=0.1cm]
\item El administrador selecciona la opción \textit{Añadir Usuario}.
\item El sistema muestra un formulario con los datos a rellenar por el usuario.
\item El administrador introduce los datos y pulsa sobre el botón \textit{Confirmar}.
\item El sistema comprueba los datos y los inserta en el sistema.
\end{enumerate}


\\
\hline
\textbf{Secuencia Alternativa} &\mbox{}\par\vspace{-\baselineskip}
\begin{enumerate}[leftmargin=0.9cm, topsep=0.1cm]
\item[3.] El administrador pulsa el botón de \textit{Cancelar}.
	\begin{itemize}
	\item[1.] El sistema cancela la acción y detiene el proceso de alta.
	\end{itemize}
\item[4.] El sistema no acepta los datos introducidos.
	\begin{itemize}
	\item[1.] El sistema cancela la acción y muestra un error al usuario.
	\end{itemize}
\end{enumerate}
\\

\hline
\textbf{Postcondiciones} & 
El usuario queda registrado en el sistema.\\
\hline
\end{longtable}



\newpage
\subsubsection*{Caso de uso: Modificar usuario }
\begin{longtable}{| p{4cm} | p{10cm} |}
\endfirsthead
\multicolumn{2}{c}{\textit{Continúa de la página anterior}}\\[12pt]
\hline
\endhead
\hline
\multicolumn{2}{c}{\textit{Continúa en la siguiente página}} \\
\endfoot
\hline
\caption{Caso de Uso: Modificar usuario}\label{fig:1}\\
\endlastfoot


\hline
\multicolumn{2}{|c|}{\textbf{CU$<$25$>$ - Modificar Usuario}} \\

\hline
\textbf{Descripción} &
Modifica los datos de un usuario.\\

\hline
\textbf{Actores} &
Administrador\\

\hline
\textbf{Punto de Extensión} &
\\

\hline
\textbf{Extiende} &
\\

\hline
\textbf{Incluye} &
\\

\hline
\textbf{Precondiciones} &
El administrador está autenticado en el sistema\\

\hline
\textbf{Secuencia Normal} &\mbox{}\par\vspace{-\baselineskip}
\begin{enumerate}[leftmargin=0.7cm, topsep=0.1cm]
\item El administrador selecciona la opción de \textit{Modificar} sobre un usuario.
\item El sistema muestra un formulario con los datos actuales del usuario seleccionado.
\item El administrador modifica los datos que desea y pulsa sobre el botón \textit{Confirmar}.
\item El sistema comprueba los datos y los actualiza.
\end{enumerate}


\\
\hline
\textbf{Secuencia Alternativa} &\mbox{}\par\vspace{-\baselineskip}
\begin{enumerate}[leftmargin=0.9cm, topsep=0.1cm]
\item[3.] El administrador pulsa el botón de \textit{Cancelar}.
	\begin{itemize}
	\item[1.] El sistema cancela la acción y no actualiza la información del usuario.
	\end{itemize}
\item[4.] El sistema no acepta los datos introducidos.
	\begin{itemize}
	\item[1.] El sistema cancela la acción y muestra un error al usuario.
	\end{itemize}
\end{enumerate}
\\

\hline
\textbf{Postcondiciones} & 
Los datos del usuario quedan actualizados en el sistema\\
\hline
\end{longtable}



\newpage
\subsubsection*{Caso de uso: Eliminar usuario }
\begin{longtable}{| p{4cm} | p{10cm} |}
\endfirsthead
\multicolumn{2}{c}{\textit{Continúa de la página anterior}}\\[12pt]
\hline
\endhead
\hline
\multicolumn{2}{c}{\textit{Continúa en la siguiente página}} \\
\endfoot
\hline
\caption{Caso de Uso: Eliminar usuario}\label{fig:1}\\
\endlastfoot


\hline
\multicolumn{2}{|c|}{\textbf{CU$<$26$>$ - Eliminar Usuario}} \\

\hline
\textbf{Descripción} &
Elimina un usuario del sistema.\\

\hline
\textbf{Actores} &
Administrador\\

\hline
\textbf{Punto de Extensión} &
\\

\hline
\textbf{Extiende} &
\\

\hline
\textbf{Incluye} &
\\

\hline
\textbf{Precondiciones} &
El administrador está autenticado en el sistema.\\

\hline
\textbf{Secuencia Normal} &\mbox{}\par\vspace{-\baselineskip}
\begin{enumerate}[leftmargin=0.7cm, topsep=0.1cm]
\item El administrador selecciona la opción de \textit{Eliminar} sobre un usuario.
\item El sistema muestra una pantalla de confirmación.
\item El administrador pulsa sobre el botón de \textit{Confirmar}.
\item El sistema realiza la acción.
\end{enumerate}


\\
\hline
\textbf{Secuencia Alternativa} &\mbox{}\par\vspace{-\baselineskip}
\begin{enumerate}[leftmargin=0.9cm, topsep=0.1cm]
\item[3.] El administrador pulsa el botón de \textit{Cancelar}.
	\begin{itemize}
	\item[1.] El sistema cancela la acción y detiene el proceso de eliminación.
	\end{itemize}
\end{enumerate}
\\

\hline
\textbf{Postcondiciones} & 
El datos del usuario quedan eliminados del sistema.\\
\hline
\end{longtable}



\newpage
\subsubsection*{Caso de uso: Consultar rutas }
\begin{longtable}{| p{4cm} | p{10cm} |}
\endfirsthead
\multicolumn{2}{c}{\textit{Continúa de la página anterior}}\\[12pt]
\hline
\endhead
\hline
\multicolumn{2}{c}{\textit{Continúa en la siguiente página}} \\
\endfoot
\hline
\caption{Caso de Uso: Consultar rutas}\label{fig:1}\\
\endlastfoot


\hline
\multicolumn{2}{|c|}{\textbf{CU$<$27$>$ - Consultar Rutas}} \\

\hline
\textbf{Descripción} &
Muestra los diferentes rutas registradas en el sistema.\\

\hline
\textbf{Actores} &
Administrador\\

\hline
\textbf{Punto de Extensión} &
\\

\hline
\textbf{Extiende} &
\\

\hline
\textbf{Incluye} &
\\

\hline
\textbf{Precondiciones} &
El usuario está autenticado en el sistema.\\

\hline
\textbf{Secuencia Normal} &\mbox{}\par\vspace{-\baselineskip}
\begin{enumerate}[leftmargin=0.7cm, topsep=0.1cm]
\item El usuario selecciona la opción \textit{Consultar Rutas}.
\item El sistema muestra una tabla con todas las rutas del sistema.
\end{enumerate}


\\
\hline
\textbf{Secuencia Alternativa} &\mbox{}\par\vspace{-\baselineskip}
\\

\hline
\textbf{Postcondiciones} & \\
\hline
\end{longtable}



\newpage
\subsubsection*{Caso de uso: Alta ruta }
\begin{longtable}{| p{4cm} | p{10cm} |}
\endfirsthead
\multicolumn{2}{c}{\textit{Continúa de la página anterior}}\\[12pt]
\hline
\endhead
\hline
\multicolumn{2}{c}{\textit{Continúa en la siguiente página}} \\
\endfoot
\hline
\caption{Caso de Uso: Alta ruta}\label{fig:1}\\
\endlastfoot


\hline
\multicolumn{2}{|c|}{\textbf{CU$<$28$>$ - Alta Ruta}} \\

\hline
\textbf{Descripción} &
Añade una nueva ruta al sistema.\\

\hline
\textbf{Actores} &
Administrador\\

\hline
\textbf{Punto de Extensión} &
\\

\hline
\textbf{Extiende} &
\\

\hline
\textbf{Incluye} &
\\

\hline
\textbf{Precondiciones} &
El usuario está autenticado en el sistema.\\

\hline
\textbf{Secuencia Normal} &\mbox{}\par\vspace{-\baselineskip}
\begin{enumerate}[leftmargin=0.7cm, topsep=0.1cm]
\item El usuario selecciona la opción \textit{Añadir Ruta}.
\item El sistema muestra un formulario con los datos a rellenar por el usuario.
\item El usuario introduce los datos y pulsa sobre el botón \textit{Confirmar}.
\item El sistema comprueba los datos y los inserta en el sistema.
\end{enumerate}


\\
\hline
\textbf{Secuencia Alternativa} &\mbox{}\par\vspace{-\baselineskip}
\begin{enumerate}[leftmargin=0.9cm, topsep=0.1cm]
\item[3.] El usuario pulsa el botón de \textit{Cancelar}.
	\begin{itemize}
	\item[1.] El sistema cancela la acción y detiene el proceso de alta.
	\end{itemize}
\item[4.] El sistema no acepta los datos introducidos.
	\begin{itemize}
	\item[1.] El sistema cancela la acción y muestra un error al usuario.
	\end{itemize}
\end{enumerate}
\\

\hline
\textbf{Postcondiciones} & 
La ruta queda registrada en el sistema.\\
\hline
\end{longtable}



\newpage
\subsubsection*{Caso de uso: Modificar ruta }
\begin{longtable}{| p{4cm} | p{10cm} |}
\endfirsthead
\multicolumn{2}{c}{\textit{Continúa de la página anterior}}\\[12pt]
\hline
\endhead
\hline
\multicolumn{2}{c}{\textit{Continúa en la siguiente página}} \\
\endfoot
\hline
\caption{Caso de Uso: Modificar ruta}\label{fig:1}\\
\endlastfoot


\hline
\multicolumn{2}{|c|}{\textbf{CU$<$29$>$ - Modificar Ruta}} \\

\hline
\textbf{Descripción} &
Modifica los datos de una ruta.\\

\hline
\textbf{Actores} &
Administrador\\

\hline
\textbf{Punto de Extensión} &
\\

\hline
\textbf{Extiende} &
\\

\hline
\textbf{Incluye} &
\\

\hline
\textbf{Precondiciones} &
El usuario está autenticado en el sistema.\\

\hline
\textbf{Secuencia Normal} &\mbox{}\par\vspace{-\baselineskip}
\begin{enumerate}[leftmargin=0.7cm, topsep=0.1cm]
\item El usuario selecciona la opción de \textit{Modificar} sobre una ruta.
\item El sistema muestra un formulario con los datos actuales de la ruta seleccionada.
\item El usuario modifica los datos que desea y pulsa sobre el botón \textit{Confirmar}.
\item El sistema comprueba los datos y los actualiza.
\end{enumerate}


\\
\hline
\textbf{Secuencia Alternativa} &\mbox{}\par\vspace{-\baselineskip}
\begin{enumerate}[leftmargin=0.9cm, topsep=0.1cm]
\item[3.] El usuario pulsa el botón de \textit{Cancelar}.
	\begin{itemize}
	\item[1.] El sistema cancela la acción y no actualiza la información de la ruta.
	\end{itemize}
\item[4.] El sistema no acepta los datos introducidos.
	\begin{itemize}
	\item[1.] El sistema cancela la acción y muestra un error al usuario.
	\end{itemize}
\end{enumerate}
\\

\hline
\textbf{Postcondiciones} & 
Los datos de la ruta quedan actualizados en el sistema.\\
\hline
\end{longtable}



\newpage
\subsubsection*{Caso de uso: Eliminar ruta }
\begin{longtable}{| p{4cm} | p{10cm} |}
\endfirsthead
\multicolumn{2}{c}{\textit{Continúa de la página anterior}}\\[12pt]
\hline
\endhead
\hline
\multicolumn{2}{c}{\textit{Continúa en la siguiente página}} \\
\endfoot
\hline
\caption{Caso de Uso: Eliminar ruta}\label{fig:1}\\
\endlastfoot


\hline
\multicolumn{2}{|c|}{\textbf{CU$<$30$>$ - Eliminar Ruta}} \\

\hline
\textbf{Descripción} &
Elimina una ruta del sistema.\\

\hline
\textbf{Actores} &
Administrador\\

\hline
\textbf{Punto de Extensión} &
\\

\hline
\textbf{Extiende} &
\\

\hline
\textbf{Incluye} &
\\

\hline
\textbf{Precondiciones} &
El usuario está autenticado en el sistema.\\

\hline
\textbf{Secuencia Normal} &\mbox{}\par\vspace{-\baselineskip}
\begin{enumerate}[leftmargin=0.7cm, topsep=0.1cm]
\item El usuario selecciona la opción de \textit{Eliminar} sobre una ruta.
\item El sistema muestra una pantalla de confirmación.
\item El usuario pulsa sobre el botón de \textit{Confirmar}.
\item El sistema realiza la acción.
\end{enumerate}


\\
\hline
\textbf{Secuencia Alternativa} &\mbox{}\par\vspace{-\baselineskip}
\begin{enumerate}[leftmargin=0.9cm, topsep=0.1cm]
\item[3.] El usuario pulsa el botón de \textit{Cancelar}.
	\begin{itemize}
	\item[1.] El sistema cancela la acción y detiene el proceso de eliminación.
	\end{itemize}
\end{enumerate}
\\

\hline
\textbf{Postcondiciones} & 
Los datos de la ruta quedan eliminados del sistema.\\
\hline
\end{longtable}




\newpage
\subsubsection*{Caso de uso: Consultar visitas }
\begin{longtable}{| p{4cm} | p{10cm} |}
\endfirsthead
\multicolumn{2}{c}{\textit{Continúa de la página anterior}}\\[12pt]
\hline
\endhead
\hline
\multicolumn{2}{c}{\textit{Continúa en la siguiente página}} \\
\endfoot
\hline
\caption{Caso de Uso: Consultar visitas}\label{fig:1}\\
\endlastfoot


\hline
\multicolumn{2}{|c|}{\textbf{CU$<$31$>$ - Consultar Visitas}} \\

\hline
\textbf{Descripción} &
Muestra los diferentes visitas registradas en el sistema.\\

\hline
\textbf{Actores} &
Administrador\\

\hline
\textbf{Punto de Extensión} &
\\

\hline
\textbf{Extiende} &
\\

\hline
\textbf{Incluye} &
\\

\hline
\textbf{Precondiciones} &
El usuario está autenticado en el sistema.\\

\hline
\textbf{Secuencia Normal} &\mbox{}\par\vspace{-\baselineskip}
\begin{enumerate}[leftmargin=0.7cm, topsep=0.1cm]
\item El usuario selecciona la opción \textit{Consultar Visitas}.
\item El sistema muestra una tabla con todas las visitas del sistema.
\end{enumerate}


\\
\hline
\textbf{Secuencia Alternativa} &\mbox{}\par\vspace{-\baselineskip}
\\

\hline
\textbf{Postcondiciones} & \\
\hline
\end{longtable}



\newpage
\subsubsection*{Caso de uso: Alta visita }
\begin{longtable}{| p{4cm} | p{10cm} |}
\endfirsthead
\multicolumn{2}{c}{\textit{Continúa de la página anterior}}\\[12pt]
\hline
\endhead
\hline
\multicolumn{2}{c}{\textit{Continúa en la siguiente página}} \\
\endfoot
\hline
\caption{Caso de Uso: Alta visita}\label{fig:1}\\
\endlastfoot


\hline
\multicolumn{2}{|c|}{\textbf{CU$<$32$>$ - Alta Visita}} \\

\hline
\textbf{Descripción} &
Añade una nueva visita al sistema.\\

\hline
\textbf{Actores} &
Administrador\\

\hline
\textbf{Punto de Extensión} &
\\

\hline
\textbf{Extiende} &
\\

\hline
\textbf{Incluye} &
\\

\hline
\textbf{Precondiciones} &
El usuario está autenticado en el sistema.\\

\hline
\textbf{Secuencia Normal} &\mbox{}\par\vspace{-\baselineskip}
\begin{enumerate}[leftmargin=0.7cm, topsep=0.1cm]
\item El usuario selecciona la opción \textit{Añadir Visita}.
\item El sistema muestra un formulario con los datos a rellenar por el usuario.
\item El usuario introduce los datos y pulsa sobre el botón \textit{Confirmar}.
\item El sistema comprueba los datos y los inserta en el sistema.
\end{enumerate}


\\
\hline
\textbf{Secuencia Alternativa} &\mbox{}\par\vspace{-\baselineskip}
\begin{enumerate}[leftmargin=0.9cm, topsep=0.1cm]
\item[3.] El usuario pulsa el botón de \textit{Cancelar}.
	\begin{itemize}
	\item[1.] El sistema cancela la acción y detiene el proceso de alta.
	\end{itemize}
\item[4.] El sistema no acepta los datos introducidos.
	\begin{itemize}
	\item[1.] El sistema cancela la acción y muestra un error al usuario.
	\end{itemize}
\end{enumerate}
\\

\hline
\textbf{Postcondiciones} & 
La visita queda registrada en el sistema.\\
\hline
\end{longtable}



\newpage
\subsubsection*{Caso de uso: Modificar visita }
\begin{longtable}{| p{4cm} | p{10cm} |}
\endfirsthead
\multicolumn{2}{c}{\textit{Continúa de la página anterior}}\\[12pt]
\hline
\endhead
\hline
\multicolumn{2}{c}{\textit{Continúa en la siguiente página}} \\
\endfoot
\hline
\caption{Caso de Uso: Modificar visita}\label{fig:1}\\
\endlastfoot


\hline
\multicolumn{2}{|c|}{\textbf{CU$<$33$>$ - Modificar Visita}} \\

\hline
\textbf{Descripción} &
Modifica los datos de una visita.\\

\hline
\textbf{Actores} &
Administrador\\

\hline
\textbf{Punto de Extensión} &
\\

\hline
\textbf{Extiende} &
\\

\hline
\textbf{Incluye} &
\\

\hline
\textbf{Precondiciones} &
El usuario está autenticado en el sistema.\\

\hline
\textbf{Secuencia Normal} &\mbox{}\par\vspace{-\baselineskip}
\begin{enumerate}[leftmargin=0.7cm, topsep=0.1cm]
\item El usuario selecciona la opción de \textit{Modificar} sobre una visita.
\item El sistema muestra un formulario con los datos actuales de la visita seleccionada.
\item El usuario modifica los datos que desea y pulsa sobre el botón \textit{Confirmar}.
\item El sistema comprueba los datos y los actualiza.
\end{enumerate}


\\
\hline
\textbf{Secuencia Alternativa} &\mbox{}\par\vspace{-\baselineskip}
\begin{enumerate}[leftmargin=0.9cm, topsep=0.1cm]
\item[3.] El usuario pulsa el botón de \textit{Cancelar}.
	\begin{itemize}
	\item[1.] El sistema cancela la acción y no actualiza la información de la visita.
	\end{itemize}
\item[4.] El sistema no acepta los datos introducidos.
	\begin{itemize}
	\item[1.] El sistema cancela la acción y muestra un error al usuario.
	\end{itemize}
\end{enumerate}
\\

\hline
\textbf{Postcondiciones} & 
Los datos de la visita quedan actualizados en el sistema.\\
\hline
\end{longtable}



\newpage
\subsubsection*{Caso de uso: Eliminar visita }
\begin{longtable}{| p{4cm} | p{10cm} |}
\endfirsthead
\multicolumn{2}{c}{\textit{Continúa de la página anterior}}\\[12pt]
\hline
\endhead
\hline
\multicolumn{2}{c}{\textit{Continúa en la siguiente página}} \\
\endfoot
\hline
\caption{Caso de Uso: Eliminar visita}\label{fig:1}\\
\endlastfoot


\hline
\multicolumn{2}{|c|}{\textbf{CU$<$34$>$ - Eliminar Visita}} \\

\hline
\textbf{Descripción} &
Elimina una visita del sistema.\\

\hline
\textbf{Actores} &
Administrador\\

\hline
\textbf{Punto de Extensión} &
\\

\hline
\textbf{Extiende} &
\\

\hline
\textbf{Incluye} &
\\

\hline
\textbf{Precondiciones} &
El usuario está autenticado en el sistema.\\

\hline
\textbf{Secuencia Normal} &\mbox{}\par\vspace{-\baselineskip}
\begin{enumerate}[leftmargin=0.7cm, topsep=0.1cm]
\item El usuario selecciona la opción de \textit{Eliminar} sobre una visita.
\item El sistema muestra una pantalla de confirmación.
\item El usuario pulsa sobre el botón de \textit{Confirmar}.
\item El sistema realiza la acción.
\end{enumerate}


\\
\hline
\textbf{Secuencia Alternativa} &\mbox{}\par\vspace{-\baselineskip}
\begin{enumerate}[leftmargin=0.9cm, topsep=0.1cm]
\item[3.] El usuario pulsa el botón de \textit{Cancelar}.
	\begin{itemize}
	\item[1.] El sistema cancela la acción y detiene el proceso de eliminación.
	\end{itemize}
\end{enumerate}
\\

\hline
\textbf{Postcondiciones} & 
Los datos de la visita quedan eliminados del sistema.\\
\hline
\end{longtable}



\newpage
\subsubsection*{Caso de uso: Consultar lugares }
\begin{longtable}{| p{4cm} | p{10cm} |}

\hline
\caption{Caso de Uso: Consultar lugares}\label{fig:1}\\
\endlastfoot


\hline
\multicolumn{2}{|c|}{\textbf{CU$<$35$>$ - Consultar Lugares}} \\

\hline
\textbf{Descripción} &
Muestra los diferentes lugares registrados en el sistema.\\

\hline
\textbf{Actores} &
Administrador\\

\hline
\textbf{Punto de Extensión} &
\\

\hline
\textbf{Extiende} &
\\

\hline
\textbf{Incluye} &
\\

\hline
\textbf{Precondiciones} &
El usuario está autenticado en el sistema.\\

\hline
\textbf{Secuencia Normal} &\mbox{}\par\vspace{-\baselineskip}

\begin{enumerate}[leftmargin=0.7cm, topsep=0.1cm]
\item El usuario selecciona la opción \textit{Consultar Lugares}.
\item El sistema muestra una tabla con todos los lugares del sistema.
\end{enumerate}


\\
\hline
\textbf{Secuencia Alternativa} &\mbox{}\par\vspace{-\baselineskip}
\\

\hline
\textbf{Postcondiciones} & \\
\hline
\end{longtable}



\newpage
\subsubsection*{Caso de uso: Modificar lugar }
\begin{longtable}{| p{4cm} | p{10cm} |}
\endfirsthead
\multicolumn{2}{c}{\textit{Continúa de la página anterior}}\\[12pt]
\hline
\endhead
\hline
\multicolumn{2}{c}{\textit{Continúa en la siguiente página}} \\
\endfoot
\hline
\caption{Caso de Uso: Modificar lugar}\label{fig:1}\\
\endlastfoot


\hline
\multicolumn{2}{|c|}{\textbf{CU$<$36$>$ - Modificar Lugar}} \\

\hline
\textbf{Descripción} &
Modifica los datos de un lugar.\\

\hline
\textbf{Actores} &
Administrador\\

\hline
\textbf{Punto de Extensión} &
\\

\hline
\textbf{Extiende} &
\\

\hline
\textbf{Incluye} &
\\

\hline
\textbf{Precondiciones} &
El usuario está autenticado en el sistema.\\

\hline
\textbf{Secuencia Normal} &\mbox{}\par\vspace{-\baselineskip}
\begin{enumerate}[leftmargin=0.7cm, topsep=0.1cm]
\item El usuario selecciona la opción de \textit{Modificar} sobre un lugar.
\item El sistema muestra un formulario con los datos actuales del lugar seleccionado.
\item El usuario modifica los datos que desea y pulsa sobre el botón \textit{Confirmar}.
\item El sistema comprueba los datos y los actualiza.
\end{enumerate}


\\
\hline
\textbf{Secuencia Alternativa} &\mbox{}\par\vspace{-\baselineskip}
\begin{enumerate}[leftmargin=0.9cm, topsep=0.1cm]
\item[3.] El usuario pulsa el botón de \textit{Cancelar}.
	\begin{itemize}
	\item[1.] El sistema cancela la acción y no actualiza la información del lugar.
	\end{itemize}
\item[4.] El sistema no acepta los datos introducidos.
	\begin{itemize}
	\item[1.] El sistema cancela la acción y muestra un error al usuario.
	\end{itemize}
\end{enumerate}
\\

\hline
\textbf{Postcondiciones} & 
Los datos del lugar quedan actualizados en el sistema.\\
\hline
\end{longtable}



\newpage
\subsubsection*{Caso de uso: Eliminar lugar }
\begin{longtable}{| p{4cm} | p{10cm} |}
\endfirsthead
\multicolumn{2}{c}{\textit{Continúa de la página anterior}}\\[12pt]
\hline
\endhead
\hline
\multicolumn{2}{c}{\textit{Continúa en la siguiente página}} \\
\endfoot
\hline
\caption{Caso de Uso: Eliminar lugar}\label{fig:1}\\
\endlastfoot


\hline
\multicolumn{2}{|c|}{\textbf{CU$<$37$>$ - Eliminar Lugar}} \\

\hline
\textbf{Descripción} &
Elimina un lugar del sistema.\\

\hline
\textbf{Actores} &
Administrador\\

\hline
\textbf{Punto de Extensión} &
\\

\hline
\textbf{Extiende} &
\\

\hline
\textbf{Incluye} &
\\

\hline
\textbf{Precondiciones} &
El usuario está autenticado en el sistema.\\

\hline
\textbf{Secuencia Normal} &\mbox{}\par\vspace{-\baselineskip}
\begin{enumerate}[leftmargin=0.7cm, topsep=0.1cm]
\item El usuario selecciona la opción de \textit{Eliminar} sobre un lugar.
\item El sistema muestra una pantalla de confirmación.
\item El usuario pulsa sobre el botón de \textit{Confirmar}.
\item El sistema realiza la acción.
\end{enumerate}


\\
\hline
\textbf{Secuencia Alternativa} &\mbox{}\par\vspace{-\baselineskip}
\begin{enumerate}[leftmargin=0.9cm, topsep=0.1cm]
\item[3.] El usuario pulsa el botón de \textit{Cancelar}.
	\begin{itemize}
	\item[1.] El sistema cancela la acción y detiene el proceso de eliminación.
	\end{itemize}
\end{enumerate}
\\

\hline
\textbf{Postcondiciones} & 
Los datos del lugar quedan eliminados del sistema.\\
\hline
\end{longtable}



\newpage
\subsubsection*{Caso de uso: Consultar eventos }
\begin{longtable}{| p{4cm} | p{10cm} |}
\endfirsthead
\multicolumn{2}{c}{\textit{Continúa de la página anterior}}\\[12pt]
\hline
\endhead
\hline
\multicolumn{2}{c}{\textit{Continúa en la siguiente página}} \\
\endfoot
\hline
\caption{Caso de Uso: Consultar eventos}\label{fig:1}\\
\endlastfoot


\hline
\multicolumn{2}{|c|}{\textbf{CU$<$38$>$ - Consultar Eventos}} \\

\hline
\textbf{Descripción} &
Muestra los diferentes usuarios registrados en el sistema.\\

\hline
\textbf{Actores} &
Administrador\newline
Gestor de Eventos\\

\hline
\textbf{Punto de Extensión} &
\\

\hline
\textbf{Extiende} &
\\

\hline
\textbf{Incluye} &
\\

\hline
\textbf{Precondiciones} &
El usuario está autenticado en el sistema.\\

\hline
\textbf{Secuencia Normal} &\mbox{}\par\vspace{-\baselineskip}
\begin{enumerate}[leftmargin=0.7cm, topsep=0.1cm]
\item El usuario selecciona la opción \textit{Consultar Eventos}.
\item El sistema muestra una tabla con todos los eventos del sistema.
\end{enumerate}


\\
\hline
\textbf{Secuencia Alternativa} &\mbox{}\par\vspace{-\baselineskip}
\\

\hline
\textbf{Postcondiciones} & \\
\hline
\end{longtable}



\newpage
\subsubsection*{Caso de uso: Alta evento }
\begin{longtable}{| p{4cm} | p{10cm} |}
\endfirsthead
\multicolumn{2}{c}{\textit{Continúa de la página anterior}}\\[12pt]
\hline
\endhead
\hline
\multicolumn{2}{c}{\textit{Continúa en la siguiente página}} \\
\endfoot
\hline
\caption{Caso de Uso: Alta evento}\label{fig:1}\\
\endlastfoot


\hline
\multicolumn{2}{|c|}{\textbf{CU$<$39$>$ - Alta Evento}} \\

\hline
\textbf{Descripción} &
Añade un nuevo evento al sistema.\\

\hline
\textbf{Actores} &
Administrador\newline
Gestor de Eventos\\

\hline
\textbf{Punto de Extensión} &
\\

\hline
\textbf{Extiende} &
\\

\hline
\textbf{Incluye} &
\\

\hline
\textbf{Precondiciones} &
El usuario está autenticado en el sistema.\\

\hline
\textbf{Secuencia Normal} &\mbox{}\par\vspace{-\baselineskip}
\begin{enumerate}[leftmargin=0.7cm, topsep=0.1cm]
\item El usuario selecciona la opción \textit{Añadir Evento}.
\item El sistema muestra un formulario con los datos a rellenar por el usuario.
\item El usuario introduce los datos y pulsa sobre el botón \textit{Confirmar}.
\item El sistema comprueba los datos y los inserta en el sistema.
\end{enumerate}


\\
\hline
\textbf{Secuencia Alternativa} &\mbox{}\par\vspace{-\baselineskip}
\begin{enumerate}[leftmargin=0.9cm, topsep=0.1cm]
\item[3.] El usuario pulsa el botón de \textit{Cancelar}.
	\begin{itemize}
	\item[1.] El sistema cancela la acción y detiene el proceso de alta.
	\end{itemize}
\item[4.] El sistema no acepta los datos introducidos.
	\begin{itemize}
	\item[1.] El sistema cancela la acción y muestra un error al usuario.
	\end{itemize}
\end{enumerate}
\\

\hline
\textbf{Postcondiciones} & 
El evento queda registrado en el sistema.\\
\hline
\end{longtable}



\newpage
\subsubsection*{Caso de uso: Modificar evento }
\begin{longtable}{| p{4cm} | p{10cm} |}
\endfirsthead
\multicolumn{2}{c}{\textit{Continúa de la página anterior}}\\[12pt]
\hline
\endhead
\hline
\multicolumn{2}{c}{\textit{Continúa en la siguiente página}} \\
\endfoot
\hline
\caption{Caso de Uso: Modificar evento}\label{fig:1}\\
\endlastfoot


\hline
\multicolumn{2}{|c|}{\textbf{CU$<$40$>$ - Modificar Evento}} \\

\hline
\textbf{Descripción} &
Modifica los datos de un evento.\\

\hline
\textbf{Actores} &
Administrador\newline
Gestor de Eventos\\

\hline
\textbf{Punto de Extensión} &
\\

\hline
\textbf{Extiende} &
\\

\hline
\textbf{Incluye} &
\\

\hline
\textbf{Precondiciones} &
El usuario está autenticado en el sistema.\\

\hline
\textbf{Secuencia Normal} &\mbox{}\par\vspace{-\baselineskip}
\begin{enumerate}[leftmargin=0.7cm, topsep=0.1cm]
\item El usuario selecciona la opción de \textit{Modificar} sobre un evento.
\item El sistema muestra un formulario con los datos actuales del evento seleccionado.
\item El usuario modifica los datos que desea y pulsa sobre el botón \textit{Confirmar}.
\item El sistema comprueba los datos y los actualiza.
\end{enumerate}


\\
\hline
\textbf{Secuencia Alternativa} &\mbox{}\par\vspace{-\baselineskip}
\begin{enumerate}[leftmargin=0.9cm, topsep=0.1cm]
\item[3.] El usuario pulsa el botón de \textit{Cancelar}.
	\begin{itemize}
	\item[1.] El sistema cancela la acción y no actualiza la información del evento.
	\end{itemize}
\item[4.] El sistema no acepta los datos introducidos.
	\begin{itemize}
	\item[1.] El sistema cancela la acción y muestra un error al usuario.
	\end{itemize}
\end{enumerate}
\\

\hline
\textbf{Postcondiciones} & 
Los datos del evento quedan modificados en el sistema.\\
\hline
\end{longtable}



\newpage
\subsubsection*{Caso de uso: Eliminar evento }
\begin{longtable}{| p{4cm} | p{10cm} |}
\endfirsthead
\multicolumn{2}{c}{\textit{Continúa de la página anterior}}\\[12pt]
\hline
\endhead
\hline
\multicolumn{2}{c}{\textit{Continúa en la siguiente página}} \\
\endfoot
\hline
\caption{Caso de Uso: Eliminar evento}\label{fig:1}\\
\endlastfoot


\hline
\multicolumn{2}{|c|}{\textbf{CU$<$41$>$ - Eliminar Evento}} \\

\hline
\textbf{Descripción} &
Elimina un evento del sistema.\\

\hline
\textbf{Actores} &
Administrador\newline
Gestor de Eventos\\

\hline
\textbf{Punto de Extensión} &
\\

\hline
\textbf{Extiende} &
\\

\hline
\textbf{Incluye} &
\\

\hline
\textbf{Precondiciones} &
El usuario está autenticado en el sistema.\\

\hline
\textbf{Secuencia Normal} &\mbox{}\par\vspace{-\baselineskip}
\begin{enumerate}[leftmargin=0.7cm, topsep=0.1cm]
\item El usuario selecciona la opción de \textit{Eliminar} sobre un evento.
\item El sistema muestra una pantalla de confirmación.
\item El usuario pulsa sobre el botón de \textit{Confirmar}.
\item El sistema realiza la acción.
\end{enumerate}


\\
\hline
\textbf{Secuencia Alternativa} &\mbox{}\par\vspace{-\baselineskip}
\begin{enumerate}[leftmargin=0.9cm, topsep=0.1cm]
\item[3.] El usuario pulsa el botón de \textit{Cancelar}.
	\begin{itemize}
	\item[1.] El sistema cancela la acción y detiene el proceso de eliminación.
	\end{itemize}
\end{enumerate}
\\

\hline
\textbf{Postcondiciones} & 
Los datos del evento quedan eliminados del sistema.\\
\hline
\end{longtable}



\newpage
\subsubsection*{Caso de uso: Consultar categorías }
\begin{longtable}{| p{4cm} | p{10cm} |}
\endfirsthead
\multicolumn{2}{c}{\textit{Continúa de la página anterior}}\\[12pt]
\hline
\endhead
\hline
\multicolumn{2}{c}{\textit{Continúa en la siguiente página}} \\
\endfoot
\hline
\caption{Caso de Uso: Consultar categorías}\label{fig:1}\\
\endlastfoot


\hline
\multicolumn{2}{|c|}{\textbf{CU$<$42$>$ - Consultar Categorías}} \\

\hline
\textbf{Descripción} &
Muestra los diferentes categorías registradas en el sistema.\\

\hline
\textbf{Actores} &
Administrador\\


\hline
\textbf{Punto de Extensión} &
\\

\hline
\textbf{Extiende} &
\\

\hline
\textbf{Incluye} &
\\

\hline
\textbf{Precondiciones} &
El usuario está autenticado en el sistema.\\

\hline
\textbf{Secuencia Normal} &\mbox{}\par\vspace{-\baselineskip}
\begin{enumerate}[leftmargin=0.7cm, topsep=0.1cm]
\item El usuario selecciona la opción \textit{Consultar Categorías}.
\item El sistema muestra una tabla con todas las categorías del sistema.
\end{enumerate}


\\
\hline
\textbf{Secuencia Alternativa} &\mbox{}\par\vspace{-\baselineskip}
\\

\hline
\textbf{Postcondiciones} & \\
\hline
\end{longtable}



\newpage
\subsubsection*{Caso de uso: Cargar categorías }
\begin{longtable}{| p{4cm} | p{10cm} |}
\endfirsthead
\multicolumn{2}{c}{\textit{Continúa de la página anterior}}\\[12pt]
\hline
\endhead
\hline
\multicolumn{2}{c}{\textit{Continúa en la siguiente página}} \\
\endfoot
\hline
\caption{Caso de Uso: Cargar categorías}\label{fig:1}\\
\endlastfoot


\hline
\multicolumn{2}{|c|}{\textbf{CU$<$43$>$ - Cargar Categorías}} \\

\hline
\textbf{Descripción} &
Obtiene las categorías de una fuente externa y las actualiza en el sistema.\\

\hline
\textbf{Actores} &
Administrador\\

\hline
\textbf{Punto de Extensión} &
\\

\hline
\textbf{Extiende} &
\\

\hline
\textbf{Incluye} &
\\

\hline
\textbf{Precondiciones} &
El usuario está autenticado en el sistema.\\

\hline
\textbf{Secuencia Normal} &\mbox{}\par\vspace{-\baselineskip}
\begin{enumerate}[leftmargin=0.7cm, topsep=0.1cm]
\item El usuario selecciona la opción \textit{Cargar Categorías}.
\item El sistema muestra una pantalla de confirmación.
\item El usuario pulsa sobre el botón de \textit{Confirmar}.
\item El sistema realiza la acción.
\end{enumerate}


\\
\hline
\textbf{Secuencia Alternativa} &\mbox{}\par\vspace{-\baselineskip}
\begin{enumerate}[leftmargin=0.9cm, topsep=0.1cm]
\item[3.] El usuario pulsa el botón de \textit{Cancelar}.
	\begin{itemize}
	\item[1.] El sistema cancela la acción y detiene el proceso de alta.
	\end{itemize}
\item[4.] El sistema no puede realizar la acción.
	\begin{itemize}
	\item[1.] El sistema cancela la acción y muestra un error al usuario.
	\end{itemize}
\end{enumerate}
\\

\hline
\textbf{Postcondiciones} & 
Los datos obtenidos de las categorías quedan registrados en el sistema.\\
\hline
\end{longtable}



\newpage
\subsubsection*{Caso de uso: Modificar categoría }
\begin{longtable}{| p{4cm} | p{10cm} |}
\endfirsthead
\multicolumn{2}{c}{\textit{Continúa de la página anterior}}\\[12pt]
\hline
\endhead
\hline
\multicolumn{2}{c}{\textit{Continúa en la siguiente página}} \\
\endfoot
\hline
\caption{Caso de Uso: Modificar categoría}\label{fig:1}\\
\endlastfoot


\hline
\multicolumn{2}{|c|}{\textbf{CU$<$44$>$ - Modificar Categoría}} \\

\hline
\textbf{Descripción} &
Modifica los datos de una categoría.\\

\hline
\textbf{Actores} &
Administrador\\

\hline
\textbf{Punto de Extensión} &
\\

\hline
\textbf{Extiende} &
\\

\hline
\textbf{Incluye} &
\\

\hline
\textbf{Precondiciones} &
El usuario está autenticado en el sistema.\\

\hline
\textbf{Secuencia Normal} &\mbox{}\par\vspace{-\baselineskip}
\begin{enumerate}[leftmargin=0.7cm, topsep=0.1cm]
\item El usuario selecciona la opción de \textit{Modificar} sobre una categoría.
\item El sistema muestra un formulario con los datos actuales de la categoría seleccionada.
\item El usuario modifica los datos que desea y pulsa sobre el botón \textit{Confirmar}.
\item El sistema comprueba los datos y los actualiza.
\end{enumerate}


\\
\hline
\textbf{Secuencia Alternativa} &\mbox{}\par\vspace{-\baselineskip}
\begin{enumerate}[leftmargin=0.9cm, topsep=0.1cm]
\item[3.] El usuario pulsa el botón de \textit{Cancelar}.
	\begin{itemize}
	\item[1.] El sistema cancela la acción y no actualiza la información de la categoría.
	\end{itemize}
\item[4.] El sistema no acepta los datos introducidos.
	\begin{itemize}
	\item[1.] El sistema cancela la acción y muestra un error al usuario.
	\end{itemize}
\end{enumerate}
\\

\hline
\textbf{Postcondiciones} & 
Los datos de la categoría quedan actualizados en el sistema.\\
\hline
\end{longtable}



\newpage
\subsubsection*{Caso de uso: Eliminar categoría }
\begin{longtable}{| p{4cm} | p{10cm} |}
\endfirsthead
\multicolumn{2}{c}{\textit{Continúa de la página anterior}}\\[12pt]
\hline
\endhead
\hline
\multicolumn{2}{c}{\textit{Continúa en la siguiente página}} \\
\endfoot
\hline
\caption{Caso de Uso: Eliminar categoría}\label{fig:1}\\
\endlastfoot


\hline
\multicolumn{2}{|c|}{\textbf{CU$<$45$>$ - Eliminar Categoría}} \\

\hline
\textbf{Descripción} &
Elimina una ruta del sistema.\\

\hline
\textbf{Actores} &
Administrador\\

\hline
\textbf{Punto de Extensión} &
\\

\hline
\textbf{Extiende} &
\\

\hline
\textbf{Incluye} &
\\

\hline
\textbf{Precondiciones} &
El usuario está autenticado en el sistema.\\

\hline
\textbf{Secuencia Normal} &\mbox{}\par\vspace{-\baselineskip}
\begin{enumerate}[leftmargin=0.7cm, topsep=0.1cm]
\item El usuario selecciona la opción de \textit{Eliminar} sobre una ruta.
\item El sistema muestra una pantalla de confirmación.
\item El usuario pulsa sobre el botón de \textit{Confirmar}.
\item El sistema realiza la acción.
\end{enumerate}


\\
\hline
\textbf{Secuencia Alternativa} &\mbox{}\par\vspace{-\baselineskip}
\begin{enumerate}[leftmargin=0.9cm, topsep=0.1cm]
\item[3.] El usuario pulsa el botón de \textit{Cancelar}.
	\begin{itemize}
	\item[1.] El sistema cancela la acción y detiene el proceso de eliminación.
	\end{itemize}
\end{enumerate}
\\

\hline
\textbf{Postcondiciones} & 
Los datos de la categoría quedan eliminados del sistema.\\
\hline
\end{longtable}
