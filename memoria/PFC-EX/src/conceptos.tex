\chapter[Conceptos Previos]{
  \label{chp:conceptos}
  Conceptos Previos
}
\minitoc
\newpage

En los últimos años, la información geográfica -también llamada información
geoespacial- a cobrado una gran importancia. La facilidad -y necesidad- de
identificar el lugar donde se han obtenido los datos ha provocado la
aparición de nuevos términos. A continuación se presenta una serie de
definiciones que ayudarán a comprender el dominio y alcance de este proyecto.


\section{Coordenadas Geográficas}

Para determinar la posición de un objeto en el globo terrestre se emplean
coordenadas geográficas, un sistema de referencia que utiliza  dos
coordenadas angulares: latitud (eje Norte-Sur) y longitud (eje Este-Oeste)
y una tercera coordenada elevación para indicar la altitud.

\begin{description}
 \item[Latitud] es el ángulo que existe entre un punto cualquiera y el
 Ecuador, medida sobre el meridiano que pasa por dicho punto. En cartografía,
 la latitud se suele expresar en grados sexagesimales. Todos los puntos
 ubicados sobre el mismo paralelo tienen la misma latitud. Aquellos que se
 encuentran al norte del Ecuador reciben la denominación Norte (N), los que se
 encuentran al sur del Ecuador reciben la denominación Sur (S).
 El rango de valores es de 0º para el Ecuador a 90º para los polos Norte y Sur
 con latitud 90º N y 90º S respectivamente.
 \item[Longitud] es el ángulo a lo largo del ecuador desde cualquier punto de
 la Tierra. Por convenio, está aceptado internacionalmente que Greenwich en Londres
 es la longitud 0. Las líneas de longitud son círculos máximos que pasan por los
 polos y se llaman meridianos. La longitud también se expresa en grados
 sexagesimales entre 0º y 180º indicando a qué hemisferio (occidental W —del
 inglés \textit{West}— y oriental E —\textit{East}—) o también entre 0º y 180º positivos
 indicando Este, o negativos indicando hacia el Oeste.
 \item[Elevación] o \textbf{altitud} es la distancia vertical de un punto de la Tierra
 respecto al nivel del mar, llamada elevación sobre el nivel medio del mar, en contraste
 con la altura, que indica la distancia vertical existente entre dos puntos de la
 superficie terrestre.
\end{description}

Combinando estos dos ángulos (latitud y longitud) junto con la elevación, es posible
expresar la posición de cualquier punto de la superficie de la Tierra.
