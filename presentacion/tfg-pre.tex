\documentclass{beamer}
 
\usepackage[utf8]{inputenc}
\usetheme{Madrid}
\usepackage{setspace}
\usepackage[spanish]{babel}
 
 
\title[Trabajo Fin de Grado] %optional
{Aplicación para la planificación y gestión de viajes turísticos}
 
\subtitle{Trabajo Fin de Grado}
 
\author{
Rodrigo Dopazo Iglesias
}

 
\institute
{Grado en Ingeniería Informática\\
Mención en Tecnologías de la Información
\and
Universidad da Coruña\\
Facultad de Informática
}
 
\date
{\today}

\logo{\includegraphics[height=0.5cm]{./img/logo.png}}
 
 
 
\begin{document}

\frame{\titlepage}
 
\begin{frame}
\setlength{\baselineskip}{18pt}
\frametitle{Introducción}
\begin{center}
\textit{¿En qué consiste el trabajo?\\
¿Qué finalidad tiene?\\
¿A quién está dirigido?\\}
\end{center}
\end{frame}


\begin{frame}
\frametitle{Demo}
\begin{center}
\includegraphics[height=5cm]{./img/video.png}
\end{center}
\end{frame}



\begin{frame}
\frametitle{Índice}
\tableofcontents
\end{frame}


\section{Objetivos}
\begin{frame}
\frametitle{Objetivos}
La realización del proyecto exige la consecución de los siguientes objetivos:
\\
\
\begin{enumerate}
 \item<1-> \textbf{Planificar rutas}
 \item<2-> \textbf{Acceso datos externos}
 \item<3-> \textbf{Gestionar Eventos y Usuarios}
 \item<4-> \textbf{Aplicación móvil y aplicación web}
 \item<5-> \textbf{Datos en tiempo real}
\end{enumerate}
\end{frame}



\section{Análisis de Alternativas}
\begin{frame}
\frametitle{Análisis de Alternativas}

\begin{itemize}
 \item Foursquare App
 \item Google Trips: Travel Planner
 \item Sygic Travel
 \item Visit a City
\end{itemize}


\vspace{0.5cm}
\begin{center}
\includegraphics[height=2cm]{./img/FoursquareApp.png}
\includegraphics[height=2cm]{./img/GoogleTrips.png}
\includegraphics[height=2cm]{./img/SygicTravel.png}
\includegraphics[height=2cm]{./img/VisitaCity.png}

\end{center}
\end{frame}



\section{Tecnologías de Desarrollo}
\begin{frame}
\frametitle{Tecnologías de Desarrollo}
\begin{center}
\includegraphics[height=1.5cm, angle=10]{./img/java.png}
\includegraphics[height=1cm]{./img/maven.png}
\includegraphics[height=1.5cm, angle=8]{./img/spring.png}
\includegraphics[height=1cm, angle=-5]{./img/ionic.png}
\includegraphics[height=1.5cm, angle=-5]{./img/jaxrs.png}
\includegraphics[height=1.5cm, angle=-5]{./img/htmlcssjs.png}
\includegraphics[height=1cm, angle=5]{./img/ajax.png}
\includegraphics[height=0.5cm, angle=5]{./img/thymeleaf.png}

\includegraphics[height=1.5cm, angle=12]{./img/bootstrap.png}
\includegraphics[height=1cm]{./img/oracledb.png}
\includegraphics[height=1.5cm, angle=-5]{./img/tomcat.png}

\end{center}
\end{frame}

\section{Metodología y Planificación}

\begin{frame}
\frametitle{Metodología y Planificación}
\begin{itemize}
\item Se ha optado por seguir una metodología basada en el \textbf{Proceso Unificado}.
\item Se busca obtener un producto final de calidad, dividiendo el trabajo en pequeños proyectos (Iteraciones) que incrementarán y mejorarán el producto. 
\begin{itemize}
\item Flujos básico de trabajo. \textit{Análisis, Diseño, Implementación y Pruebas}.
\item División en tareas.
\item Planificación en tiempo, coste y esfuerzo.
\end{itemize}
\end{itemize}
\end{frame}


\begin{frame}
\frametitle{Metodología y Planificación}
\textbf{Iteraciones}:
\begin{itemize}
\item Flujos básico de trabajo. \textit{Análisis, Diseño, Implementación y Pruebas}.
\item División en tareas.
\item Planificación en tiempo, coste y esfuerzo.
\end{itemize}
\end{frame}



\section{Flujos de Trabajo}
\begin{frame}


\end{frame}

\section{Conclusiones y Líneas Futuras}
\begin{frame}

\end{frame}
 
\end{document}